%%% This file contains definitions of various useful macros and environments %%%
%%% Please add more macros here instead of cluttering other files with them. %%%

%%% Minor tweaks of style

% These macros employ a little dirty trick to convince LaTeX to typeset
% chapter headings sanely, without lots of empty space above them.
% Feel free to ignore.
\makeatletter
%\def\@makechapterhead#1{
%  {\parindent \z@ \raggedright \normalfont
%   \Huge\bfseries \thechapter. #1
%   \par\nobreak
%   \vskip 20\p@
%}}

%\def\@makechapterhead#1{
%  
%  \parindent\z@\normalfont
%  %\fontsize{32}{36}\selectfont
%  \raggedright%
%  \Huge\bfseries #1 
%  %\par\nobreak
%  %\bgroup
%  %\moveright-1in\vbox to 0pt{
%  	\hbox{}\vskip4pt\fontsize{86}{68}\selectfont
%  	\textcolor{sussexFlintCol}{\thechapter}\par%\vss
%  %	}%
%  
%  \vskip36pt
%}

\def\thickhrulefill{\leavevmode \leaders \hrule height 1.2ex \hfill \kern \z@}

\def\@makechapterhead#1{%  
\thispagestyle{empty}%  
%\vspace*{50\p@}%
%\vspace*{10\p@}%  
{\parindent \z@ 
    \reset@font
    %\fontsize{86}{68}
    \Huge
    \scshape\selectfont\textcolor{sussexFlintCol}{\@chapapp{} \thechapter}
    \par\nobreak
    %%
    %\vskip 80\p@
    }
    \begin{tikzpicture}[remember picture, overlay]
      \draw let \p1 = (current page.west), \p2 = (current page.east) in
      node (A) [minimum width=\x2-\x1, minimum height=2.7cm, draw, rectangle, fill=sussexFlintCol, anchor=north west] at ([yshift=-35mm]$(current page.north west)$) {}
      node[anchor=west, align=flush left, text width=\x2-\x1-7cm, text=sussexGreyCol] at ([xshift=35mm]A.west) {\fontsize{27}{40}\textbf{#1}\par};
    \end{tikzpicture}
    \vspace*{100\p@}
    }
%\fontsize{27}{32}\bfseries #1   
%\node[anchor=south east] (fig) at ([xshift=-10mm,yshift=-10mm]A.east) {\includegraphics[width=0.4\textwidth]{#2}};

\def\@makeschapterhead#1{
  {\parindent \z@ \raggedright \normalfont
   \Huge\bfseries #1
   \par\nobreak
   \vskip 20\p@
}}
\makeatother

\definecolor{sussexFlintCol}{RGB}{0, 59, 73}
\definecolor{sussexBlueCol}{RGB}{29, 66, 137}
\definecolor{sussexGreyCol}{RGB}{214, 210, 196}

% This macro defines a chapter, which is not numbered, but is included
% in the table of contents.
\def\chapwithtoc#1{
\chapter*{#1}
\addcontentsline{toc}{chapter}{#1}
}

% Draw black "slugs" whenever a line overflows, so that we can spot it easily.
\overfullrule=1mm

%%% Set personal note commands
\newcommand{\todo}[1]{\textcolor{red!90!black}{TO DO: \textit{#1}}}
\newcommand{\note}[1]{\textcolor{green!70!black}{COMMENT: \textit{#1}}}

%%% Macros for definitions, theorems, claims, examples, ... (requires amsthm package)

\theoremstyle{plain}
\newtheorem{thm}{Theorem}
\newtheorem{lemma}[thm]{Lemma}
\newtheorem{claim}[thm]{Claim}

\theoremstyle{plain}
\newtheorem{defn}{Definition}

\theoremstyle{remark}
\newtheorem*{cor}{Corollary}
\newtheorem*{rem}{Remark}
\newtheorem*{example}{Example}

%%% An environment for proofs

%%% FIXME %%% \newenvironment{proof}{
%%% FIXME %%%   \par\medskip\noindent
%%% FIXME %%%   \textit{Proof}.
%%% FIXME %%% }{
%%% FIXME %%% \newline
%%% FIXME %%% \rightline{$\square$}  % or \SquareCastShadowBottomRight from bbding package
%%% FIXME %%% }

%%% An environment for typesetting of program code and input/output
%%% of programs. (Requires the fancyvrb package -- fancy verbatim.)

\DefineVerbatimEnvironment{code}{Verbatim}{fontsize=\small, frame=single}

%%% The field of all real and natural numbers
\newcommand{\R}{\mathbb{R}}
\newcommand{\N}{\mathbb{N}}

%%% Useful operators for statistics and probability
\DeclareMathOperator{\pr}{\textsf{P}}
\DeclareMathOperator{\E}{\textsf{E}\,}
\DeclareMathOperator{\var}{\textrm{var}}
\DeclareMathOperator{\sd}{\textrm{sd}}

%%% Transposition of a vector/matrix
\newcommand{\T}[1]{#1^\top}

%%% Various math goodies
\newcommand{\goto}{\rightarrow}
\newcommand{\gotop}{\stackrel{P}{\longrightarrow}}
\newcommand{\maon}[1]{o(n^{#1})}
\newcommand{\abs}[1]{\left|{#1}\right|}
\newcommand{\dint}{\int_0^\tau\!\!\int_0^\tau}
\newcommand{\isqr}[1]{\frac{1}{\sqrt{#1}}}

%%% Various table goodies
\newcommand{\pulrad}[1]{\raisebox{1.5ex}[0pt]{#1}}
\newcommand{\mc}[1]{\multicolumn{1}{c}{#1}}

%%% Anti-neutrinos
\newcommand*{\anu}[1]{%
  \mathsurround=0pt\overline{\raisebox{0pt}[1.2\height]{#1}}%
}
