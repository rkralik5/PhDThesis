%%% The main file. It contains definitions of basic parameters and includes all other parts.

%% Settings for single-side (simplex) printing
% Margins: left 40mm, right 20mm, top and bottom 25mm
% (but beware, LaTeX adds 1in implicitly)
\documentclass[12pt,a4paper]{report}
\setlength\textwidth{145mm}
\setlength\textheight{247mm}
\setlength\oddsidemargin{10mm}
\setlength\evensidemargin{10mm}
\setlength\topmargin{-10mm}
%\setlength\headsep{0mm}
\setlength\headsep{10mm}
\setlength\headheight{0mm}
% \openright makes the following text appear on a right-hand page
\let\openright=\clearpage

%% Settings for two-sided (duplex) printing
% \documentclass[12pt,a4paper,twoside,openright]{report}
% \setlength\textwidth{145mm}
% \setlength\textheight{247mm}
% \setlength\oddsidemargin{14.2mm}
% \setlength\evensidemargin{0mm}
% \setlength\topmargin{0mm}
% \setlength\headsep{0mm}
% \setlength\headheight{0mm}
% \let\openright=\cleardoublepage

% Need to load scolor before pdfx otherwise it complains
\usepackage[dvipsnames,table,xcdraw]{xcolor} % typesetting in color

%% Generate PDF/A-2u
\usepackage[a-2u]{pdfx}

%% Character encoding: usually latin2, cp1250 or utf8:
\usepackage[utf8]{inputenc}

%% Prefer Latin Modern fonts
%\usepackage{lmodern}
%\usepackage[sc]{mathpazo}
\usepackage{libertine}
%\usepackage[libertine,cmintegrals,cmbraces,vvarbb]{newtxmath}

%% Further useful packages (included in most LaTeX distributions)
\usepackage{amsmath}        % extensions for typesetting of math
\usepackage{amssymb}
\usepackage{amsfonts}       % math fonts
\usepackage{braket}         % bra-ket notation
\usepackage{siunitx}        % units
\usepackage{amsthm}         % theorems, definitions, etc.
\usepackage{bbding}         % various symbols (squares, asterisks, scissors, ...)
\usepackage{float}          % for placing float with H
\usepackage{bm}             % boldface symbols (\bm)
\usepackage{rotating}       % for sideways figures
\usepackage{graphicx}       % embedding of pictures
\usepackage{units}          % another units package
\usepackage{gensymb}        % for the degree symbol
\usepackage{subcaption}
\usepackage{hyphenat}       % for allowing automatic hyphenation of hyphenated words
\usepackage{setspace}       % for setting line spacing
\usepackage{fancyvrb}       % improved verbatim environment
\usepackage{mathtools}      % introduces rcases for example
\usepackage[numbers,sort&compress]{natbib}% citation style AUTHOR (YEAR), or AUTHOR [NUMBER]
\usepackage[nottoc]{tocbibind} % makes sure that bibliography and the lists
			                   % of figures/tables are included in the table
			                   % of contents
\usepackage{dcolumn}        % improved alignment of table columns
\usepackage{booktabs}       % improved horizontal lines in tables
\usepackage{paralist}       % improved enumerate and itemize
%\usepackage[table]{xcolor}  % typesetting in color
\usepackage{tabularx}       % for title setting
\usepackage{placeins}       % for FloatBarrier
\usepackage{tikz}           % for tikz pictures
\usepackage{multirow}       % for multirows in tables (duh)
\usepackage{doi}            % automatically link doi to hyperlinks
\usepackage{url}            % use normal font for the URL in bibliography
\usepackage{tikz-feynman}   % for feynman diagrams
\usepackage{breqn}          % Equation breaking
\usepackage{contour}
\usepackage[absolute,overlay]{textpos}
\usepackage{slashed}        % Dirac notation
%\usepackage{cooltooltips}   % for links showing when hovering above
\usetikzlibrary{calc}       % for coloured header
\usetikzlibrary{decorations.pathreplacing} % for curly braces

%\usepackage[dvipsnames,table,xcdraw]{xcolor} % typesetting in color

%\usepackage[xindy]{glossaries} % for automatic glossary
\usepackage[acronym,toc,hyperfirst=false,nogroupskip]{glossaries} % for automatic glossary
%\glossarystyle{list} % by order of appearance
%\glossarystyle{listgroup} % alphabetically
%\glossarystyle{listhypergroup} % alphabetically and adds links to the letters
\setacronymstyle{long-sc-short}
\newacronym{SM}{SM}{Standard Model}
\newacronym{BSM}{BSM}{Beyond Standard Model}
\newacronym{PMNS}{PMNS}{Pontecorvo-Maki-Nakagawa-Sakata}
\newacronym{Fermilab}{Fermilab}{Fermi National Accelerator Laboratory}
\newacronym[description=Charge conjugation - Parity (symmetry)]{CP}{CP}{Charge conjugation - Parity}
\newacronym{QED}{QED}{Quantum Electro Dynamics}
\newacronym{2p2h}{2p2h}{two particle - two hole}
\newacronym{SK}{SK}{Super-Kamiokande}
\newacronym{SNO}{SNO}{Sudbury Neutrino Observatory}
\newacronym{MSW}{MSW}{Mikheyev-Smirnov-Wolfenstein}
\newacronym{LBL}{LBL}{Long Baseline}
\newacronym[description=NuMI Off-axis $\nu_e$ Appearance (experiment)]{NOvA}{NOvA}{NuMI Off-axis $\nu_e$ Appearance}
\newacronym[description=Tokai to Kamioka (experiment)]{T2K}{T2K}{Tokai to Kamioka}
\newacronym{DUNE}{DUNE}{Deep Underground Neutrino Experiment}
\newacronym{HK}{HK}{Hyper-Kamiokande}
\newacronym{NP}{NP}{New Physics}
\newacronym{HNL}{HNL}{Heavy Neutral Lepton}
\newacronym{NuMI}{NuMI}{Neutrinos from the Main Injector}
\newacronym{MI}{MI}{Main Injector}
\newacronym{POT}{POT}{Protons On Target}
\newacronym[description=Forward Horn Current (neutrino mode)]{FHC}{FHC}{Forward Horn Current}
\newacronym[description=Reverse Horn Current (antineutrino mode)]{RHC}{RHC}{Reverse Horn Current}
\newacronym{CC}{CC}{Charged Current}
\newacronym{ND}{ND}{Near Detector}
\newacronym{FD}{FD}{Far Detector}
\newacronym{NDOS}{NDOS}{Near Detector on the Surface}
\newacronym{TB}{TB}{Test Beam}
\newacronym{PVC}{PVC}{Polyvinyl chloride}
\newacronym[description=Wavelength Shifting (fibre)]{WLS}{WLS}{Wavelength Shifting}
\newacronym{APD}{APD}{Avalanche Photodiode}
\newacronym{FEB}{FEB}{Front End Board}
\newacronym{DAQ}{DAQ}{Data Acquisition}
\newacronym{ASIC}{ASIC}{Application-Specific Integrated Circuit}
\newacronym{ADC}{ADC}{Analog-to-Digital Converter}
\newacronym{FPGA}{FPGA}{Field Programmable Gate Array}
\newacronym{PE}{PE}{Photo Electron}
\newacronym{DCM}{DCM}{Data Concentration Module}
\newacronym{MC}{MC}{Monte Carlo}
\newacronym{PPFX}{PPFX}{Package to Predict the Flux}
\newacronym[description=Main Injector Particle Production (experiment)]{MIPP}{MIPP}{Main Injector Particle Production}
\newacronym[description=Quasi Elastic (interaction)]{QE}{QE}{Quasi-Elastic}
\newacronym{Res}{Res}{Resonant baryon production}
\newacronym{DIS}{DIS}{Deep Inelastic Scattering}
\newacronym{CEvNS}{CEvNS}{Coherent Elastic $\nu$-Nucleus Scattering}
\newacronym[description=Coherent \ensuremath{\pi} (production)]{COHpi}{COH\ensuremath{\pi}}{Coherent \ensuremath{\pi}}
\newacronym{MEC}{MEC}{Meson Exchange Current}
\newacronym{FSI}{FSI}{Final State Interaction}
\newacronym{CMC}{CMC}{Comprehensive Model Configuration}
\newacronym{NC}{NC}{Neutral Current}
\newacronym{BPF}{BPF}{Break Point Fitter}
\newacronym{ML}{ML}{Machine Learning}
\newacronym{PID}{PID}{Particle Identification}
\newacronym{CNN}{CNN}{Convolutional Neural Network}
\newacronym{CVN}{CVN}{Convolutional Visual Network}
\newacronym{BDT}{BDT}{Boosted Decision Tree}
\newacronym{ReMId}{ReMId}{Reconstructed Muon Identifier}
\newacronym{MIP}{MIP}{Minimum Ionising Particle}
\newacronym{FB}{FB}{Fibre Brightness}
\newacronym{LOWESS}{LOWESS}{Locally Weighted Scatter plot Smoothing}
\newacronym{PECorr}{PECorr}{Corrected Photo Electrons}
\newacronym{MEU}{MEU}{Muon Energy Unit}
\newacronym{PCA}{PCA}{Principal Component Analysis}
\newacronym{FTBF}{FTBF}{Fermilab Test Beam Facility}
\newacronym{MWPC}{MWPC}{Multiwire Proportional Chamber}
\newacronym{ToF}{ToF}{Time of Flight}
\newacronym{CRY}{CRY}{Cosmic-Ray Shower Generator}
\newacronym{nuone}{\ensuremath{\nu}-on-e}{neutrino-on-electron}
\newacronym{LDM}{LDM}{Light Dark Matter}
\newacronym{FOM}{FOM}{Figure Of Merit}
\makeglossaries
%\glsdisablehyper

\setstretch{1.5}            % Set line spacing

%% Set page numbers and section titles in the header
\usepackage{fancyhdr}
\fancypagestyle{plain}{
  \fancyhf{}
  \chead{\thepage}
  \renewcommand{\headrulewidth}{0pt}
}

%\hfuzz=99pt  %Removing warning messages and ugly black rectangles at overflown text

%%% Basic information on the thesis

% Thesis title in English (exactly as in the formal assignment)
\def\ThesisTitle{Measuring the Muon Neutrino Magnetic Moment in the NOvA Near Detector}

% Author of the thesis
\def\ThesisAuthor{Róbert Králik}

% Loaction of the University
\def\DeptLocation{Brighton, United Kingdom}

% Year when the thesis is submitted
\def\MonthAndYearSubmitted{April 2024}

% Name of the University
\def\Department{School of Mathematical and Physical Sciences}
\def\University{University of Sussex}

% Is it a department (katedra), or an institute (ústav)?
\def\DeptType{Institute}

% Thesis supervisor: name, surname and titles
\def\Supervisor{Dr.~Lily~Asquith}
\def\SecondSupervisor{Prof.~Jeffrey~Hartnell}

% An optional dedication: you can thank whomever you wish (your supervisor,
% consultant, a person who lent the software, etc.)
\def\Acknowledgements{%
%In the introduction to your thesis, you should set out the sources of your information, such as particular libraries, archives, organisational records, private papers and department files
%You should also set out the plan of your research procedures, indicating what general categories of persons you interviewed and you should indicate any special conditions of access to information.
}

% Abstract (recommended length around 80-200 words; this is not a copy of your thesis assignment!)
\def\Abstract{%
Measuring an enhanced neutrino magnetic moment would be a clear indication of physics beyond the Standard Model (BSM), shedding light on the correct BSM theory or the potential Majorana nature of neutrinos. It would manifest in the NOvA near detector as an excess of neutrino-on-electron elastic scattering interactions at low electron recoil energies. Leveraging an intense and highly pure muon neutrino beam, along with the finely segmented liquid scintillator detector technology specifically designed for electromagnetic shower separation, enables NOvA to achieve a potentially world-leading sensitivity in probing the effective muon neutrino magnetic moment. Despite facing statistical limitations stemming from the low cross section of the signal process, systematic uncertainties have a significant impact on this result. To address these challenges, the NOvA Test Beam experiment focuses on mitigating some of the largest systematic uncertainties within NOvA by investigating particle interactions and energy deposition in a small-scale replica NOvA detector. This thesis describes the calibration of the NOvA Test Beam detector, which is a crucial step in analysing the Test Beam data before they can be utilised to reduce NOvA systematic uncertainties.
}

% 3 to 5 keywords (recommended), each enclosed in curly braces
\def\Keywords{%
{neutrino} {NOvA} {electromagnetic} {testbeam} {calibration}
}

%% The hyperref package for clickable links in PDF and also for storing
%% metadata to PDF (including the table of contents).
%% Most settings are pre-set by the pdfx package.
\hypersetup{unicode}
\hypersetup{breaklinks=true}
\hypersetup{hidelinks}        %hides those boxes around references
\hypersetup{
    colorlinks,
    citecolor=sussexBlueCol,
    filecolor=black,
    linkcolor=sussexBlueCol,
    urlcolor=sussexBlueCol
}

% Definitions of macros (see description inside)
%%% This file contains definitions of various useful macros and environments %%%
%%% Please add more macros here instead of cluttering other files with them. %%%

%%% Minor tweaks of style

% These macros employ a little dirty trick to convince LaTeX to typeset
% chapter headings sanely, without lots of empty space above them.
% Feel free to ignore.
\makeatletter
%\def\@makechapterhead#1{
%  {\parindent \z@ \raggedright \normalfont
%   \Huge\bfseries \thechapter. #1
%   \par\nobreak
%   \vskip 20\p@
%}}

%\def\@makechapterhead#1{
%  
%  \parindent\z@\normalfont
%  %\fontsize{32}{36}\selectfont
%  \raggedright%
%  \Huge\bfseries #1 
%  %\par\nobreak
%  %\bgroup
%  %\moveright-1in\vbox to 0pt{
%  	\hbox{}\vskip4pt\fontsize{86}{68}\selectfont
%  	\textcolor{sussexFlintCol}{\thechapter}\par%\vss
%  %	}%
%  
%  \vskip36pt
%}

\def\thickhrulefill{\leavevmode \leaders \hrule height 1.2ex \hfill \kern \z@}
\def\@makechapterhead#1{%  
\thispagestyle{empty}%  
%\vspace*{50\p@}%
%\vspace*{10\p@}%  
{\parindent \z@ 
    \reset@font
    %\fontsize{86}{68}
    \Huge
    \scshape\selectfont\textcolor{sussexFlintCol}{\@chapapp{} \thechapter}
    \par\nobreak
    %%
    %\vskip 80\p@
    }
    \begin{tikzpicture}[remember picture, overlay]
      \draw let \p1 = (current page.west), \p2 = (current page.east) in
      node (A) [minimum width=\x2-\x1, minimum height=2.7cm, draw, rectangle, fill=sussexFlintCol, anchor=north west] at ([yshift=-35mm]$(current page.north west)$) {}
      node[anchor=west, align=flush left, text width=\x2-\x1-7cm, text=sussexGreyCol] at ([xshift=35mm]A.west) {\fontsize{27}{32}\bfseries #1};
    \end{tikzpicture}
    \vspace*{100\p@}
    }
    
%\node[anchor=south east] (fig) at ([xshift=-10mm,yshift=-10mm]A.east) {\includegraphics[width=0.4\textwidth]{#2}};

\def\@makeschapterhead#1{
  {\parindent \z@ \raggedright \normalfont
   \Huge\bfseries #1
   \par\nobreak
   \vskip 20\p@
}}
\makeatother

\definecolor{sussexFlintCol}{RGB}{0, 59, 73}
\definecolor{sussexBlueCol}{RGB}{29, 66, 137}
\definecolor{sussexGreyCol}{RGB}{214, 210, 196}

% This macro defines a chapter, which is not numbered, but is included
% in the table of contents.
\def\chapwithtoc#1{
\chapter*{#1}
\addcontentsline{toc}{chapter}{#1}
}

% Draw black "slugs" whenever a line overflows, so that we can spot it easily.
\overfullrule=1mm

%%% Set personal note commands
\newcommand{\todo}[1]{\textcolor{red!90!black}{TO DO: \textit{#1}}}
\newcommand{\note}[1]{\textcolor{green!70!black}{COMMENT: \textit{#1}}}

%%% Macros for definitions, theorems, claims, examples, ... (requires amsthm package)

\theoremstyle{plain}
\newtheorem{thm}{Theorem}
\newtheorem{lemma}[thm]{Lemma}
\newtheorem{claim}[thm]{Claim}

\theoremstyle{plain}
\newtheorem{defn}{Definition}

\theoremstyle{remark}
\newtheorem*{cor}{Corollary}
\newtheorem*{rem}{Remark}
\newtheorem*{example}{Example}

%%% An environment for proofs

%%% FIXME %%% \newenvironment{proof}{
%%% FIXME %%%   \par\medskip\noindent
%%% FIXME %%%   \textit{Proof}.
%%% FIXME %%% }{
%%% FIXME %%% \newline
%%% FIXME %%% \rightline{$\square$}  % or \SquareCastShadowBottomRight from bbding package
%%% FIXME %%% }

%%% An environment for typesetting of program code and input/output
%%% of programs. (Requires the fancyvrb package -- fancy verbatim.)

\DefineVerbatimEnvironment{code}{Verbatim}{fontsize=\small, frame=single}

%%% The field of all real and natural numbers
\newcommand{\R}{\mathbb{R}}
\newcommand{\N}{\mathbb{N}}

%%% Useful operators for statistics and probability
\DeclareMathOperator{\pr}{\textsf{P}}
\DeclareMathOperator{\E}{\textsf{E}\,}
\DeclareMathOperator{\var}{\textrm{var}}
\DeclareMathOperator{\sd}{\textrm{sd}}

%%% Transposition of a vector/matrix
\newcommand{\T}[1]{#1^\top}

%%% Various math goodies
\newcommand{\goto}{\rightarrow}
\newcommand{\gotop}{\stackrel{P}{\longrightarrow}}
\newcommand{\maon}[1]{o(n^{#1})}
\newcommand{\abs}[1]{\left|{#1}\right|}
\newcommand{\dint}{\int_0^\tau\!\!\int_0^\tau}
\newcommand{\isqr}[1]{\frac{1}{\sqrt{#1}}}

%%% Various table goodies
\newcommand{\pulrad}[1]{\raisebox{1.5ex}[0pt]{#1}}
\newcommand{\mc}[1]{\multicolumn{1}{c}{#1}}

%%% Anti-neutrinos
\newcommand*{\anu}[1]{%
  \mathsurround=0pt\overline{\raisebox{0pt}[1.2\height]{#1}}%
}


% Title page and various mandatory informational pages
\begin{document}

%\glsaddall
%\glsresetall

%%% Each chapter is kept in a separate file
%%%% Title page of the thesis and other mandatory pages

%%% Title page of the thesis

\pagestyle{empty}
\hypersetup{pageanchor=false}
\begin{center}

\centerline{\mbox{\includegraphics[width=100mm]{/home/robert/Templates/Images/sussex_logo_color_on_white.png}}}

%\vspace{-8mm}
\vfill

{\LARGE\bfseries\ThesisTitle}

\vfill

{\LARGE\ThesisAuthor}

\end{center}

\vfill

\begin{spacing}{2}
\begin{large}
\noindent
Supervisor of the doctoral thesis: \Supervisor \\
Second supervisor of the doctoral thesis: \SecondSupervisor \\
Submitted for the degree of Doctor of Philosophy \\
University of Sussex \\
% Year
Brighton, UK, \MonthAndYearSubmitted
\end{large}
\end{spacing}

%Submitted for the degree of Doctor of Philosophy

%University of Sussex

% Year
%Brighton, UK, \MonthAndYearSubmitted

\newpage

%%% A page with a solemn declaration to the master thesis

\openright
\hypersetup{pageanchor=true}
\pagestyle{plain}
\pagenumbering{roman}
\vglue 0pt plus 1fill

\noindent
I hereby declare that I carried out this thesis independently, and only with the cited sources, literature and other professional sources.

\medskip\noindent
I also declare that this thesis has not been and will not be, submitted in whole or in part to another University for the award of any other degree.

\vspace{15mm}

\noindent Signature:\\

\hbox{\hbox to 0.5\hsize{%
In ................... date ..............	% FIXME!
\hss}\hbox to 0.5\hsize{%
\ThesisAuthor
\hss}}

\vspace{20mm}
\newpage

%%% Dedication

\openright

\noindent
\Dedication

\newpage

%%% Mandatory information page of the thesis

\openright

\vbox to 0.5\vsize{
\setlength\parindent{0mm}
\setlength\parskip{0mm}

Title:
\ThesisTitle

Author:
\ThesisAuthor

\DeptType:
\Department

Supervisor:
\Supervisor

Second supervisor:
\SecondSupervisor

Abstract:
\Abstract

Keywords:
\Keywords

\vss}

\newpage

\openright
\pagestyle{plain}
\pagenumbering{arabic}
\setcounter{page}{1}

%\chapter*{Introduction}\label{sec:Introduction}
\chapter{Theory of neutrino physics}\label{sec:NeutrinoTheory}

Very brief history - Pauli, Fermi,...
Fermi was the first to use them in his beta decay theory, after Pauli proposed them in his letter. First time detected by Reines and Cowan in 1956.

%[nuMM/nuElmagInt2015.pdf] However, there was no sign of a neutrino mass. After the discovery of parity violation in 1957, Landau (1957), Lee and Yang (1957), and Salam (1957) proposed the two-component theory of massless neutrinos, in which a neutrino is described by a Weyl spinor and there are only left-handed neutrinos and right-handed antineutrinos. It was, however, clear (Case, 1957; Mclennan, 1957; Radicati and Touschek, 1957) that two-component neutrinos could be massive Majorana fermions and that the two-component theory of a massless neutrino is equivalent to the Majorana theory in the limit of zero neutrino mass. The two-component theory of massless neutrinos was later incorporated in the standard model of Glashow (1961), Weinberg (1967), and Salam (1969), in which neutrinos are massless and have only weak interactions. In the standard model Majorana neutrino masses are forbidden by the $\textsf{SU}\left(2\right)_L\times \textsf{U}\left(1\right)_{\gamma}$ symmetry.

\section{Neutrinos in the Standard Model}
Neutrinos are fermions, their interactions are...

%[nuMM/nuElmagInt2015.pdf] In the standard model of electroweak interactions (Glashow, 1961; Weinberg, 1967; Salam, 1969), neutrinos are described by two-component massless left-handed Weyl spinors (Giunti and Kim, 2007). The masslessness of neutrinos is due to the absence of right-handed neutrino fields, without which it is not possible to have Dirac mass terms, and to the absence of Higgs triplets, without which it is not possible to have Majorana mass terms.

\section{Neutrinos beyond the Standard Model}
Neutrino oscillate and therefore have mass.

%[nuMM/nuElmagInt2015.pdf] We now know that neutrinos are massive, because many experiments observed neutrino oscillations (Giunti and Kim, 2007; Bilenky, 2010; Xing and Zhou, 2011; Beringer et al., 2012; Gonzalez-Garcia et al., 2012; Bellini et al., 2014), which are generated by neutrino masses and mixing (Pontecorvo, 1957, 1958, 1968; Maki, Nakagawa, and Sakata, 1962). Therefore, the standard model must be extended to account for the neutrino masses. There are many possible extensions of the standard model that predict different properties for neutrinos (Ramond, 1999; Mohapatra and Pal, 2004; Xing and Zhou, 2011). Among them, most important is their fundamental Dirac or Majorana character. In many extensions of the standard model neutrinos also acquire electromagnetic properties through quantum loop effects which allow direct interactions of neutrinos with electromagnetic fields and electromagnetic interactions of neutrinos with charged particles.

Theories of neutrino mass generation

I should discuss everything that is even briefly mentioned in the neutrino magnetic moment theory section.
\begin{itemize}
\item Dirac vs Majorana neutrinos
\item Neutrino masses
\item Neutrino interactions with electrons and nuclei
\item Neutrino oscillations and their implications
\end{itemize}
%\chapter{The NOvA experiment}\label{sec:NOvA}
%%% OVERVIEW OF THE NOvA EXPERIMENT %%%

The NuMI Off-axis $\nu_e$ Appearance (NOvA) experiment \cite{NOvAWebsite} is a long-baseline neutrino oscillation experiment based at the Fermi National Accelerator Laboratory (Fermilab) \cite{FNALWebsite}. NOvA receives an off-axis $\nu_\mu$ and $\overline{\nu}_\mu$ beam from Fermilab's NuMI neutrino source, described in Sec.~\ref{fig:NOvANuMI}, and measures $\nu_e$/$\overline{\nu}_e$ appearance and $\nu_\mu$/$\overline{\nu}_\mu$ disappearance between its two highly active and finely segmented detectors, described in Sec.~\ref{fig:NOvADetectors} \cite{PhysicsOfNOvA.pdf}. 

The capability to measure both $\nu_e$ and $\overline{\nu}_e$ appearance, coupled with a significant matter effect induced by the long baseline, allows NOvA to address some of the most important questions in neutrino physics to date, such as the neutrino mass ordering, the octant of $\theta_{23}$, and the possible CP violation in the neutrino sector \cite{PhysicsOfNOvA.pdf,NOvAStatusAndOutlook.pdf,FirstNOvAResult.pdf,2019NOvAFHCRHCResults.pdf,NOvAResults2021.pdf}. NOvA data also enables measurements of the values of $\theta_{13}$, $\theta_{23}$ and $\left|\Delta m^2_{atm}\right|$ \cite{PhysicsOfNOvA.pdf}, measurements of neutrino differential cross sections in the near detector \cite{NOvANCPi0XSecMeasurement2019.pdf, NOvANumuCCXSexMeasurement2023.pdf, NOvANueCCXSecMeasurement2023.pdf, NOvANuMuCCPi0XSecMeasurement2023.pdf}, constraints on the possible sterile neutrino models \cite{NOvASterilesFHCResults2017.pdf, NOvASterilesFHCRHCResults2021.pdf}, monitoring for supernova neutrino activity \cite{NOvASupernovaMeasurements2020.pdf, NOvASupernovaCoincidenceMeasurements2021.pdf}, searches for magnetic monopoles \cite{NOvASlowMagMonopoles2021.pdf}, and constraints on the neutrino electromagnetic properties (this thesis). Using two functionally identical detectors mitigates the systematic uncertainties of neutrino oscillation measurements, described in Sec.~\ref{sec:NOvASystematics}.

%\note{Should I mention here when did NOvA start taking data and how long is it planning to run for? Maybe future analysis? Size of the collaboration? - YES!}

NOvA started taking data in February 2014 and is expected to run through 2026 \cite{NOvAHalfTimeOverview2022.pdf}.
\todo{Add DUNE into this and find the LOI reference}

%From NOvAHalfTimeOverview2022.pdf: NOvA started physics data-taking with the first 5 kt of the far detector in February 2014, and saw the completion of the near and far detectors completed later that year. Annual beam exposure, measured in protons-on-target (POT) to NuMI ramped up as the design beam power for NOvA of 700 kW was achieved in 2017. Further improvements to the NuMI target system now allow even higher power, with a record 1 hour average power of 843 kW. As of May 2021, the far detector has recorded data for nearly 17 × 1020 POT delivered to NuMI in neutrino mode, when weighting data collected during construction for the fraction of the detector active at the time, and 12.7 × 1020 POT delivered in antineutrino mode. NOvA expects to continue data-taking until 2026 and hopes to double the current exposure in both neutrinos and antineutrinos.
%NOvA is expected to take data through 2026 [67], when the Fermilab accelerator complex is shutdown for construction of the Long-Baseline Neutrino Facility (LBNF) for DUNE. During the remaining running time, the beam power delivered to NuMI should increase. Following installation of additional beam dampers and collimators scheduled in 2023-4 as part of the PIP-II project [68], the Fermilab accelerator complex should be capable of delivering more than 900 kW to NuMI. The ultimate exposure delivered to NuMI will depend on the timeline of the remaining power improvements, other demands on beam from the Main Injector, and the total run length. The current projection is between 60 and 70 × 1020 protons on target.
%With the NOvA test beam effort underway and ongoing improvements to neutrino interaction measurements and modeling, the sensitivity of NOvA to three-flavor oscillation parameters will remain statistics-limited through the end of the experiment [77]. The expected total beam exposure will bring additional compelling milestones into reach. For the Mass Hierarchy, NOvA will achieve 95\% a priori sensitivity for 40-60\% of possible δCP values and 4-5$\sigma$ a priori sensitivity for the most favorable combinations of the true values of the oscillation parameters. For CP-violation, a median, a priori sensitivity of 2-sigma is projected for 20-30\% of the $\delta$CP range. The physics reach of NOvA will be complimented by a joint analysis effort underway between NOvA and T2K [78].
%NOvA has already informed the design of the next generation of neutrino experiments from the insights gained from the performance of its beamline and detectors, and from its experience in operations and development of analysis techniques. NOvA has also informed the neutrino interaction and oscillation landscape with results from across its full portfolio of physics topics, and will continue to do so until the onset of the DUNE and T2HK era.

\section{The Neutrino Beam}\label{sec:NuMI}

The neutrino beam for NOvA comes from the Fermilab-based \textit{Neutrinos at the Main Injector} (NuMI) neutrino source \cite{NuMI.pdf}. The schematic description of NuMI is shown in Fig.~\ref{fig:NOvANuMI}, starting on the left hand side with $\unit[120]{GeV}$ protons from the Main Injector (MI), part of the Fermilab accelerator complex. The proton beam is divided into $\unit[10]{\mu s}$ long pulses, with $\sim5\times 10^{13}$ protons on target (POT) per spill every $\sim\unit[1.3]{s}$ long cycle time, resulting in a proton beam power of $\sim\unit[800]{kW}$, with upgrades currently underway to surpass $\unit[1]{MW}$ \cite{NuMIUpgradeToMWProceedings2022.pdf}.

\begin{figure}[!hbtp]
\centering
%is pdf-a
\includegraphics[width=\textwidth]{Plots/NOvAExperiment/BeamlineAlternative.jpg}
\caption[The schematic of the NuMI beam facility]{
The NuMI neutrino beam starts on the left hand side with protons from the Main Injector impinged on a graphite target producing mainly pions and kaons. These are then focused and charge-selected by two focusing horns, after which they decay inside the decay pipe into a high-purity $\nu_\mu$, or $\overline{\nu}\mu$ beam. The residual hadrons are stopped and monitored in the hadron absorber and the remaining muons are recorded with muon monitors and absorbed inside the rock. Figure from \cite{NuMI.pdf}.
%The schematic of the NuMI beam facility. The beam travels from left to right. The individual components shown are not to scale. Protons originate as $H^-$ ions, which are converted into protons in the Booster, sent to the main injector, where they are finally accelerated to $120\,\si{\giga\electronvolt}$, bent downward by $58\,\si{\milli\radian}$ and transported $350\,\si{\meter}$ to the $1.2\,\si{\meter}$ long NuMI target. The protons are incident on the graphite target and the produced hadrons are focused by two magnetic horns, located in about $40\,\si{\meter}$ long target hall, with about $19\,\si{\meter}$ separation between the two horns. Hadrons then enter a $675\,\si{\meter}$ long decay pipe made of steel, with $2\,\si{\meter}$ diameter, which serves as vacuum or low density environment for the mesons to propagate and decay into tertiary mesons, charged leptons and neutrinos. A hadron monitor is located at the end of the decay volume just in front of the $5\,\si{\meter}$ thick aluminium, steel and concrete absorber to record the profile of the residual hadrons. Of the particles interacting in the absorber, the principal component (approximately $80\%$) is the proton beam that has not interacted. The remainder are mainly mesons which have not decayed in the pipe or secondary protons. The absorber not only stops most of the particles still remaining in the beam but also acts as a shield against radiation. Muons and neutrinos deposit little or no energy in the absorber and continue into unexcavated rock with three muon monitors allowing measurement of the residual muon flux. The $240\,\si{\meter}$ of rock following the absorber stops the muons remaining in the beam but allows the neutrinos to pass\cite{numi}. 
}
\label{fig:NOvANuMI}
\end{figure}

The proton beam passes through a collimating baffle before hitting a $\sim\unit[1.2]{m}$-long (equal to about two interaction lengths) graphite target \cite{LEOFluxPredictionAtNuMI.pdf}, producing hadrons, predominantly pions and kaons \cite{NuMI.pdf}. These are then focused and selected by two parabolic magnetic "horns". The focused hadrons pass through a $\unit[675]{m}$-long decay pipe filled with helium to create a low density environment for hadrons to propagate and decay in flight into either neutrinos or antineutrinos. High energy hadrons that do not decay in the decay pipe are absorbed within a massive aluminium, steel, and concrete hadron absorber and monitored with a hadron monitor. The leftover muons are ranged out in dolomite rock after the absorber and monitored using three muon monitors. The hadron and all the muon monitors are ionization chambers, used to monitor the quality, location and relative intensity of the beam.

Using a positive current inside the horns focuses positively charged particles, which then decay into neutrinos, and removes negatively charged particles. Reversing the horn current focuses negatively charged particles, which decay into antineutrinos, and defocuses positively charged particles. The neutrino mode is therefore called Forward Horn Current (FHC) and the antineutrino mode is called Reverse Horn Current (RHC). The composition of the neutrino beam for both these modes at the NOvA near detector, shown as a rate of charged current (CC) events, is presented in Fig.~\ref{fig:NOvABeamComponents}, displaying the very high purity $\nu_\mu$ component in the FHC beam, and the high purity $\overline{\nu}_\mu$ component in the RHC mode \cite{NuMI.pdf}.

\begin{figure}[!htb]  
  \centering
  \includegraphics*[width=.495\textwidth]{Plots/NOvAExperiment/NuMIBeamComponentsCCEvtsFHC.pdf}
  \noindent\centering
  \includegraphics*[width=.495\textwidth]{Plots/NOvAExperiment/NuMIBeamComponentsCCEvtsRHC.pdf}
  \caption[NuMI neutrino beam components in the NOvA near detector]{The charged current event rates for different neutrino flavours, as measured at the NOvA near detector in the FHC regime shown on the left, or the RHC regime on the right. The contribution of neutrino flavours to the event rates is also displayed, showing the high purity of the neutrino beam for NOvA. Figure from internal NOvA repository.}
 \label{fig:NOvABeamComponents}
\end{figure}

The resulting neutrino beam energy distribution is peaked at $\sim\unit[7]{GeV}$ with a wide energy band. However, thanks to the kinematics of the dominant pion decay, by placing NOvA detector $\unit[14.6]{mrad}$ ($\approx\unit[0.8]{\degree}$) off the main NuMI beam axis, we achieve a narrow band neutrino flux peaked at $\unit[1.8]{GeV}$ \cite{NOvAResults2021.pdf,NOvATechreport.pdf}, as can be seen in Fig.~\ref{fig:NOvAOffAxis}. Using an off-axis neutrino flux increases the neutrino beam around $\unit[2]{GeV}$ about 5-fold compared to the on-axis flux and narrow-band peak enhances background rejection for the $\nu_e$ appearance analysis \cite{NOvATechreport.pdf}.

%Protons originate as $\textsc{H}^-$ ions, accelerated by the Linac to $\unit[400]{MeV}$, converted to protons and further accelerated to $\unit[8]{GeV}$ in the Booster, to be passed to the Main Injector which finally accelerates them to $\unit[120]{GeV}$ . Protons are then extracted, bent down to point towards the MINOS/NOvA Far Detector, and transported to the NuMI target \cite{NuMI.pdf}. The current beam power is $\sim\unit[700]{kW}$ with a plan \cite{PIP2.pdf} of reaching more than $\unit[1]{MW}$ beam power in the future upgrades.
%The NuMI target is a graphite fin, $\unit[7.4]{mm}$ wide, $\unit[63]{mm}$ tall and $\approx\unit[120]{cm}$ long (along the beam direction)\footnote{Previous target proportion were $\unit[6.4]{mm}$ W, $\unit[15]{mm}$ H and $\unit[95.38]{cm}$ L used in low energy design (see lower) \cite{NuMI.pdf}.} \cite{LEOFluxPredictionAtNuMI.pdf}. Protons interact in the target producing hadrons, predominantly pions and kaons \cite{NuMI.pdf}.

%The off-axis location means that both NOvA detectors are sited $14.6\,\si{\milli\radian}$ off the NuMI beam axis, in contrast to the MINOS Far Detector. This is because at around $14\,\si{\milli\radian}$, the energy of the neutrino does not have a strong dependence on the energy of the parent pion (fig. \ref{angleoff}), and also at this angle, the medium energy beam produces a narrow energy beam with approximately five times more neutrinos at $2\,\si{\giga\electronvolt}$ (fig. \ref{off-axis}), which is well-matched to the oscillation maximum expected to be at $1.6\,\si{\giga\electronvolt}$, thus maximizing the experiment’s neutrino oscillation sensitivity. In addition to the increased flux, the narrowness of the off-axis spectra enhances background rejection.\cite{techreport}

\begin{figure}[!htb]  
  \centering
  \includegraphics*[width=.48\textwidth]{Plots/NOvAExperiment/PionOffAxis.pdf}
  \noindent\centering
  \includegraphics*[width=.51\textwidth]{Plots/NOvAExperiment/OffAxisFluxPionEmbedded.pdf}
  \caption[The NOvA off-axis beam concept]{(Left) Dependence of the neutrino energy on the parent pion's energy and (right) neutrino energy distribution for an on-axis beam and three different off-axis beam designs. The case for NOvA is shown here in red and results in a narrow neutrino energy distribution around $\unit[2]{GeV}$, with limited dependence on the parent pion's energy. Figure from \cite{NOvATechreport.pdf}}
 \label{fig:NOvAOffAxis}
\end{figure}

\section{The NOvA Detectors}\label{sec:NOvADetectors}

The two main NOvA detectors are the Near Detector (ND), located in Fermilab $\sim\unit[1]{km}$ from the NuMI target and $\sim\unit[100]{m}$ under ground, and the Far Detector (FD), located $\sim\unit[810]{km}$ from Fermilab at Ash River in north Minnesota, partially underground with a rock overburden \cite{NOvATechreport.pdf}. NOvA also operated a detector prototype called Near Detector on the Surface (NDOS) used for early research and development of detector components and analysis \cite{NOvAStatusAndOutlook.pdf}. Additionally, NOvA operated a Test Beam (TB) detector, described in detail in Sec.~\ref{sec:TBDetector}. The scale of ND and FD is shown in Fig.~\ref{fig:NOvADetectors}.

%The FD has an approximately 130kHz of cosmics

\begin{figure}[ht]
\centering
%is pdf-a
\includegraphics[width=1\textwidth]{Plots/NOvAExperiment/NOvADetectors.png}
\caption[NOvA detectors]{Schematic description of scale and composition of the NOvA detectors. The inset shows a photo of the orthogonal planes made out of PVC cells. An example of a far detector cell containing liquid scintillator and a loop of wavelength sifting fibre attached to an avalanche photodiode is shown on the right \cite{NeutrinoDetectorsForOscExp.pdf}.}
\label{fig:NOvADetectors}
\end{figure}

All NOvA detectors are highly segmented, highly active, functionally identical tracking calorimeters made up of polyvinyl chloride (PVC) cells filled with liquid scintillator. Each cell is a long rectangular cuboid with depth of $\unit[5.9]{cm}$ and width of $\unit[3.8]{cm}$ (with some variations), with cell length extending to the full width/height of each detector, which is $\sim\unit[4.1]{m}$ for the ND and $\sim\unit[15.6]{m}$ for the FD \cite{NOvATechreport.pdf}. An example of a FD cell is shown on the right of Fig.~\ref{fig:NOvADetectors}.

Cells are connected side-by-side into a 16 cell-wide extrusions with $\unit[3.3]{mm}$-wide walls between cells and $\unit[4.9]{mm}$-wide walls on the outsides of the extrusions. The first and last cell of each extrusion are $\sim\unit[3]{mm}$ narrower than the rest of the cells. Two extrusions are connected side-by-side to form a 32 cell-wide module, with each module having a separate readout (see Sec.~\ref{sec:DAQ}). In the FD, 12 modules are connected side-by-side to form one plane of the detector. In the ND only 3 modules make up a plane. Planes are positioned one after another, alternating between vertical and horizontal orientation, and grouped into diblocks, each containing 64 planes. The FD contains 14 diblocks, totalling 896 planes, whereas the ND contains 3 diblocks totalling 192 planes. However, the ND also consists of a Muon Catcher region, positioned right after the active region, consisting of 22 planes of the normal NOvA detector design, 2 modules high and 3 modules wide, sandwiched with 10 steel plates to help range out muons mainly from the $\nu_\mu$ charged current interactions~\cite{NOvAStatusAndOutlook.pdf,NOvATechreport.pdf}. 

Each cell is filled with a liquid scintillator consisting of mineral oil with $4.1\%$ pseudocumene as the scintillant \cite{NOvAScintillators.pdf}. Each cell contains a single wavelength shifting fibre with double the length of the cell, looping at one end and connecting to the readout at the other. As light travels through the fibre, it is attenuated by about a fraction of ten for the FD cells \todo{Figure out what is the correct statement here}. The PVC walls of the detector cells are loaded with highly reflective titanium dioxide, with light typically bouncing off the PVC walls about 8 times before being captured by the fibre \cite{NOvATechreport.pdf}. 

The final dimensions of the FD are $\unit[15.6]{m}\times\unit[15.6]{m}\times\unit[60]{m}$ with a total mass of $\unit[14]{kT}$ and for the ND the dimensions are $\unit[3.8]{m}\times\unit[3.8]{m}\times\unit[12.8]{m}$ with a mass of about $\unit[0.3]{kT}$ \cite{NOvAHalfTimeOverview2022.pdf}. The active volume, consisting only of the liquid scintillator without the PVC structure, makes up about $70\%$ of the total detector volume \cite{NOvATechreport.pdf}.

The NOvA detectors are specifically designed for electromagnetic shower identification, with a radiation length of $\unit[38]{cm}$, which amounts to $\sim 7$ planes for particles travelling perpendicular to the detector planes \cite{NOvAStatusAndOutlook.pdf}. This is particularly useful to distinguish electrons and $\pi^0$s.

\todo{Talk here or in the next section about minimum electron energy to be recorded by NOvA detector and electronics. Maybe in all sections including reconstruction to tie them together}
%The MIP energy loss for electrons (similarly to muons) can be found with a similar method as used in the AbsCal\_technote\_1stAna in TestBeam (page 2)

\section{Readout and Data Acquisition}\label{sec:DAQ}

The signal from the wavelength shifting (WLS) fibres is read out by an Avalanche Photodiode (APD), converting the scintillation light into electrical signal, with a high quantum efficiency of $\sim 85\%$ and a gain of $100$ \cite{NOvATechreport.pdf}. An example APD is shown in Fig.~\ref{fig:NOvAAPD}. Both ends of each fibre are connected to one of the 32 pixels on the APD, with each APD reading out signal from one module. To maximise the signal to noise ratio, the APDs are cooled to $\unit[-15]{\degree C}$ by a thermoelectric cooler, with heat carried away by a water cooling system.

The combination of the APD quantum efficiency and the light yield, determined by the PVC reflectivity and scintillator and WLS fibre response, result in a signal requirement of at least 20 photoelectrons in response to minimum ionizing radiation at the far end of the FD cell.

%TDR-12.2.1:"NOvA readout electronics requires, at minimum, a 20 photoelectron signal in response to minimum ionizing radiation at the far-end of a 15.5 m NOvA cell as discussed in Chapter 6. The signal strength is due to the APD quantum efficiency and the light yield in response to ionizing radiation. The light yield, in turn, is due to a combination of the PVC reflectivity, the scintillator and wavelength-shifting fiber responses."

%For some reason the TDR I have downloaded doesn't have the full chapter 14. Full TDR can be found in docdb:2678 chapter by chapter.

\begin{figure}[!htb]  
  \centering
  \includegraphics*[width=.495\textwidth]{Plots/NOvAExperiment/NOvAAPDMountedWithLabels.jpg}
  \noindent\centering
  \includegraphics*[width=.495\textwidth]{Plots/NOvAExperiment/NOvAAPDBottomWithLabels.jpg}
  \caption[NOvA Avalanche Photo Diods]{The modules with Avalanche Photo Diodes for NOvA mounted on top of the detector on the left picture, and shown from the bottom on the right. The individual components of the module are described. The left picture shows a disconnected ribbon cable and ground cable, which are normally connected to the front end board.}
 \label{fig:NOvAAPD}
\end{figure}

Each APD is connected to a single Front End Board (FEB), shown in Fig.~\ref{fig:NOvAFEB}. The FEB amplifies and integrates the APD signal, determines its amplitude and arrival time, before passing it to the data acquisition system (DAQ). On the FEB the APD signal is first passed to a custom NOvA Application-Specific Integrated Circuit (ASIC), which is design to maximize the detector sensitivity to small signals. ASICs amplify, shape and combine the signal, before sending it to an Analog-to-Digital Converter (ADC). The combined noise from the APD and the amplifier is equivalent to about 4 photoelectrons, which, compared to an average photoelectron yield from the far end of the FD cell of 30, results in a good signal and noise separation \cite{NOvATechreport.pdf}. The digitized data from an ADC is sent to a Field Programmable Gate Array (FPGA), which extracts the time and amplitude of the ADC signals, while subtracting noise based on a settable threshold. The FPGAs employ multiple correlated sampling methods to reduce noise and improve time resolution of the signal \cite{NOvADAQ.pdf}.

\todo{Find out what is the pedestal/threshold that's being subtracted}

\begin{figure}[!htb]  
  \centering
  \includegraphics*[width=\textwidth]{Plots/NOvAExperiment/NOvAFEBWithLabels.jpg}
  \caption[NOvA Front End Board]{An example of a NOvA Front End Board with individual components labelled.}
 \label{fig:NOvAFEB}
\end{figure}

%TDR:Major components are the carrier board connector location at the left, which brings the APD signals to the NOvA ASIC, which performs integration, shaping, and multiplexing. The chip immediately to the right is the ADC to digitize the signals, and FPGA for control, signal processing, and communication. Data from the ADC is sent to an FPGA where multiple correlated sampling is used to remove low frequency noise. This type of Digital Signal Processing (DSP) also reduces the noise level and increases the time resolution.

All of the NOvA front end electronics (APDs and FEBs) are operated in a continuous readout mode, without requiring any external triggers \cite{NOvATechreport.pdf}. Due to higher detector activity during beam spills, the ND FEBs work at a higher frequency of $\unit[8]{MHz}$, whereas the FD FEBs suffice with $\unit[2]{MHz}$ sampling frequency \cite{NOvADAQ.pdf}.

Data from up to 64 FEBs are concentrated in a Data Concentration Module (DCM), which concatenates and packages the data into $\unit[5]{ms}$ time slices, before sending it to the buffer nodes. DCMs are also connected to the timing system and pass a single unified timing measurement to the FEBs to maintain synchronization across the detector~\cite{NOvADAQ.pdf}.
%The timing information is calibration off-line to account for differences between fibre and cable distances and to achieve a superb timing resolution \cite{NinerThesis} - I don't think I need to talk about the timing calibration here tbh. The NOvA calibration process technically also involves \textbf{timing calibration}, which corrects for the time differences of the signal to be processed \cite{NinerThesis}.

The buffer nodes cache the data for at least 20 seconds while receiving information from the trigger system. Each trigger uses a time window based either on the time of the NuMI beam spill, on a periodic interval for readout of comic events for detector calibration and monitoring, or on a time of activity-based data-driven trigger \cite{NOvADAQ.pdf}. Data that fall within any of the trigger windows are sent to a data logger system, where they are merged to form events, before being written to files for offline processing, or sent to an online monitoring system.

%Data for calibration and non-beam physics channels is collected and a variety of data-driven triggers using real-time reconstruction algorithms run on data stored in an online buffer farm at each detector [26].  Beam events are collected in both detectors using a 550 μs time window centered on the 10 μs beam spill window, triggered by a signal derived from the accelerator controls system. The wide time window of data collection compared to the beam spill is used to sample cosmic-ray activity and noise under the precise detector conditions that apply to the beam data. [NOvAHalfTimeOverview2022.pdf]

%Maybe I should write here what is the event rate?

%Maybe write about what information do we have about each event at this stage: timing, peak ADC, cell, plane, run, subrun. Grouped into triggered windows?
% "the time, location, and pulseheight of those signals are recorded as a hit" [NOvAHalfTimeOverview2022.pdf]

\note{Maybe talk about data quality as well? Probably should since I want to talk about good runs and bad channels in the TB calib chapter}

\section{Simulation}\label{sec:NOvASimulation}

\note{Should I divide the simulation into the individual stages? I had it like that originally, but for example the simulation of cosmics is only very short and not sure what section would I put it under}

To extract neutrino oscillation parameters, or to test a hypothesis, NOvA uses a series of simulations to make predictions according to various physical models \cite{NOvASimulationOld-Fluka.pdf}.
%To simulation the expected signal from the NOvA detectors,
%To get a prediction of any possible signal events or their background in the NOvA detectors,
% we use a series of simulations, tuned or corrected by both internal and external measurements to better match the current state of the art knowledge \cite{NOvASimulationOld-Fluka.pdf}.
The simulation chain can be divided into four parts: simulation of the neutrino beam, simulation of neutrino interactions within the NOvA detectors, simulation of cosmic particles interacting in the NOvA detector and simulation of the detector response.

To simulate the neutrino beam, NOvA uses the \texttt{GEANT4} v9.2.p03~\cite{GEANT4.pdf} based Monte Carlo (MC) simulation with a detailed model of the NuMI beamline \cite{ZPavlovicThesisG4NuMI_2008.pdf}, as it was described in Sec.~\ref{sec:NOvASimulation}. The simulation starts with MI protons interacting within the long carbon target and producing hadrons, mainly $\pi, K$ and $p$, followed by transport and possible further interaction of these hadrons within the focusing system, until finally ending with hadron decays producing the neutrino beam.

To account for the imprecise theoretical models used in GEANT4, we use the Package to Predict the Flux (PPFX) to incorporate external measurements of yields and cross sections of hadron production inside the target and other NuMI materials into the prediction~\cite{NuMIFlux.pdf}. The current version of PPFX is limited by the results available during its creation and only corrects the most frequent interactions while assigning large systematics uncertainties to the rest (see Sec.~\ref{sec:NOvASystematics}). For the most common $\pi$ production, PPFX uses the NA49 measurements \cite{NA49:Inclusive_production_of_charged_pions.pdf} of $\unit[158]{GeV/c}$ protons interacting on a thin (few percent of interaction length) carbon target, with a few data point from Barton et al~\cite{BartonHadProd1983.pdf} to expand the kinematic coverage. These then have to be scaled to the $\unit[20-120]{GeV/c}$ incident proton energies seen in NOvA using the FLUKA \cite{FLUKA_01,FLUKA_02} MC generator. For the $K$ production from $p+C$ interaction, important for higher neutrino energies and electron neutrinos, PPFX uses the NA49 $K$ data \cite{NA49DataKaons.pdf} together with the NA49 $\pi$ data \cite{NA49:Inclusive_production_of_charged_pions.pdf} multiplied by the $K/\pi$ ratios of yields on thin carbon target from the MIPP experiment \cite{pionToKaonIn_pC.pdf}. Lastly, for the nucleon production, PPFX uses the NA49 data on quasi elastic interactions \cite{NA49pc-proton2013.pdf}. All the other interactions inside NuMI, such as interaction in non-carbon targets, or interactions with hadrons other than protons, are either extrapolated from the previously mentioned measurements, or are not corrected for and a significant systematic uncertainty is assigned to them \cite{NuMIFlux.pdf}.

There's two new experiments that measured the production and interaction of hadrons on various targets and incident energies, specifically designed to improve the prediction of neutrino beams. I worked on implementing data from the NA61 experiment on hadron production from $p+C$ interaction on a thin carbon target at $\unit[31]{GeV/c}$~\cite{2015_hadron_prod_pC_2009data.pdf}, motivated by possible reduction in the $K$ production systematic uncertainty. This work is still ongoing and will be implemented into PPFX and NOvA together with the rest of the NA61 measurement. The most impactful ones will be the measurement of hadron production from $p+C$ interaction on a thin carbon target at $\unit[120]{GeV/c}$ \cite{NA61_hadprodFrompC_120GeV_2023.pdf} (no energy scaling required), measurements of $p+C$ and $p+Be$ at different incident energies \cite{2019_NA61_ProdAndInelXSec_protonOnDiffTargets60And120GeV._results.pdf}, $\pi+C$ and $\pi+Be$ measurements at $\unit[60]{GeV/c}$~\cite{2019_had_prod_at_Pi_on_C_and_Be.pdf}, resonance production measurements from $\unit[120]{GeV/C}$ $p+C$ \cite{NA61_ResonanceProdFrompC_120GeV_2023.pdf}, and probably the most impactful one, the yet unpublished measurement of hadron production yield on a NOvA-era NuMI replica target at $\unit[120]{GeV/c}$ \cite{ThickTargetLimit.pdf}. NA61 also measured the hadron production yield for the T2K experiment's replica target \cite{2019_hadron_yields_T2K_replica.pdf}, which significantly reduced the neutrino flux systematic uncertainty for the T2K measurements \cite{ThickTargetLimit.pdf}. The second experiment is EMPHATIC~\cite{EMPHATICProposal2019.pdf}, which is currently analysing their data on a broad range of hadron production measurements, mainly the secondary and tertiary interactions of various projectiles with a wide range of incident energies and thin target materials, complementary to the NA61 measurements.

%From NOvAResults2021.pdf: The neutrino flux delivered to the detectors is calculated using GEANT4-based simulations of particle production and transport through the beamline components [NuMI.pdf,GEANT4.pdf] reweighted to incorporate external measurements using the package to predict the flux (PPFX) [NuMIFlux.pdf,30–48].

%From NOvAResultsCombinedNuAnu2019.pdf: The flux of neutrinos delivered to the detectors is calculated using a simulation of the production and transport of particles through the beamline components [22,25] reweighted to incorporate external measurements [26–45].

%From NOvAHalfTimeOverview2022.pdf: Neutrino interactions in NOvA are simulated with a chain of software packages. The neutrino flux is modeled using G4NuMI, a geant based description of the NuMI beam line [Z. Pavlovic thesis]. The raw flux from G4NuMI is then modified using the PPFX package to better match the products of the interactions in the extended target to the world’s hadron production data [NuMIFlux.pdf].

%%%%%%%%%%%%%%%%%%%%%%%%%%%%%%%%%%%%%%%%%%%%%%%%%%%%%%%%%%%%%%%%%%%%%%%%%%%%%%%
%%%% Master's thesis on flux simulation
\iffalse
There are often multiple interactions within the target and in the materials downstream of it and since the hadron production process is governed by non-perturbative QCD and occurs in the nucleus, highly accurate theoretical predictions are not possible \cite{NuMIFlux.pdf,LEOFluxPredictionAtNuMI.pdf}. NOvA therefore tunes and corrects possible mismodeling of the model using external data in a package developed for MINERvA experiment called Package to Predict the Flux (PPFX) \cite{LEOFluxPredictionAtNuMI.pdf}.

%...Those models are not necessarily accurate but can be tuned or benchmarked by comparing their predictions to measurements of hadron production. Recent measurements of pion production on a thick (two interaction length) carbon target have been released by MIPP [4], and measurements of pion production on a thin (few per cent interaction length) carbon target are available from NA49 [5]. In addition, there are several other hadron production measurements on various materials, using both proton and pion beams, that can be used to constrain a neutrino beamline simulation.[NuMIFlux.pdf]

%Roughly 85% of the interactions that produce particles that lead to muon neutrinos passing through MINERvA are from protons interacting on carbon. Other relevant materials are aluminum (horns), iron (decay pipe walls), helium (decay pipe gas), and air (target hall). Interactions of π ± , K ± and n created in the initial proton interaction, or subsequent interactions, are subdominant but non-negligible. When protons collide with carbon, the interactions can produce pions, kaons, neutrons, strange baryons, and lower energy protons. These particles, if they do not decay first, can interact either in the target or in other downstream material to create tertiary particles that can also decay into neutrinos. \cite{NuMIFlux.pdf}

PPFX is used to correct each interaction of neutrino's ancestry chain by weighting it with a factor computed from external experimental measurements of yields or invariant differential cross-sections \cite{LEOFluxPredictionAtNuMI.pdf}

The kinematic values of the initial particles (like the initial $\unit[120]{GeV}$ proton interacting on carbon in NuMI) are not always the same between the measured interaction and the required values. To solve this we use the \textit{Feynman-x} ($x_{F}$) scaling variable. Feynman speculated \cite{feynman1969.pdf} that expressing the cross-sections of inclusive high energy hadronic collisions in terms of $x_{F}$ would make the cross-section scaling energy independent \cite{LEOFluxPredictionAtNuMI.pdf}.
%$c_i$ is the \textit{central value} of the weight

There are two main experiments whose results are used in the PPFX. NA49 \cite{NA49:Inclusive_production_of_charged_pions.pdf}, which used $\unit[158]{GeV}$ protons interacting on carbon thin target, and MIPP \cite{pionToKaonIn_pC.pdf} which used protons from the Main Injector and both thin carbon target and the low energy NuMI target (thick target) \cite{PPFXTechnote2017.pdf}. Energy scaling of the external data to calculate the PPFX weight is performed by FLUKA\cite{NuMIFlux.pdf}

%%% Hadron production datasets:
%There are two major datasets available to constrain the process where protons interact on carbon and produce charged pions. One measurement, from NA49 [5], uses a thin target with an incident proton momentum of 158 GeV/c. The other measurement, from MIPP [4], uses an actual NuMI LE target and 120 GeV/c protons. These two datasets will be used to make separate “thin target” and “thick target” flux predictions by weighting each interaction leading to a neutrino going through MINERvA. We also use additional datasets to constrain kaon and nucleon production, and the absorption of particles in beamline materials. Where multiple interactions are constrained with data, the overall weight applied to the neutrino event is simply the product of the weights for each interaction.\cite{NuMIFlux.pdf}

For kaons with $x_{F}<0.2$ PPFX uses weights based on NA49 measurements\cite{NA49DataKaons.pdf} and for kaons with $0.2<x_{F}<0.5$ PPFX uses the $K/\pi$ yield ratio from the MIPP thin target measurements\cite{pionToKaonIn_pC.pdf} multiplied by NA49 thin target yields.
\fi
%%% End of master's on flux simulation
%%%%%%%%%%%%%%%%%%%%%%%%%%%%%%%%%%%%%%%%%%%%%%%%%%%%%%%%%%%%%%%%%%%%%%%%%%%%%%%

%Good description of PPFX, beam transport and the principal components is in the NOvA-T2K technote for Flux (doc-db:54582) https://nova-docdb.fnal.gov/cgi-bin/sso/ShowDocument?docid=54582

\note{The description of neutrino interactions, including QE/Res/DIS scattering and nuclear effects will probably be in the theory chapter. If not I'll add it here.}
\note{Might have to describe some of these interaction models a bit more if any of the cross section uncertainty for the magnetic moment analysis turns up to be significant}

The output of the neutrino beam simulation is passed to the simulation of neutrino interactions inside the detectors, which is done with the GENIE v3.0.6~\cite{GENIE.pdf} neutrino MC generator. GENIE allows users to choose the particular models for different types of neutrino interactions and particle propagation within the nucleus, as well as possible tunes to external measurements. The four main interaction modes in GENIE are the quasi elastic (QE) CC scattering, the resonant baryon production (Res), the deep-inelastic scattering (DIS), and the coherent pion production (COH$\pi$). Special case of CC interaction with two nucleons producing two holes via meson exchange currents (MEC) is also considered. Particles created in these processes are then propagated inside the nucleus according to the final state interaction model (FSI). All of these are set by the Comprehensive Model Configurations (CMCs) and NOvA currently uses the \texttt{N1810j0000} CMC. Additionally, NOvA adds a tune to NOvA $\nu_\mu$CC data for the CCMEC interactions and a set of external $\pi$ interaction measurements to constrain the FSI model. Table~\ref{tab:NuIntSimulationModels} shows the list of models and tunes for different interaction modes in NOvA \cite{NOvAResults2021.pdf}.

\begin{table}[!ht]
\centering
%\def\arraystretch{1.4}
\caption{Models and tunes used in the NOvA simulation of neutrino interactions.}
\begin{tabular}{|l|l|l|}
\hline
Interaction & Model                  & Tune\\\hline
CCQE & Val\`{e}ncia \cite{ValenciaModel_NOvACCQE_2004.pdf} & External $\nu-\textsf{D}$ data \cite{NuDeuteriumScattering_NOvACCQETune_2016.pdf}\\
CCMEC       & Val\`{e}ncia \cite{ValenciaModel_NOvACCQEMEC_2011.pdf,ValenciaModel_NOvAMEC_2013.pdf} & NOvA $\nu_\mu$CC data\\
Res. \& COH$\pi$ & Berger-Sehgal \cite{BergerSehgal_ResonancePionProd_2007.pdf,BergerSehgalModel_CohPionProd_2009.pdf}          & External $\nu-A$ data\\
DIS         & Bodek-Yang \cite{BodekYangModel_NOvADIS_2003.pdf,HadronizationModelForNuDIS_NOvADIS_1988.pdf}            & External $\nu-A$ data\\
FSI         & Semi-classical cascade \cite{FSIModel_hNSemiClassicalCascade_1988.pdf} & External $\pi-^{12}\textsf{C}$ data\\\hline
\end{tabular}
\label{tab:NuIntSimulationModels}
\end{table}

%NOvANuMuCCPi0XSecMeasurement2023.pdf: Neutrino interactions are simulated with GENIE [17] v2.10.2. The GENIE simulation generates interactions via its four default production processes: quasielastic scattering, resonant baryon production, deep-inelastic scattering, and coherent pion production. Particles created via these primary processes are subsequently propagated though the nuclear medium using GENIE’s hA effective cascade FSI model [18,19].

% Describe that GENIE is highly costumizable and you can set up any generators you like. Ideally describe what parts are purely theoretical and what parts are tuned to external data. Also mention that NOvA is doing an internal tune to some of the parameters.

%From NOvAResults2021.pdf: Neutrino interactions are simulated using a custom model configuration of GENIE 3.0.6 [49,50] tuned to external and NOvA ND data.
%In this configuration, charged-current (CC) quasielastic (QE) scattering is simulated using the model of Nieves et al. [53], which includes the effects of long-range nucleon correlations calculated according to the random phase approximation (RPA) [53–55]. The CCQE axial vector form factor is a z-expansion parametrization tuned to neutrino-deuterium scattering data [56].
%CC interactions with two nucleons producing two holes (2p2h) are given by the IFIC València model [57,58]. The initial nuclear state is represented by a local Fermi gas in both the QE and 2p2h models, and by a global relativistic Fermi gas for all other processes.
%Baryon resonance (RES) and coherent pion production are simulated using the Berger-Sehgal models with final-state mass effects taken into account [59,60].
%Deep inelastic scattering (DIS) and nonresonant background below the DIS region are described using the Bodek-Yang model [61] with hadronization simulated by a data-driven parameterization [62] coupled to PYTHIA [63].
%Bare nucleon cross sections for RES, DIS, and nonresonant background processes are tuned by GENIE to neutrino scattering data.
%Final-state interactions (FSI) are simulated by the GENIE hN semi-classical intranuclear cascade model in which pion interaction probabilities are assigned according to Oset et al. [64] and pion-nucleon scattering data.

%The 2p2h and FSI models in this GENIE configuration are adjusted to produce a NOvA-specific neutrino interaction model tune. The 2p2h model is fit to $\nu_\mu$CC inclusive scattering data from the NOvA ND. Inspired by Gran et al. [65], this 2p2h tune enhances the base model as a function of energy and momentum transfer to the nucleus and is applied to all CC 2p2h interactions for both the neutrino and antineutrino beams. 
%The parameters governing $\pi^\pm$ and $\pi^0$ FSI are adjusted to obtain agreement with $\pi^+$ on $^{12}\textsf{C}$ scattering data [66–72].

%From NOvAResultsCombinedNuAnu2019.pdf: Neutrino interactions in the detector are simulated using GENIE [46] tuned to improve agreement with external measurements and ND data, reducing uncertainties in the extrapolation of measurements in the ND to the FD. As in Ref. [21], we set MA in the quasielastic dipole form factor to 1.04 GeV/c2 [47] and use corrections to the charged-current (CC) quasielastic cross section derived from the random phase approximation [48,49]. In this analysis, we also apply this effect to baryon resonances as a placeholder for the unknown nuclear effect that suppresses rates at a low four-momentum transfer in our and other measurements [50–53]. Additionally, we increase the rate of deep-inelastic scattering with hadronic mass W > 1.7 GeV/c2 by 10\% to match our observed counts of short track-length $\nu_\mu$CC events. We model multinucleon ejection interactions following Ref. [54] and adjust the rates in bins of energy transfer, $q_0$, and three-momentum transfer, $\left|\overrightarrow{q}\right|$, for $\nu_\mu$ and $\overline{\nu}_\mu$ separately to maximize agreement in the ND. The calculation of the $\nu_e$ and $\overline{\nu}_e$ rates uses these same models.

%From NOvAHalfTimeOverview2022.pdf: Neutrino interactions and final state interactions are modeled using the GENIE neutrino interaction generator [31]

%From master thesis: From there GENIE event generator \cite{GENIE.pdf} simulates neutrino interactions in the detector \cite{2019NOvAFHCRHCResults.pdf} and another GEANT4 simulates the detector response \cite{NOvASimulationOld-Fluka.pdf}. NOvA also tunes the cross-section model of the GENIE simulation to the ND data to reduce uncertainties in the extrapolation of measurements on the ND to the FD \cite{2019NOvAFHCRHCResults.pdf}.

Since the FD is on the surface we also need to include a simulation of cosmic rays generated with the CRY \cite{CRY} MC generator. The simulated cosmic muons are also used to calibrate NOvA detectors \cite{NuMIFlux.pdf}. 

Particles that are created from neutrino interactions and cosmic rays are propagated through the NOvA detectors using an updated version of \texttt{GEANT4} v10.4.p02~\cite{GEANT4.pdf}. The output of this simulation is the energy deposited in the scintillator, which is then passed to a custom NOvA simulation software \cite{NuMIFlux.pdf}. The scintillation light generated by the deposited energy is parametrized using the Birks-Chou model \cite{BirksChouParametrization_1952.pdf}, which corrects for recombination in organic scintillators at high deposited energies. The normalization factors for the produced scintillation light (the light yield), as well as for the Cherenkov light, which can affect the light readout, are tuned to NOvA cosmic data~\cite{NOvANumuCCXSexMeasurement2023.pdf}. The light collection by the WLS fibres and its transport to the APDs, as well as the APD response use a parametrized simulation, which makes use of the fact that all the NOvA cells and their readout are generally the same across the detectors \cite{NuMIFlux.pdf}. The simulation of the readout electronics is done by another custom NOvA parametrized model, which mainly account for a random electronics noise, with output in the same format as raw data.

%NuMIFlux.pdf: While GEANT4 is capable of simulating optical photon processes, generating scintillation light and propagating it through the cell, up the fiber, and to the APD is very time consuming. Instead, we observe that the NOvA detectors are composed of many identical readout cells as shown in Fig. 6, so if we can generate templates to parameterize photon transport once, we can use them everywhere. The processes we must be able to parameterize are: the collection of scintillation photons by the fiber, the transport of light up the fiber, and the response of the APD to the captured light.

Due to the high neutrino rate in the ND, there are neutrinos interacting in the surrounding rock creating particles that make it to the detector and act as background. To simulate these rock events we use the same simulation as for neutrino interactions inside the detector. However, since only a few particles make it into the detector, it would be very time consuming to run this simulation for every neutrino. Therefore, we create a separate simulation that includes the surrounding rock and then overlay the results into the normal NOvA simulation chain, which doesn't include the rock, so that the rate matches the NuMI neutrino rate \cite{NuMIFlux.pdf}.

%%% Rock simulation and overlay
%NuMIFlux.pdf: Due to the high beam intensity at the near detector, many neutrinos interact in the rock in front of the detector. Simulating these interactions requires allowing GEANT4 to propagate muons through a very large rock volume which is a slow process, and only a few of these muons will make it into our detector. To correctly account for this, we simulate many neutrino interactions with the mother volume including a large rock volume in front of the detector, and only keep those that leave energy in the detector. During normal simulation, with the mother volume only including the detector and the immediate detector hall, we overlay these rock ’singles’ at a rate determined during the generation of flux files after the GEANT4 stage.



%%%% End of master's thesis for Sim
%%%%%%%%%%%%%%%%%%%%%%%%%%%%%%%%%%%%%%%%%%%%%%%%%%%%%%%%%%%%%%%%%%%%%%%%%%%%%%%

\section{Data Processing and Event Reconstruction}
Both data and simulation events for all NOvA detectors are passed through the same event reconstruction and particle identification algorithms. The reconstruction was specifically developed with the $\nu_e$ appearance search in mind, focusing on identifying the $\nu_e$CC signal against the $\nu_\mu$CC and neutral current (NC) backgrounds. Each NOvA detector has to deal with a different challenges, with multiple neutrinos interacting in the ND during one beam spill, and a large cosmic background in the FD \cite{NOvAReco.pdf}.

The most common topologies for particles interacting in NOvA detectors are shown on Fig.~\ref{fig:NOvAEventTopologies}. Muons are easily identifiable as a single long track which decays into an electron (or positron) if it stops inside the detector. Both electrons and $\pi^0$'s produce electromagnetic showers, but thanks to the low-Z composition and high granularity of the detector, there is a gap between the interaction vertex and the electromagnetic shower.

\begin{figure}[ht]
\centering
%is pdf-a
\includegraphics[width=1\textwidth]{/home/robert/Documents/School/PhD/Thesis/Plots/NOvAExperiment/NOvAEventTopology.pdf}
\caption[NOvA detectors event topologies]{Different event topologies as seen in the NOvA detectors with corresponding Feynman diagrams \cite{NOvAReco.pdf}. Each event is a simulated $\unit[2.15]{GeV}$ neutrino interacting in a NOvA detector producing a $\unit[0.78]{GeV}$ proton and a second $\unit[1.86]{GeV}$ particle depending on the interactions type. The figure show one view and the colouring represents the deposited energy.}
\label{fig:NOvAEventTopologies}
\end{figure}

The readout from each cell from the DAQ (see Sec.~\ref{sec:DAQ}) is called a \textit{channel} and the DAQ output from each channel is called a \textit{raw hit}. DAQ groups hits into $\unit[550]{\mu s}$ windows and passes them to an offline reconstruction chain~\cite{NOvAReco.pdf}. Reconstruction starts by grouping hits into \textit{slices} based on their proximity to other hits in both time and space~\cite{DBSCAN.pdf}.
\note{Maybe include rawhit to cellhit to calhit to recohit. Find out where this is described}

For events that produce hadronic and electromagnetic showers, we first identify lines through major features using a modified Hough transform~\cite{HoughTransform.pdf}. These lines representing momentum directions are then passed to the Elastic Arms algorithm~\cite{ElasticArms.pdf} to identify \textit{vertex} candidates from their intersection points. Hits are then clustered into \textit{prongs}, group of hits with a start point and a direction, using a k-means algorithm called FuzzyK \cite{FuzzyKClustering.pdf,FuzzyKFuzzyness.pdf}. Here "fuzzy" means that each hit can belong to multiple prongs. Prongs are first created separately for each view (also called 2D prongs) and then, if possible, view-matched into 3D prongs (or just prongs)~\cite{NOvAReco.pdf}. Figure~\ref{fig:NOvARecoEVD} shows an example simulated electron shower with the reconstructed vertex (red cross) and prong (red shaded area) grouping all hits that should be a part of the shower together, while removing background hits in grey.

For tracks with do kalman tracks. For cosmic muons we do window cosmictrack. We can also do break point fitter from the vertex and prong information.

\begin{figure}[ht]
\centering
%is pdf-a
\includegraphics[width=1\textwidth]{/home/robert/Documents/School/PhD/Thesis/Plots/NOvAExperiment/ElectronRecoEVD.png}
\caption[NOvA reconstruction of a single electron]{Reconstruction of a simulated single electron event in the NOvA ND. The red cross is the reconstructed vertex, the shaded area shows the cluster of hits into a shower and the dotted red line shows the estimated momentum of that shower. The blue dotted line shows the true momentum of the scattering neutrino and the solid red line the true momentum of the scattered electron. Figure from internal NOvA database.}
\label{fig:NOvARecoEVD}
\end{figure}

NOvANumuCCXSexMeasurement2023.pdf: Hits in an event are then grouped into possible particle trajectories (tracks) via a Kalman filter-based algorithm in both the horizontal and vertical two-dimensional detector views [35]. Three-dimensional tracks are formed by combining tracks from the two views based on their overlap in the longitudinal direction. The track reconstruction algorithm assumes a start point at the most upstream hit, and requires a minimum of 4 hits in each detector view.

PID description. CVN and BDT and others...

%From First NOvA results on FHC+RHC: Cells with activity above threshold (hits) are grouped based on their proximity in space and time to produce candidate neutrino events. Events are assigned a vertex, and clusters are formed from hits likely to be associated with particles produced there [NOvAReco.pdf]. These clusters are categorized as electromagnetic or hadronic in origin using a convolutional neural network (CNN) [Fernanda Psihas thesis]. Hits forming tracks are identified as muons by combining information on the track length, dE=dx, vertex activity, and scattering into a single particle identification (PID) score [Raddatz thesis]. The same reconstruction algorithms are applied to events from data and simulation in both detectors.

%From NOvAHalfTimeOverview2022.pdf: To reconstruct neutrino events, hits are first grouped in time and space into slices. A slice is further analyzed to identify individual, final state particle tracks, called prongs. Flavor classification works on the group of hits in a slice and relies on versions of convolutional neural networks adapted from computer vision applications. Specifically, NOvA uses a convolutional visual network (CVN) trained on hit maps of simulated events to learn topological features that distinguish the flavors of neutrinos interacting in the detector [32]. The energy of charged current neutrino interactions is estimated from the sum of energies of the lepton and the hadronic recoil system. A Kalman-like algorithm is used with a BDT) based on energy loss, multiple scattering, and length parameters of a candidate track to identify muons and reconstruct their energy using track length. Electron showers are identified using a CVN trained to classify individual particle tracks within a neutrino event. The energy of electron neutrinos is estimated from a quadratic function of the calorimetric energies of the electron candidate and the hadronic recoil system derived from the simulation [33, 34].

%%%%%%%%%%%%%%%%%%%%%%%%%%%%%%%%%%%%%%%%%%%%%%%%%%%%%%%%%%%%%%%%%%%%%%%%%%%%%%%
%%% Master's thesis on NOvA Reco:
\iffalse
Since most of the primary goals of NOvA depend on successfully observing $\nu_e$ charged-current interaction, the reconstruction chain was tailored for this task \cite{NOvAReco.pdf}. Different interactions as seen in a NOvA detector are shown in fig.\ref{interactions}. Reconstruction begins by clustering hits into slices by time and space "density". Than a modified Hough transform identifies prominent features, which are used to determine the global 3D vertex for the slice. This vertex is used in the fuzzy k-means algorithm to produce prongs (a collection of cell hits with a start point and a direction) which are fed to a neural network to classify the degree to which the slice was like a $\nu_e$ CC (or other) interaction \cite{NOvAReco.pdf}. NOvA uses a convolutional neural network called Convolution Visual Network (CVN) originally based on the GoogLeNet architecture. CVN identifies neutrino interactions based on their topology and therefore without the need for detailed reconstruction \cite{CVN.pdf}.
\fi
%%%% End of master's thesis for Reco
%%%%%%%%%%%%%%%%%%%%%%%%%%%%%%%%%%%%%%%%%%%%%%%%%%%%%%%%%%%%%%%%%%%%%%%%%%%%%%%

%%%%%%%%%%%%%%%%%%%%%%%%%%%%%%%%%%%%%%%%%%%%%%%%%%%%%%%%%%%%%%%%%%%%%%%%%%%%%%%
%%%%%%%%%                   Detector calibration                     %%%%%%%%%%
%%%%%%%%%%%%%%%%%%%%%%%%%%%%%%%%%%%%%%%%%%%%%%%%%%%%%%%%%%%%%%%%%%%%%%%%%%%%%%%
\section{Detector Calibration}\label{sec:NOvACalibration}

\todo{Reorganize the calibration section, ideally make it much shorter and more concise}

%From NOvAResults2021.pdf: The absolute energy scale for both detectors is calibrated using the minimum ionizing portion of stopping cosmic-ray muon tracks [76]. The calibration procedure is now applied separately to the data in shorter time periods to account for an observed 0.3\% decrease in detected light per year.

%From NOvAHalfTimeOverview2022.pdf: The variations in light output between cells and those due to attenuation along the readout fiber, in both data and simulation, are calibrated using cosmic-ray muons. The overall energy response of the detectors is calibrated using stopping muon tracks along a window from 200 cm to 100 cm before the end of the track. The absolute energy scale is cross-checked and bench marked against simulation using beam-induced protons, muons, and neutral pions at the Near Detector.

%%% General introduction to calibration
The purpose of calibration is to make sure that we get the same amount of energy wherever or whenever it's deposited in whichever of NOvA's detectors and to express this amount of energy in physical units. The NOvA calibration uses cosmic ray muons, which provide a consistent, abundant, and well-understood source of energy deposition.

%. The main aims are to correct for the attenuation of light along fibers, to remove differences in energy deposition within the detector, and to provide an absolute energy scale from collected charge to physical energy units. [TB paper]

%%% Calibration samples, trigger, reconstruction, and selection. Also simulation, fiber brightness and so on...

\subsubsection*{Creating calibration samples}\label{secCreatingCalibrationSamples}

We want to select good quality cosmic ray muons. First, we remove beam related events based on their time stamps relative to the time of the beam spill. Next we apply reconstruction to get the CellHit, slicer, and track information, followed by a track-based selection to remove misreconstructed and poor quality events.

Since energy deposition depends on the path length particle travels through a cell, we only use hits for which we can reliably calculate their path length. We call these hits \textbf{tricell} hits, as we require that all accepted hits are accompanied by a recorded hit in both neighbouring cells of the same plane, as shown in Fig. \ref{figTricellCondition}. In case there is a bad channel in a neighbouring cell, we ignore this channel and look one cell further. We can then calculate the path length simply as the cell width divided by the cosine of the direction angle \cite{NOVA-doc-13579,NOVA-doc-7410}.

\begin{figure}[hbtp]
\centering
\begin{subfigure}[b]{0.49\textwidth}
\centering
\includegraphics[width=0.6\textwidth]{Plots/NOvAExperiment/TricellConditionWithDescription.png}
\caption{}
\end{subfigure}
\begin{subfigure}[b]{0.49\textwidth}
\centering
\includegraphics[width=0.6\textwidth]{Plots/NOvAExperiment/TricellConditionWithBadChannel.png}
\caption{}
\end{subfigure}
\caption[Tricell condition for calibration hits in NOvA]{Illustration of the tricell condition. We only use hits that have two surrounding hits in the same plane to be used in the NOvA calibration. This is to ensure a good quality of the path length reconstruction (d), which is calculated from the known cell height (h) and the reconstructed track angle $\left(\varphi\right)$. In case the hit is next to a bad channel, as shown on the right plot, we ignore this bad channel and require a hit in the next cell over.}
\label{figTricellCondition}
\end{figure}

%The first step in the attenuation calibration is to select the suitable hits from tracks of cosmic ray muons. Because a reliable estimate of pathlength is required, not all hits are suitable for use.  If a cell has each of its neighbors in the same plane hit, then we know, for a Y view cell, that the track entered through the upper wall, and exited through the lower wall. The pathlength then is just the width of the cell divided by the direction cosine. This selection also significantly decreases the chance that the hit in question is a noise hit. Allowance is made for neighboring dead cells, so e.g. “hit, dead, hit, hit” would still lead the 3rd cell to be selected. The second best hit selection, in cases where there are too many dead neighboring cells on each side, is the so-called “z” estimator, where a hit is required at the same cell number in each of the neighboring planes in the same view. The pathlength is then the ratio of cell depth to cz.[docdb:13579 - SA The Attenuation and Threshold Calibration of the NOvA detector, copied from docdb:7410]

%from docdb:7410: A requirement that the track be “throughgoing” (lowest endpoint outside the fiducial volume) was applied, but doesn’t make much difference. I think this selection was broken by the recent changes to StopperSelection anyway. (So it seems that it was required for the relative calibration that the muons are through-going, but I assume this was discarded somewhere down the line)

For the absolute calibration we select muons that stop inside the detector, by identifying muons with a Michel electron at the end of their track \cite{NOVA-doc-13579-FACalorimetricEnergyScale}.

%Stopping muon selection (from docdb:13579 - FA\_Calorimetric\_energy\_scale): There are two avenues for selecting stopping muons; i) selecting tracks whose reconstructed end point is contained within the detector and ii) selecting tracks that have a Michel electron at one end. Michel electrons are useful for both identifying muons and effective tagging of the end point of muon tracks. The stopping selection requires the reconstructed end point of the muon track to be at least 50 cm from the detector edge. The identification of a Michel electron at the end of a muon track has two stages of both temporal and spatial range requirements. Firstly, a candidate Michel electron hit is required to occur between 1 and 30 microseconds after the mean time of the hits on the track. Furthermore, the candidate Michel electron must be within a 30 cm sphere surrounding the reconstructed track end point. The candidate Michel electron hit for a muon track is the hit that produces the largest detector response among the hits that pass the above cuts. Secondly, cell hits surrounding the candidate Michel electron hit are associated with the Michel electron if they occur within a 30 cm sphere surrounding the Michel electron candidate. Furthermore, to be associated with the Michel electron the cell hits must occur between 0.5 microseconds before and 0.5 microseconds after the candidate Michel electron. Michel electrons at the end of muon tracks are reconstructed using the candidate and associated Michel electron hits. The stopping muon selection requires a Michel electron at the end of the muon track.

For each data period or epoch and for each version of the simulation we create two calibration samples that are used as the input for the relative and absolute calibration. The samples are called \cite{NOVA-doc-13579-CalibrationMetaREADFIRST}
\begin{itemize}
\item pclist = \textbf{list} of \textbf{p}re-\textbf{c}alibrated hist; Contains all selected cosmic muon events and is used in the relative calibration;
\item pcliststop = pclist files only containing stopping muons used for the absolute calibration 
\end{itemize}

\subsubsection*{Fibre brightness}

For data, the relative calibration is done for each individual cell in each plane to properly account for any potential variations, repeating the attenuation fit $N_{cell}\times N_{plane}$ times. However, generating enough simulated events turned out to be computationally expensive. Therefore, assuming the simulated detector is approximately uniform plane to plane, for simulation we can "consolidate" the detector planes and only consider variations in the two views. Therefore for simulation we would repeat the fit $N_{cell}\times N_{view}$ times \cite{NOVA-doc-13579-SAAttenuationAndThreshold,NOVA-doc-34909}.

However, there are some variations in the detector response cell by cell that can be caused by different fibre brightnesses, but also by different qualities of the scintillator, air bubbles, APD gains, looped or zipped fibres and potentially others. We want to include these variations in the simulation to better match data. To emulate these differences in the simulation without the need to simulate every cell individually, we divide each detector into 12 brightness bins, as shown in Fig. \ref{figFiberBrightnessBins}. These brightness bins describe the relative differences in the detector response between individual cells \cite{NOVA-doc-34909}. Therefore in the end, for simulation we perform the attenuation fit $N_{cell}\times N_{view}\times N_{BrightnessBin}$ times.

%Do I need to describe here how we do it or is it enough if I just say that we do?
%To divide the detector into brightness bins, we use the results of the attenuation fit and look at the fitted response at cell centre, which should represent the average response in that cell. Then we normalize the 

\begin{figure}[hbtp]
\centering
\begin{subfigure}[b]{0.495\textwidth}
\centering
\includegraphics[width=\textwidth]{Plots/NOvAExperiment/BrightnessIndex.png}
\end{subfigure}
%\hfill
\begin{subfigure}[b]{0.495\textwidth}
\centering
\includegraphics[width=\textwidth]{Plots/NOvAExperiment/BrightnessIndexToValue.pdf}
\end{subfigure}
\caption[Fibre Brightness bins for the NOvA calibration]{The Test Beam detector is (like the standard NOvA detectors) divided into 12 brightness bins (left plot), each representing a relative difference in energy response (right plot) due to different brightnesses of the fibres, scintillators, or readout.}
\label{figFiberBrightnessBins}
\end{figure}

To divide each detector into the 12 brightness bins, we use results from the relative calibration. Specifically we take the result of the attenuation fit (equal to the average response) in the centre of each cell to fill a 2D histogram. Then we normalize this histogram by dividing the response in each $\textsf{Cell}\times\textsf{View}\times\textsf{Plane}$ by the average response in the corresponding $\textsf{Cell}\times\textsf{View}$. All uncalibrated cells get assigned the average response (1 in normalized histogram). Then we make a 1D histogram filled with the normalized responses of each cell and divide this histogram into 12 equally populated bins (so each bin represents approximately the same number of detector cells, shown on the left plot of Fig. \ref{figFiberBrightnessBins}). The mean normalized response in each bin represents the relative brightness value of this bin (right plot of Fig. \ref{figFiberBrightnessBins}).



%%% How do we divide the calibration. List the four parts (how about timing calibration? - talk about in DAQ)

\todo{Describe the absolute and relative calibration just in text}
The NOvA calibration consists of two parts \cite{NOVA-doc-7410}:
\begin{enumerate}
\item The \textbf{relative calibration} corrects for attenuation of scintillator light as it travels through the cell to the readout, as well as for differences between detector cells. This correction is calculated for each cell separately.
\item Followed by the \textbf{absolute calibration}, which only uses stopping muons when they are minimum ionising particles. In the absolute calibration we calculate a scale between the measured energy deposition, corrected by the relative calibration, and the simulated energy deposition in physical units of $\unit{MeV}$. This scale is calculated for each time period and each detector separately, which ensures the energy deposition is directly comparable wherever or whenever it occurred.
\end{enumerate}

%Calibration is necessary to convert electronic signals to physically meaningful energy in units of GeV. Two calibration steps precede the calorimetric energy calibration. First raw ADCs (Analogue Digital Conversion) are converted to units of photo-electrons (PE) using the known average response of the APDs; secondly an attenuation calibration corrects for the position dependent response [6]. A drift calibration may be included in the future to correct for changes in detector response over time. The calorimetric energy scale calibration is the last step in the calibration chain and the detector response should already be uniform in space and eventually also in time. [docdb:13579 - FA\_Calorimetric\_energy\_scale]

%Using the average expected APD response, integrated charge from the ADCs are converted to units of photo-electrons (PE) [SA Absolute energy scale]

%I think I should actually include some kind of a description of the ADC to PE conversion.
\iffalse
The scaling of the ADC to PE depends only on the gain and the version of the FEB. Otherwise it's just a very simple scaling (explain this at the PE definition):
\begin{equation}
PE=\frac{\textsf{peakADC}}{\textsf{ADCPerPE}},
\end{equation}
\begin{equation}
\textsf{ADCPerPE}=\textsf{Gain}\times\frac{4095}{ADCScale}
\end{equation}
where ADCScale is 217000 for FEBv4.1 and 204800 for FEBv5.2.
\fi

%The PECorr scaling is 75.0 (NDOS), 37.51 (ND), 39.91 (FD) and 39.91 (TB)

\todo{Just describe these units in text instead of here}
The basic units and variables used to define energy deposited in the NOvA detectors are listed in table \ref{tabCalibrationVars}.
\begin{table}[!ht]
\centering
\def\arraystretch{1.4}
\begin{tabular}{m{0.1\textwidth} m{0.84\textwidth}}
ADC & The digitized charge collected by the APDs from the \textbf{A}nalog to \textbf{D}igital \textbf{C}onverter \cite{NOVA-doc-13518}.\\
PE & Number of \textbf{P}hoto \textbf{E}lectrons. Calculated by a simple rescaling of the best estimate of the peak ADC. The PE per ADC scale only depends on the FEB type and the APD gain settings. This conversion is done before the calibration and PE serves as the base unit for calibration.\\
PECorr & \textbf{Corr}ected \textbf{PE} after applying the relative calibration results. The correction is a ratio between an average energy response (a pre-determined semi-arbitrary number) and the result of the the relative calibration fit (also called attenuation fit), which depends on w, cell, plane, epoch and detector. This makes the energy response equivalent across each detector.\\
MEU & \textbf{M}uon \textbf{E}nergy \textbf{U}nit is the mean detector response to a stopping cosmic minimum ionising muon.  For true variables it's equivalent to the mean MeV/cm and for reconstructed variables to the mean PECorr/cm.\\
MeV & Estimated energy deposited in the scintillator calculated from PECorr using the results of the absolute calibration. Additional correction for dead material needs to be made in order to get an estimate of the calorimetric energy.
\end{tabular}
\caption{Definitions of variables commonly used in calibration \cite{NOVA-doc-13579,NOVA-doc-7410}.}
\label{tabCalibrationVars}
\end{table}

\todo{Change this equation to be simpler and also include T/S corrections}
The final result of the NOvA calibration is the deposited energy in terms of physical units, which is in effect calculated as:
\begin{equation}
\begin{tikzpicture}[baseline=(current  bounding  box.center)]
\node {\(\displaystyle
E_{dep} [\unit{MeV}]=\frac{\textsf{MEU}_{truth} [\unit{MeV/cm}]}{\textsf{MEU}_{reco} [\unit{PECorr/cm}]}\times \frac{\textsf{Average response} [\unit{PECorr}]}{\textsf{Fitted response} [\unit{PE}]}\times \left[\frac{\unit{PE}}{\unit{ADC}}\right] \times \textsf{Signal} [\unit{ADC}],
\)};
\draw[decorate,decoration={brace,amplitude=10pt,mirror},red] (-5.4,-0.6) -- (-1.8,-0.6) node (A) [midway,yshift=-20pt]{Absolute calibration};
\node[below,yshift=-10pt,red] at (A) {(Detector, epoch)};
\draw[decorate,decoration={brace,amplitude=10pt,mirror},blue] (-1.2,-0.6) -- (3.2,-0.6) node (B) [midway,yshift=-20pt]{Relative calibration};
\node[below,yshift=-10pt,blue] (B2) at (B) {(Detector, epoch,};
\node[below,yshift=-10pt,blue] at (B2) {plane, cell, w)};
\draw[decorate,decoration={brace,amplitude=10pt,mirror},green!80!black] (3.8,-0.6) -- (5.1,-0.6) node (C) [midway,yshift=-20pt]{Scale};
\node[below,yshift=-10pt,green!80!black] at (C) {(APD Gain, FEB)};
\end{tikzpicture}
\end{equation}
where both the relative calibration results (blue fraction) and the absolute calibration results (red fraction) are stored in a database and applied together with the ADC-to-PE scale during processing of every hit in the NOvA detectors.

%The light is attenuated while traveling through the fiber. To find the correct energy of the incident particle these losses are corrected by using cosmic ray muons. The cosmic ray muons are used to calibrate the NOvA detectors because they provide a source of consistent energy across the detectors. The purpose of the attenuation calibration is to provide constants and formulae such that an amount of energy deposited in the detector and registered by an APD can be expressed in comparable units, PECorr which are the corrected photo-electrons (PE) no matter where the deposition occurred. Variations in time are to be handled by the drift calibration. The purpose of the absolute calibration is to provide a scaling factor, independent of channel since all of that variation should have been taken out by the relative calibration, so that energy deposits can be expressed in physically meaningful units (GeV).
%For both purposes cosmic rays are used as probes. For the attenuation calibration they represent a source of consistent energy deposits across the detector of approximately 1 minimum ionizing particle’s energy, MIP, but this is not assumed. Any average value consistent across the detector would do. For absolute calibration, stopping muons are used, whose precise energy deposits should be estimateable from the Bethe Bloch formula. [docdb:13579 - SA The Attenuation and Threshold Calibration of the NOvA detector, but a lot of this is actually just copied from Backhouse's original calibration technote docdb:7410]

%(Dividing data into periods and epochs) A new period is started for a major change to running conditions such as a horn current change, a long shutdown, target replacement, etc. Periods are divided into epochs. A new epoch is started whenever analysis or production reasons dictate. Calibration has been performed for all the periods separately and has used the data that are determined by the Data Quality group to be good. The effects of aging, temperature, partial filling, and cooling are neglected. The drift calibration should be able to account for all of these (but drift calibration doesn't really exist yet afaik). [docdb:13579 - SA The Attenuation and Threshold Calibration of the NOvA detector]


\subsection{Threshold and shielding correction}

Energy deposited far away from the readout may get attenuated enough to be shifted below the threshold. These low energy depositions would be missing from the attenuation fit, biasing it towards larger light levels with increasing distance from the readout. Similar effect, specifically for the vertical cells, is caused by using cosmic muons for calibration and applying it to beam muons. The top of the detector effectively shields the bottom, skewing the energy distribution of cosmic muons. To correct for both of these effect, we use the simulation pclist sample to calculate the threshold and shielding (also called threshold and shadowing) correction by comparing the true and reconstructed information. We apply this correction before the attenuation fits \cite{NOVA-doc-13579-SAAttenuationAndThreshold}.

%Should I write anything more? Maybe about how do we calculate this more specifically, or that it's done for view X fb bin X cell X w

%In the Far Detector data and MC a large divergence between calibrated and true energies as a function of W was observed [8]. This was traced back to the much longer cell lengths in the FD meaning that thresholds play a large role at the foot of a cell. Also self-shielding of the detector by its own mass lay a role in the observed discrepancy. Thresholds mean that for a hit to be seen by an APD, it may need to have a slight upwards fluctuation in the number of photons produced by the energy deposition. Self-shielding means that the average visible energy depositions from MIPs are not truly spatially uniform in the detector. If not corrected for these effects, there will be a bias in the set of hits that the attenuation fit sees, and leads it to overestimate the light-level, and so under-estimate real hit energies by tens of percent. The approach adopted to solve this problem was to create a correction factor as a function of view, cell, and position along the cell which would be applied before the attenuation correction to remove the effect of thresholds and shielding. To this end MC truth information about the calibration hit sample is used to create a combined threshold and shadowing correction for each cell and view combination,
%\begin{equation}
%T=\frac{PE}{\lambda}\frac{E_{true}}{E_{mip}},
%\end{equation}
%where $T$ is the combined “threshold and shielding” correction factor, $PE$ is the simulated photoelectrons recorded at the readout, $\lambda$ is the number of simulated photons which would be seen at the readout out in the absence of fluctuations, $E_{true}$ is the true energy deposited in the cell and $E_{mip}$ is the naive energy you would expect to be deposited based on the pathlength through the cell. In this way it encodes a threshold correction based on the simulated readout PE with and without the fluctuations, with $\lambda$ dependent on your simulated threshold, as well as a shielding correction based on the simulated energy deposition and a naive no shielding approximation. This equation gives us a cell by cell correction but we use an empirical polynomial fit to that distribution which removes statistical noise from the correction and well describes the initial distribution. This correction factor is applied to the cell by cell data and MC PE/cm distributions before the attenuation fits. [docdb:13579 - SA The Attenuation and Threshold Calibration of the NOvA detector, reference 8 is for docdb:7247, a talk by Backhouse]

\subsection{Relative calibration}\label{secRelativCalibration}

%Detailed description can be found in the "Instructions for the Attenuation Calibration Job" technote from Prabhjot from docdb:13579 (list of all calibration technotes) and on the relative calibration wiki page.

Relative calibration corrects for the attenuation of the scintillator light by fitting the average detector response over the position in each cell (w), separately for every cell inside each detector. Dividing the "average response" of the detector by the result of the attenuation fit for each $\textsf{Plane}\times\textsf{Cell}\times\textsf{w}$ combination effectively removes relative differences within and between all cells across the entire detector. The average response is a single constant number chosen to approximately represent the average response in the middle of the cell. Its value is for the far detector and Test Beam 39.91~PE/cm and for the near detector 37.51~PE/cm. The value of the average response has no impact of the calibration results, as the absolute scale of the detector response is determined during the absolute calibration and relative calibration only serves to remove the relative differences \cite{NOVA-doc-7410,NOVA-doc-13579-SAAttenuationAndThreshold}.
 
To create the attenuation fit we use the following procedure \cite{NOVA-doc-7410}:
\begin{enumerate}
\item Create the \textit{attenuation profiles}. Attenuation profiles are essentially profile histograms of detector response in terms of $\unit{PE/cm}$ as a function of position in the cell (w) for each cell in all planes. We construct the attenuation profiles over a little wider range than the actual length of the cell and always with 100 bins for each detector. This means that smaller detectors, like the Test Beam detector, have a finer binning ($\sim 3\unit{cm}$/bin) compared to the Far Detector ($\sim 18\unit{cm}$/bin).
\item Analyse the attenuation profiles. The job to create the attenuation profiles also creates validation histograms, which should be analysed prior to performing the attenuation fit to make sure the attenuation profiles look as expected.
\item Apply the threshold and shielding correction that was created before the relative calibration.
\item Do the attenuation fit over the full length of each cell. The fit consists of
\begin{enumerate}
\item an exponential fit, which combines two cases. First, when the scintillating light travels the short distance straight to the readout, and second, when it goes to the far side of the cell and loops around before going to the readout. The fitted function has a form:\\
\begin{equation}
y=C+A\left(\exp\left(\frac{w}{X}\right)+\exp\left(-\frac{L+w}{X}\right)\right),
\end{equation}
where $y$ is the fitted response, $L$ is the length of the cell and $C$, $A$ and $X$ are the fitted parameters. $X$ also represents the attenuation length.
\item Smoothing of the residuals from the exponential fit, mainly at the end of cells, with the LOcally WEighted Scatter plot Smoothing (LOWESS) method.
\end{enumerate}
\item Check the plots of the attenuation fit for a selection of cells.
\item Save the fit result to the database in the form of two csv tables. The \texttt{calib\_atten\_consts.csv} table holds the results of the exponential fit, together with the final $\chi^2$ of the fit. The \texttt{calib\_atten\_points.csv} table holds the results of the LOWESS smoothing.
\end{enumerate}

To ensure the quality of the attenuation fit, we only apply the results if the final $\chi^2<0.2$. If $\chi^2>0.2$, we ignore the results for this cell and mark it as \textit{uncalibrated}.

%Attenutation profiles have a constant binnin fNBins=100 (in w), same for ND, FD and TB. This results in an effectively finer binning for TB compared to ND and FD. For FD w = (-900,+900), ND: (-250,+250), TB: (-150,+150). TB: 3cm/bin, ND: 5cm/bin, FD: 18cm/bin. What effect could this have on the relative calibration results? Particularly on the calibration shape?

%where y is the response, L is the cell length, C, A and X are the free parameters in the fit. X gives the attenuation length as well. Initially, the fit is to the central part of the cell, which is different for each detector. In addition to the approximately quartic behavior at the ends of every channel there are in many channels fairly large residuals. They don’t appear to follow any consistent pattern. The leading hypothesis is that these are due to varying fiber position within the cell. Usually the fiber lies in the corners of the cell, but if it is somehow twisted so that it rises into the center of the cell, then it should collect more light, to an extent comparable to what is seen here. To remove such an irregular pattern, the residual from the analytic fit is simply fit with LOcally WEighted Scatter plot Smoothing, LOWESS. The LOWESS curve at each point is formed from the weighted mean of the deviations. The weighting function is the traditional tri-cube, (insert equation, likely not needed for this technote) [docdb:13579 - SA The Attenuation and Threshold Calibration of the NOvA detector, already in 1stAna and Backhouse's technote]

%For NDOS the fit was a very little bit different, where we didn't use $L$ but $3L/2$. Also it says that "Over the length of an NDOS cell, the effect of the long attenuation length is imperceptible, and is modelled as a constant (If you put a long attenuation term in, the fit drives the length scale to infinity anyway). [docdb:7410]

%In many channels, fairly large residuals are visible. They don’t appear to follow any consistent pattern. The hypothesis is that these are due to varying fibre position within the cell. Usually the fibre lies in the corners of the cell, but if it is somehow twisted so that it rises into the centre of the cell, then it should collect more light, to an extent comparable to what is seen here. To remove such an irregular pattern, the residual from the analytic fit is simply fit with LOWESS (locally weighted scatterplot smoothing). The LOWESS curve at each point is formed from the weighted mean of the deviations. The weighting function is the traditional tri-cube:
%\begin{equation}
%w_i=\left(1-\left|\frac{x-x_i}{\sigma}\right|^3\right)^3.
%\end{equation}
%The smoothing length scale $\sigma$ is 30cm. 20 points calculated by this method are stored, to be linearly interpolated between to approximate the full LOWESS curve. If the LOWESS fit at any point exceeds 15\% the original attenuation fit was very bad, and the channel is marked uncalibrated. Figure 4 shows an example of large (10\%) deviations being fitted. This variation is not seen in the MC, and so the LOWESS fit is skipped there. Due to the lower stats available in MC, instead of being collated by plane and cell, the curves are only calculated by view and cell. [docdb:7410]

%The current value of $\sigma$ in the code is $1.5\times\textsf{DetWidth}/20$

\subsection{Absolute calibration}

To find the absolute energy scale, we apply the relative calibration results on the stopping muon sample and look at the energy they deposited in cells 1-2 meters from the end of their tracks. In this track window they are approximately minimum ionising particles and their energy deposition is almost constant and well understood. Additionally, we don't use hits from the edges of a cell, as those might be affected by the lower number of events, fibre endings, or loops. 

For each calibrated data and simulation sample we take a mean of the corrected deposited energy distribution, separate for each view. We then take a simple average from the two views to get the final $\textsf{MEU}_{reco}$ in units of $\unit{PECorr/cm}$ for each sample \cite{NOVA-doc-13579-FACalorimetricEnergyScale}. Additionally, from simulation we can get the mean of the distribution of the true deposited energy in the scintillator, $\textsf{MEU}_{truth}$ in units of $\unit{MeV/cm}$ for the same sample of stopping muons. 

We ignore the energy that is lost in the dead material (PVC extrusions) and deal with it separately. The absolute energy scale for each sample is then the ratio of $\textsf{MEU}_{truth}/\textsf{MEU}_{reco}$. We save these absolute energy scales in another csv table called \texttt{calib\_abs\_consts.csv} which stores the $\textsf{MEU}$ values and their errors.

As part of the absolute calibration we also produce validation plots that show the effect of calibration on the distribution of the stopping muons. We analyse these plots and if everything looks all right we load all the csv tables into the database.

%Stopping muons provide a good sample of known energy deposits. If we can collect a “golden” sample, they should provide the scale factor to convert PECorr to GeV. So far, the method used has been imperfect, and the absolute calibration constants are known to be off by approx. 10\%. Since a factor already has to be derived to correct for dead material, this is not significantly impeding current efforts, but work was recently gone into improving this area. [docdb:7410 - this was likely before the track window cut was introduced] (Here it says that it's not such a big a problem since we have to scale for the dead material anyway. But nowadays we have to account for a large systematic uncertainty in the absolute energy scale in our analyses. How is the dead material correction different from the energy scale uncertainty?)

%...the calibration of the calorimetric energy scale of the NOvA detectors uses the energy deposited by stopping muons as a standard candle. To reduce systematic uncertainties, only those energy deposits in a 1-2 m window away from the muon track end point are used. The mean of the detector response distribution is found for data and MC in both near and far detectors. The mean of the distribution of true energy deposits in the track window is used to provide a conversion factor between the detector response and the true energy deposited in the scintillator for minimum ionising muons. The simulated dE/dx is uniform within about 1.8\% for hits around the minimum between 100-200 cm from the track end. The energy that a muon deposits within each cell is estimated using Geant 4 and stored in Fibre Liquid Scintillator (FLS) hits. FLS hits are only those within the active material (liquid scintillator) and energy loss within the passive material (plastic extrusions) is ignored. an estimate of the minimum energy loss rate of stopping muons in the NOvA scintillator is found to be,
%\begin{equation}
%\left.\frac{dE}{dx}\right|_{\textsf{mip}}=\left(1.7915\pm 0.0035\right)\unit{MeV/cm}.
%\end{equation}
%For stopping muons in NOvA it is also important to consider their decay. The muon has a vacuum lifetime of about 2.2 microseconds and favourably decays, with a branching ratio approx. 100\%, into an electron, an electron anti-neutrino and a muon neutrino. The electron produced in this decay is called a Michel electron and is used to select muons that stop within the NOvA detectors. The energy scale calibration is performed using cosmic ray muons. The calibration measures the detector response in data and MC in both near and far detectors and normalises them all by providing a conversion factor, for all four cases, that converts the detector response to energy in GeV. The energy loss rate (dE/dx) of stopping muons is well described by the Bethe-Bloch and is a function of the distance from the stopping point. A track window technique is used to minimise the variations in detector response that depend on the distance to the track end. Using this technique only hits within a region of distances from the track end are used. The position of the track window is chosen such that a mis-reconstruction of the track end point has the minimum effect on the mean detector response. The track window is currently set to be in the range from 100 cm to 200 cm from the track end.[docdb:13579 - FA\_Calorimetric\_energy\_scale]

%\FloatBarrier
%\newpage

\section{Energy estimation}
\note{Should I include this section or not?}

\section{Systematic Uncertainties at NOvA}\label{sec:NOvASystematics}
\todo{Describe the general systematic uncertainties for NOvA}
\note{These subsections below might end up just being paragraphs, depends how much I want to write about them}

\subsection{Systematic Uncertainties Related to the NOvA Neutrino Beam}

\todo{Describe the Hadron production and focusing systematic uncertainties}

\todo{Principal component analysis}

\note{Maybe briefly also mention the POT scaling normalization uncertainty.}

%From docdb:54582: NOvA analyses use two sets of flux uncertainties: beam transport, which cover differences between the simulation and the working conditions of the NuMI beam, and hadron production, which concern the rates of production of pions and kaons from proton collisions on the carbon target. The beam transport uncertainties include the horn and target position, the horn current, the beam position on the target, the beam spot size, and the effect of the Earth’s magnetic field in the beam pipe (which is not simulated in G4NuMI). The effect of each uncertainty is below 5% at the flux peak for both the ND and FD. PPFX is used to constraint the hadron production models for the NuMI beam using external hadron production data and theory. The systematic effect is assessed by generating 100 universes where the uncertainty on the external data and theoretical assumptions are allowed to float. Flux uncertainties are known to show correlations across most neutrino energy bins, especially in off-axis measurements like NOvA. NOvA looks at 100 PPFX universes, and for each PPFX universe we further sample the beam transport parameters 20 times, creating a total of 2000 universes representing many different possibilities of the flux uncertainty. These universes are used to estimate the bin-to-bin covariances in true energy for each neutrino flavour, detector and beam mode. A Principal Component Analysis (PCA) is applied to the covariance matrices to find sets of uncorrelated shifts - Principal Components (PCs). The covariance matrices for each neutrino flavour and beam mode are calculated in the (ND, FD/ND) basis to represent the approximate extrapolated error at the Far Detector. These are then diagonalized to give the Principal Components, given by the eigenvectors i.e, $PC_i=\sqrt{\lambda_i}v_i$, where $\lambda_i$ represents the $i$-th largest eigenvalue and $v_i$ its corresponding eigenvector. The components are then converted to the (ND, FD) basis. Each PC can be used as a systematic shift in the oscillation fit with a pull of $1\sigma$. By ordering each of the PCs in terms of the magnitude of their eigenvalues, one can also capture most of the information embedded in the covariance matrix with just a few components. Five principal components are used in the oscillation fit. The hadron production uncertainty on the neutrino flux is evaluated using a "multi-universe" technique. This is $\sim 7\%$ at the spectrum peak dominated by interactions where relevant data to be included in the constraining procedure is currently not available (mostly mesons and proton quasi-elastic interactions). Beam transport uncertainties are incorporated by propagating uncertainties in the alignment of beamline elements, including the beam position on the target, the horns current and position, the beam spot size, and the effect of the Earth magnetic field in the decay pipe. The beam optic uncertainty is $\sim 4\%$ at the peak.

\subsubsection{Constraining the Hadron Production Systematic Uncertainty in NOvA}
%Again, should I discuss it here or somewhere else? Maybe not necessary as a full section. Not sure if I should include a discussion on nu-on-e, or low nu studies here, or just PPFX improvements.

\subsection{Systematic uncertainties for NOvA detectors}

\subsubsection{Neutrino interaction systematic uncertainties}

\subsubsection{Energy scale systematic uncertainty}

%WORK IN PROGRESS

%First Analysis systematic uncertainties due to calibration:
%Sources of systematic uncertainty of particular concern are those introduced by residual variations remaining after calibration. Systematic errors are introduced by spatial and temporal variations in detector response. Further, any difference between the two detectors may introduce a relative shift in the energy scale between the detectors. A source of systematic uncertainty can be introduced by mis-reconstructing the end point of the muon track. Such a mis-reconstruction would shift the window within which hits are selected and hence the dE/dx of the muon.  The figure shows that the detector response varies by up to about 60\% over the range from 0 to 500 cm to the track end. This large variation illustrates the importance of careful consideration of the track window position and size. The detector response for both data and MC is minimum at about 130 cm from the track end and is flat to about 1\% in the range from 100 cm to 200 cm from the track end. For a track window starting at 100 cm from the track end, a conservative mis-reconstruction of the track end point by 10cm will shift the start of the track window to between 90cm and 110cm. This shift will alter the MEU value by less than 0.4\% over the range.
%If the calibration procedure was ideal the detector response would not vary with position in either data or MC. The calibration is not ideal and the detector response and recorded simulated energy deposition varies with position of the hit within the detector, such variations will introduce systematic errors. The position of a hit can be defined by the plane, cell within the plane, and distance along the cell (w) of the hit. The variation in detector response and simulated energy deposition vs. plane, cell and w for each view has been studied to quantify the systematic uncertainty introduced by these sources.
%The rise in detector response at the far end of FD y-view cells is an issue with several potential sources. The rise in response may be due to an acceptance effect or a light-level threshold effect among other possibilities. An acceptance effect is where greater energy must be deposited at the far end of the cells so that the light can travel along the fibre, hit the APD and be recorded as a hit. Both an acceptance effect and a light-level effect would introduce a bias towards higher energy hits toward the far end of cells.
%Another source of systematic uncertainty is introduced by the variation in detector response with time. The FD response is stable to about 1\% during the period from October 2014 to March 2015. The ND response needs further study but there was no significant trend over 6 months at 5\%. 
%As mentioned in Section 5, the version (7.1) of the calibration used for first analysis has been adjusted based on studies of muons from beam neutrinos interacting in the detector [8]. A shift of 3.6\% was introduced based on the average response of muons where large sections of the track were used. When only a track window of 100-200cm is used on the beam muons the difference is only 2.7\% [8]. Our best hypothesis for this residual 2.7\% difference is that it is caused by showery events that are present in ND data but not ND MC: it was shown in [9] that doing the calorimetric energy scale calibration using a truncated mean (or a median or a fit to the peak) gave a data/MC ratio that differed by 2.7\% compared to using the untruncated mean as described in this document. A comparison of various cross checks of the calorimetric energy scale was undertaken (in [10] and [11]) and concluded that the nearly 5\% difference between ND data and MC seen in a sample of Michele electrons [12] should be applied as both an absolute and relative shift to the calorimetric energy scale. The difference between the level of calorimetric energy resolution of stopping muons was studied and it was found that data and MC agreed best when an 8\% additional smearing was introduced. Studies for the NuMu analysis indicated that this was a negligible systematic uncertainty [13]. 
%[docdb:13579 - FA\_Calorimetric\_energy\_scale]

\subsubsection{Cell edge calibration systematic uncertainty}

\subsubsection{Detector ageing systematic uncertainty}

\note{Should I include the neutron systematics, muon energy scale systematic, or tau scale systematics? Are these detector systematics? Should find out...}

%Also include Chenerkov and light level tune uncertainties
%\chapter{NOvA Test Beam detector calibration}\label{sec:TBCalibrationSection}

In this chapter I describe the details of the Test Beam detector calibration as it was finalized in June 2023. This version includes a new purpose-made data-based simulation of cosmic muons as described in Chapter~\ref{sec:DataBasedSimulation} and all the measured Test Beam data, with the exception of the period 1 data.

The data calibration samples for Test Beam were created using the same procedures as the \gls{ND} and \gls{FD} calibration samples, described in Sec.~\ref{sec:NOvACalibration}. However, there are two cuts from the event election, that were not included for Test Beam during the processing of the data samples. This can be seen on Tab.~\ref{tab:DataBasedSimEventSelection}, where the two bottom rows show the two excluded cuts. One cut contains the vertex close to the edge of the detector ensuring we only use cosmic events, the other contains the end of track close to the edge, ensuring we only use through-going muons for the relative calibration. Given that we remove beam events and that all the other cuts are designed to select cosmic events, the first cut has only a negligible effect on the final selection. Additionally, the stopping muons only make up a small fraction of the total cosmic muon events, rendering the second cut also negligible.

This section is organized as follows. I first describe the Test Beam versions of the fibre brightness map in Sec.~\ref{sec:FibreBrightnessTB} and the threshold and shielding correction in Sec.~\ref{sec:TBThresholdCorrection}, as they were introduced in Sec.~\ref{sec:NOvACalibration}. I then go over the simulation sample and the three data samples (for periods 2, 3, and 4) in Sec.~\ref{sec:SimulationResults}-\ref{sec:TBPeriod4}, showing distributions of hits selected for calibration and of the uncorrected energy deposition before calibration. I discuss considerations going into calibration, including splitting the individual periods into smaller samples, or describing issues that could affect the calibration results.  Afterwards, I am showing a selection of attenuation fit results for each sample together with an overview of the relative calibration effects. Lastly, I discuss the absolute calibration results in Sec.~\ref{sec:TBAbsoluteCalib} for all the samples combined, as well as the validation and conclusion of the Test Beam calibration in Sec.~\ref{sec:TBCalibValidation} and \ref{sec:TBCalibSummary} respectively.

%Temperature study (small overview - probably not needed at all, depends if Randeeth want to add his work to this technote)

%From Teresa's thesis
%Along with setting the energy scale of the detector, we need to calibrate the timing of the readout system for the detector. The Data Concentrator Modules (DCMs) responsible for collating the data from multiple FEBs get their timing information via a daisy chain originating at the detector TDU. Each DCM in the chain has a timing offset relative to the DCM before it, with the last DCM having the earliest ti. Following the procedure described in [66], I used timing information from hits on cosmic ray muon tracks that pass through multiple DCMs to determine the relative offsets between DCMs, shown in Figure 3.20.

%From Teresa's thesis:
%"For Test Beam, we have three beam-based triggers, one pulsed trigger, and two data-driven triggers. The data-driven triggers are both activity-based triggers. The first is intended to record cosmic ray induced events for use in calibrating the detector.

\section{Fibre brightness}\label{sec:FibreBrightnessTB}

To divide the Test Beam detector into \gls{FB} bins we use the attenuation fit results for Test Beam period 4 data (described in Sec.~\ref{sec:TBPeriod4}), as that is the best detector conditions data we have. Since we need the \gls{FB} map in order to run the attenuation fits and we need the attenuation fit results to create the \gls{FB} map, we proceeded iteratively. We first run the attenuation fit with an older version of the \gls{FB} map and use the results to create a new \gls{FB} map, discussed here, which is then used in a new attenuation fit.

We are only using the attenuation fit results in the centre of each cell to create the \gls{FB} map, therefore, we decided to allow some cells that failed the calibration condition ($\chi^2>0.2$), to be still used for the creation of the \gls{FB} map. Otherwise, all the officially uncalibrated cells are assigned an average response across the entire detector, resulting in a loss of information on their relative brightness. As can be seen in Fig.~\ref{fig:FiberBrightnessExamples}, some attenuation fits have $\chi^2>0.2$, even though they correctly represent the energy deposition in the centre of that cell. By carefully investigating all the uncalibrated Test Beam cells (doable for Test Beam, due to its small number of cells), we concluded that all the cells with $\chi^2<0.7$ can be used to create the \gls{FB} map, since the response in their centre is described reasonably well by their attenuation fits. We use this loosened calibration condition only to create the \gls{FB} map and we keep the original condition for the actual calibration results.

%Describe and show plots that since we are only using the fitted response at cell centre we can allow fits with $\chi^2>0.2$. Show examples of responses with chisq larger than that and say what is the final chisq chosen. No need to show the final distribution of the fb bins here as they were technically shown in the general calibration description. But might refer back to it...

\begin{figure}[h]
\centering
\begin{subfigure}[b]{0.495\textwidth}
\centering
\includegraphics[width=\textwidth]{Plots/TBCalibration/ExampleForBrightFile_fb0_030_000.png}
\end{subfigure}
%\hfill
\begin{subfigure}[b]{0.495\textwidth}
\centering
\includegraphics[width=\textwidth]{Plots/TBCalibration/ExampleForBrightFile_fb5_061_061.png}
\end{subfigure}
\caption[Example of failed attenuation fits used for the Test Beam fibre brightness file]{Examples of attenuation fits for two cells that fail the calibration condition, but the fit (blue line) still correctly represents the energy deposition in the centre of that cell (dashed vertical line in the middle). The total $\chi^2$ between the data (black) and the attenuation fit for both plots are included.}
\label{fig:FiberBrightnessExamples}
\end{figure}

The final distribution of relative \gls{FB} values that are used to populate the \gls{FB} bins for the Test Beam detector is shown in Fig.~\ref{fig:TBFiberBrightnessMap}. The resulting map of \gls{FB} bins and their corresponding relative brightnesses was shown in the previous chapter in Fig.~\ref{fig:NOvAFiberBrightness}.

\begin{figure}[hbtp]
\centering
\includegraphics[width=.8\textwidth]{Plots/TBCalibration/TBFiberBrightnessMap.png}
\caption[Fibre Brightness map for the Test Beam detector]{\acrshort{FB} map representing relative differences in energy response due to different brightnesses of the fibres, scintillators, or readout. Create from the attenuation fit results of the \acrshort{NOvA} Test Beam detector with a shifted calibration condition from $\chi^2<0.2\rightarrow 0.7$ to enable using the attenuation fits that are officially uncalibrated, but correctly represent energy deposition in cell centre. Otherwise, all the uncalibrated cells get assigned a mean detector response, represented by number 1 on this map.}
\label{fig:TBFiberBrightnessMap}
\end{figure}

\section{Threshold and shielding corrections}\label{sec:TBThresholdCorrection}
%Describe in generall what is TS correction for
The threshold and shielding correction is intended to mitigate biases arising from differences between cosmic events used for calibration and beam events. It is only used prior to the attenuation fits and is omitted when applying the results of the relative calibration, whether during the absolute calibration or for beam events. Additionally, it is derived exclusively from simulation.

%TS correction dependence on w
We created a new threshold and shielding correction for the Test Beam detector using the new simulation described in Sec.~\ref{sec:DataBasedSimulation}. The correction is calculated for both views, across 12 \gls{FB} bins, 64 cells, and 100 $w$ bins, where $w\in\left(\unit[-130]{cm},\unit[130]{cm}\right)$. Two examples of the correction as a function of $w$ are shown in Fig.~\ref{fig:TBThresholdCorrectionExamples}, demonstrating an almost uniform behaviour along a cell. Relative variations of the correction in the X view range from $\unit[1-2]{\%}$,  primarily concentrated at the edges of the cell. In the Y view, the correction exhibits sub-$\unit[1]{\%}$ variations. These trends are consistent across all the \gls{FB} bins and views. Given that the threshold and shielding correction precedes relative calibration, the absolute value of the correction is irrelevant and only the relative variations along $w$ and between cells matter.
 
\begin{figure}[!hbtp]
\centering
\begin{subfigure}[t]{\textwidth}
\centering
\includegraphics[width=\textwidth]{Plots/TBCalibration/ThresholdCorrectionExample_axview_fb0_P4DataBasedSim.pdf}
\end{subfigure}
\begin{subfigure}[b]{\textwidth}
\centering
\includegraphics[width=\textwidth]{Plots/TBCalibration/ThresholdCorrectionExample_ayview_fb3_P4DataBasedSim.pdf}
\end{subfigure}
\caption[Example threshold and shielding correction for Test Beam detector]{Examples of threshold and shielding corrections as a function of the position within a cell in X view (top) and Y view (bottom) for the Test Beam detector.}
\label{fig:TBThresholdCorrectionExamples}
\end{figure}

%Where do the variations come from?
This uniformity of the distributions is expected, considering the relatively small size of the Test Beam detector compared to the \gls{FD}, which prompted the investigation into threshold and shielding effects. The Test Beam detector's cell length of $\unit[2.6]{m}$ has only a negligible impact on the threshold saturation or on the energy distribution of cosmic muons, resulting in  the uniformity of the threshold and shielding correction for Test Beam detector. The larger correction at cell edges is likely caused by lower event counts in those areas. However, since this relative sparsity of events also influences relative calibration due to large variation in the energy response, the relatively larger threshold and shielding correction at cell edges is not detrimental.

%Dependence of the TS correction on plane and cell
The distribution of the threshold and shielding correction across Test Beam detector's cells and planes, shown in the top of Fig.~\ref{fig:TBThresholdCorrectionMap}, demonstrates, that while the correction is expected to be generally uniform across the detector, there are notable variations between cells and planes forming a discernible pattern. These variations and their shape primarily stem from the threshold component of the correction, shown in the bottom of Fig.~\ref{fig:TBThresholdCorrectionMap}.

\begin{figure}[!hbtp]
\centering
\begin{subfigure}[t]{\textwidth}
\centering
\includegraphics[width=.8\textwidth]{Plots/TBCalibration/ThresholdCorrectionMap_P4DataBasedSim.pdf}
\end{subfigure}
\begin{subfigure}[b]{\textwidth}
\centering
\includegraphics[width=.8\textwidth]{Plots/TBCalibration/ThresholdOnlyCorrectionMap_P4DataBasedSim.pdf}
\end{subfigure}
\caption[Map of the threshold and shielding correction across the Test Beam detector]{Map of the threshold and shielding correction (top) and only of the threshold part of the correction (bottom) as a function of the Test Beam detector's cell and plane number. Each bin shows the mean correction for all the simulated events in that cell.}
\label{fig:TBThresholdCorrectionMap}
\end{figure}

%What is the threshold correction
The threshold part of the correction can be expressed as
\begin{equation}
\textsf{Threshold correction}=\frac{\gls{PE}_{\mathrm{Poisson}\lambda}}{\gls{PE}_{\mathrm{Reco}}},
\end{equation}
where $\gls{PE}_{\mathrm{Poisson}\lambda}$ represents the mean of the Poisson distribution of the true deposited energy (in terms of $\gls{PE}_{\mathrm{True}}$), and $\gls{PE}_{\mathrm{Reco}}$ is the reconstructed number of \gls{PE} from simulation. Both $\gls{PE}_{\mathrm{Poisson}\lambda}$ and $\gls{PE}_{\mathrm{True}}$ are direct outputs of the light model simulation, as detailed in Sec.~\ref{sec:NOvASimulation}. After the light model simulation, $\gls{PE}_{\mathrm{True}}$ is passed through the readout simulation, which includes a \gls{PE}-to-\gls{ADC} function for calculating the peak \gls{ADC} value. This value is then converted into $\gls{PE}_{\mathrm{Reco}}$ using the \gls{ADC}-to-\gls{PE} scale described in Sec.~\ref{sec:NOvACalibration}. The observed shape in the threshold correction can thus be attributed to differences between $\gls{PE}_{\mathrm{True}}$ and $\gls{PE}_{\mathrm{Poisson}\lambda}$, as well as to various effects introduced by the readout simulation. However, the differences between $\gls{PE}_{\mathrm{True}}$ and $\gls{PE}_{\mathrm{Poisson}\lambda}$ are marginal (below $\unit[1]{\%}$) and contribute minimally to the threshold correction. Therefore, the predominant influence on the observed pattern comes from the effects introduced by the readout simulation.

%FEB versions
There are two prominent features in the threshold correction variations in Fig.~\ref{fig:TBThresholdCorrectionMap}. Firstly, the two blue vertical lines in planes 16-17 and 48-49. These planes are using the \gls{FEB} version 5.2, used in the \gls{ND}, instead of  the \gls{FEB} version 4.1, used in the \gls{FD} and in all the other Test Beam planes, as explained in Sec.~\ref{sec:TBExperiment}. Both the readout simulation and the \gls{ADC}-to-\gls{PE} scale do account for the expected disparity in the \gls{ADC}/\gls{PE} ratio between the two \gls{FEB} versions. However, it is expected that \gls{FEB}v5 would exhibit a lower response to the same energy compared to \gls{FEB}v4. Therefore, for the same $\gls{PE}_{\mathrm{Poisson}\lambda}$ values, the $\gls{PE}_{\mathrm{Reco}}$ for \gls{FEB}v5 should be smaller than that for \gls{FEB}v4. Consequently, the \gls{FEB}v5 planes should have a larger threshold correction compared to the \gls{FEB}v4. However, as was shown in Fig.~\ref{fig:TBThresholdCorrectionMap}, the observed correction is contrary to this expectation, suggesting a potential error in the readout simulation regarding the handling of different \gls{FEB} versions.

%APD relative gain map
The second notable feature in Fig.~\ref{fig:TBThresholdCorrectionMap} is the variation of the threshold correction across cells, which appears to be consistent across all planes, depicted by the presence of red horizontal lines. The origin of this dependency is in the \gls{APD} structure, where each \gls{APD} collects signal from 32 cells arranged in 4 rows of 8 \gls{APD} pixels, as explained in Sec.~\ref{sec:DAQ}. Manufacturing discrepancies \cite{NOvA-doc-5239} lead to relative gain variations among the \gls{APD} pixels, typically exhibiting either increasing or decreasing trend along each of the four rows. To incorporate these variations into the readout simulation, the mean relative gain across cells of every module (comprising 32 cells) is used in the \gls{PE}-to-\gls{ADC} function. Consequently, these variations are consistent across all modules in the simulated detector, despite their inherent randomness in actual data.

The distribution of the relative gain for each `pixel number' is shown on the left of Fig.~\ref{fig:TBThresholdCorrectionGainMap}. However, it is important to note that `pixel number' is a \gls{NOvA} jargon and does not correspond directly to the \gls{APD} pixel position or cell number; instead it denotes the purely technical routing of \gls{APD} pixels to the \gls{FEB} \cite{NOvA-doc-11570}. Therefore, the depicted distribution of gain variation on the left of Fig.~\ref{fig:TBThresholdCorrectionGainMap} is incorrect and should instead describe the distribution with respect to the cell number rather than the `pixel number'. Simply translating `pixel numbers' to cell numbers yields the distribution shown on the right of Fig.~\ref{fig:TBThresholdCorrectionGainMap}. Comparing this to the positions of the red horizontal lines in Fig.~\ref{fig:TBThresholdCorrectionMap} demonstrates that this (incorrect) relative gain variation is responsible for the observed pattern in the threshold correction.

\begin{figure}[!hbtp]
\centering
\begin{subfigure}[t]{.495\textwidth}
\centering
\includegraphics[width=\textwidth]{Plots/TBCalibration/ReadoutSimulation_GainPixelMap.pdf}
\end{subfigure}
\begin{subfigure}[t]{.495\textwidth}
\centering
\includegraphics[width=\textwidth]{Plots/TBCalibration/ReadoutSimulation_GainCellMap.pdf}
\end{subfigure}
\caption{The relative gain variation as a function of the `pixel number' (left) and cell number (right).}
\label{fig:TBThresholdCorrectionGainMap}
\end{figure}

%Conclusion
In summary, the threshold and shielding correction exhibits significant variations concentrated within specific planes and cells, arising from various effects in the readout simulation. However, it is evident that these effects are not limited to cosmic events and therefore should not be incorporated into the threshold and shielding correction. Given that these effects are corrected out for simulation before the attenuation fits, they are not accounted for in the relative calibration and therefore remain present for the absolute calibration and for beam events. Moreover, the two main effects outline above are not implemented into the simulation correctly, resulting in discrepancies between actual data and simulation. This means, that in data these variations are either not present or present in a different way than in simulation. Therefore, applying the simulation-based threshold and shielding corrections to data introduces new variations that would otherwise not exist for data. As a result, these new variations are incorporated into the attenuation fits for data, resulting in incorrect relative calibration results applied to both absolute calibration and beam events.

%future, solutions
Several approaches can address these issues. For simulation-related discrepancies, the only viable solution is to rectify the identified faults and to remake the simulation, albeit this would be computationally very intensive. However, for data-related concerns, efforts are underway to devise a new data-driven threshold and shielding correction \cite{NOvA-doc-15223}, eliminating any influence of simulation on the relative calibration of data. If a purely data-driven correction is not viable, there is another possible improvement to the threshold correction while still using simulation, which is to not use the $\gls{PE}_{\mathrm{Poisson}\lambda}$ directly, but to pass it through the readout simulation in the same way as $\gls{PE}_{\mathrm{True}}$ and create an alternative $\gls{PE}_{\mathrm{Poisson}\lambda\mathrm{Reco}}$.

\section{Simulation}\label{sec:SimulationResults}
The distribution of tricell hits from the simulated cosmic muon events selected for calibration, mapped across the Test Beam detector's planes and cells, is shown in Fig.~\ref{fig:CalibhistSim}. As this is a simulated detector, we will use this `ideal conditions' distribution of tricell hits to illustrate the main features, which are also present in all the data samples discussed below. We can clearly see the difference in the number of events between the vertical (even) and the horizontal (odd) planes. This is expected as cosmic muons are generally vertical and a single cosmic track often passes more horizontal planes than vertical planes. We can also see that due to the tricell condition there are no hits in cells 0 and 63, which are on the edge of the detector. These cells can still be calibrated by including hits from the `z tricell' condition, which is not shown in the plot. The three clear horizontal lines of relatively lower response going across the detector correspond to pairs of cells 15 + 16, 31 + 32, and 47 + 48. Together with cells 0 and 63, they represent the first and the last cells of each 16 cell-wide extrusion, which makes up half of a module, which in turn makes up half of a Test Beam plane. As was mentioned in Sec.~\ref{sec:NOvADetectors}, these cells are $\unit[3]{mm}$ narrower than the rest, resulting in fewer hits and a lower deposited energy. However, using the deposited energy divided by path length for calibration should compensate for this effect. Overall, Fig.~\ref{fig:CalibhistSim} demonstrates that the tricell hits are distributed fairly uniformly in the centre of the detector, with the number of hits dropping off towards the front, back and corners of the detector. This is a result of the event selection applied to the cosmic tracks for calibration.

\begin{figure}[h]
\centering
\includegraphics[width=\textwidth]{Plots/TBCalibration/Attenprofs_Simulation_CellPlane.pdf}
\caption[Plane-Cell distribution of hits for the simulation sample]{Distribution of tricell hits used for the calibration of the simulated Test Beam detector. Features are described in text.}
\label{fig:CalibhistSim}
\end{figure}

The distributions of deposited energy per path length though the cell before the calibration in units of $\unit{PE/cm}$ as a function of $w$, cell and plane number, are shown in Fig.~\ref{fig:Calibhist_simulation}. These distributions should be uniform after applying the results of the calibration and can be used to identify the main features that will need to be corrected for during the calibration. The shallow rise of the energy response along $w$ is caused by the attenuation of light along the \gls{WLS} fibres. The drop in the response at the edges of the cell is caused by the fibres looping and connecting to the \glspl{APD}, while the larger statistical uncertainties at the edges of the cell reflect the lower number of hits passing the event selection including the tricell condition.

\begin{figure}[h]
\centering
\begin{subfigure}[b]{0.495\textwidth}
\centering
\includegraphics[width=\textwidth]{Plots/TBCalibration/Attenprofs_Simulation_WPE_corr_xy_X_Prof.pdf}
\end{subfigure}
\begin{subfigure}[b]{0.495\textwidth}
\centering
\includegraphics[width=\textwidth]{Plots/TBCalibration/Attenprofs_Simulation_WPE_corr_xy_Y_Prof.pdf}
\end{subfigure}
\begin{subfigure}[b]{0.495\textwidth}
\centering
\includegraphics[width=\textwidth]{Plots/TBCalibration/Attenprofs_Simulation_CellPE_X_Prof.pdf}
\end{subfigure}
\begin{subfigure}[b]{0.495\textwidth}
\centering
\includegraphics[width=\textwidth]{Plots/TBCalibration/Attenprofs_Simulation_CellPE_Y_Prof.pdf}
\end{subfigure}
\begin{subfigure}[b]{0.495\textwidth}
\centering
\includegraphics[width=\textwidth]{Plots/TBCalibration/Attenprofs_Simulation_PlanePE_X_Prof.pdf}
\end{subfigure}
\begin{subfigure}[b]{0.495\textwidth}
\centering
\includegraphics[width=\textwidth]{Plots/TBCalibration/Attenprofs_Simulation_PlanePE_Y_Prof.pdf}
\end{subfigure}
\caption[Uncorrected energy response along $w$, cell and plane for simulation]{Uncorrected average energy response as a function of the position within a cell ($w$ - top), cell number (middle), or plane number (bottom) for the Test Beam detector simulation of cosmic muon hits selected for calibration. Left side shows distributions for the X view (vertical) planes and right side for the Y view (horizontal) planes. Each plot is a profile histogram, with uncertainties representing statistical variations. Red lines on the bottom two plots depict the boundaries between different scintillators. Features explained in text.}
\label{fig:Calibhist_simulation}
\end{figure}

The rise of the response with the cell number, visible in the middle plots in Fig.~\ref{fig:Calibhist_simulation}, is due to the varying distance of cells to the readout. Since the \glspl{APD} are located on one side of each module, light from cells on the opposite side has to travel along the \gls{WLS} fibre for an additional module width, compared to the cells closer to the readout. Light undergoes additional attenuation along these so-called `pig tails', causing the difference of the energy response. The additional variations across cells within a module, notably the relatively lower response in cells 0, 1, 9, 10, 23, 24, 31 and 32, is caused by including the relative gain differences into the simulation, as explained above in Sec.~\ref{sec:TBThresholdCorrection}.

The uncorrected energy response as a function of plane number is shown in the bottom row of Fig.~\ref{fig:Calibhist_simulation}, illustrating large fluctuations between planes in both views. We can clearly identify the three distinctly different responses delineated by red lines, corresponding to the three scintillator variations used, as described in Sec.~\ref{sec:TBExperiment}. Additionally, planes 16, 17, 48 and 49 use the \gls{FEB}v5.2 instead of \gls{FEB}v4.1, resulting in a relatively lower response. All these variations between planes in simulation are caused by consolidating the planes and replacing possible discrepancies with the \gls{FB} map (Sec.~\ref{sec:FibreBrightnessTB}), which is used for simulation to emulate real detector conditions. The rest of the variations are caused by differences between readout electronics and individual cells, but are exacerbated by the \gls{FB} binning, which groups otherwise smooth variations across planes into 12 discrete bins, thus amplifying them.

\subsection*{Simulation relative calibration results}

An overview of the attenuation fit results for simulation is shown in Fig.~\ref{fig:CellCentreResponseSim} as a map of average fitted response in the centre of each cell. Blank cells mark the uncalibrated cells which failed the calibration condition (attenuation fit $\chi^2>0.2$). All the uncalibrated cells but one are on the edges of the detector, which is expected, as they have much fewer events that pass the calibration sample selection. There are 43 uncalibrated cells out of the total 4032 cells in the Test Beam detector, resulting in 1.07\% of the simulated detector remaining uncalibrated.

\begin{figure}[h]
\centering
\includegraphics[width=\textwidth]{Plots/TBCalibration/CellResponseAtCentre_Prod4DataBasedSim_Limited_NOvAPlotStyle.pdf}
\caption[Map of fitted response at cell centre for simulation]{Overview of the attenuation fit results for the simulated Test Beam detector. Each cell represents the result of the attenuation fit to the energy response in the centre of that cell. The blank cells are uncalibrated as the attenuation fit did not satisfy the calibration condition.}
\label{fig:CellCentreResponseSim}
\end{figure}

For simulation, the attenuation fit is done for each \gls{FB} bin and each cell separately. Examples of detector response for different cells in various \gls{FB} bins are shown in Fig.~\ref{fig:AttenfitResultsSimulation}. Here the red line shows the initial exponential fit and the blue line depicts the final attenuation fit after the \gls{LOWESS} correction, as described in Sec.~\ref{sec:NOvACalibration}. The cells on the edge of the detector failed the calibration conditions due to the low number of entries causing large fluctuation in the mean response.

There is only one cell in the middle of the detector that is left uncalibrated. This is the cell 32 in a vertical plane in \gls{FB} bin 5, shown on the top right of Fig.~\ref{fig:AttenfitResultsSimulation}, with $\chi^2=0.227$. Apparently, the reason the attenuation fit for this cell failed the calibration condition is the unusually high response with a large uncertainty in the right-most bin. It is unclear why this bin has such an elevated mean response, but since this only causes an issue for a single cell, we decided to ignore it and leave it uncalibrated.

\begin{figure}[h]
  \begin{subfigure}{0.495\textwidth}
    \includegraphics[width=\linewidth]{Plots/RelativeCalibrationResults/sim_fb2_001_050.png}
  \end{subfigure}
  \begin{subfigure}{0.495\textwidth}
    \includegraphics[width=\linewidth]{Plots/RelativeCalibrationResults/sim_fb5_000_032.png}
  \end{subfigure}
  \begin{subfigure}{0.495\textwidth}
    \includegraphics[width=\linewidth]{Plots/RelativeCalibrationResults/sim_fb6_001_000.png}
  \end{subfigure}
  \begin{subfigure}{0.495\textwidth}
    \includegraphics[width=\linewidth]{Plots/RelativeCalibrationResults/sim_fb10_000_063.png}
  \end{subfigure}
  \caption[Example attenuation fits for simulation]{Attenuation fits for a selection of cells in various \acrshort{FB} bins in the calibration of the Test Beam simulation. Top left is an example of a successful attenuation fit, top right is a failed fit due to statistical fluctuation in the last bin and the bottom plots show failed fits for cells on the edges of the detector.}
  \label{fig:AttenfitResultsSimulation}
\end{figure}

\section{Period 2 data}\label{sec:TBPeriod2}
The distribution of cosmic muon tricell hits selected for calibration in Test Beam period 2 data is shown in Fig.~\ref{fig:CalibhistMap_period2}.
The issue with underfilled cells described in Sec.~\ref{sec:TBExperiment} was present throughout period 2. The underfilled cells were marked as bad channels and therefore ignored during production of calibration samples. This also visibly affects the event count in the neighbouring cells to the underfilled cells, which have fewer calibration hits due to the tricell condition (see Sec.~\ref{sec:NOvACalibration}). However, since the underfilled cells 63 are also on the edge of the detector, labelling them as bad channels can't mitigate the effect on the neighbouring cells 62.

\begin{figure}[h]
\centering
\includegraphics[width=\textwidth]{Plots/TBCalibration/Attenprofs_P2Data_CellPlane_AllRuns.pdf}
\caption[Plane-Cell distribution of tricell hits for period 2 data]{Distribution of tricell hits as a function of Test Beam detector cells and planes in the entire period 2 data calibration sample. The rows of empty cells 31 and 62 across all the horizontal planes are caused by the underfilled cells (and tricell condition), as explained in text. There are several areas with relatively fewer hits. Notably cells 38-40 in plane 48 and cells 45-47 in plane 55. Both of these spots comprise of three cells, pointing towards the middle cell being a dead channel (for a limited time) and the two surrounding cells being affected by the tricell condition. Additionally, the bottom half of planes 55 and 57 have noticeably lower number of hits than their top halves (one half corresponds to a single readout).}
\label{fig:CalibhistMap_period2}
\end{figure}

We can also observe areas with relatively fewer hits, likely due to channels that were dead for some time. This also affects their immediate neighbours due to the tricell condition. Additionally, there are planes that have noticeably fewer hits in one half than in the other, and since half of a plane corresponds to a single readout (one \gls{FEB} and \gls{APD}), which means an entire readout was faulty for a certain time.

Officially, period 2 is divided into six epochs labelled 2a - 2f, based on specific Test Beam detector running conditions. Generally, smaller calibration samples reduce time-dependent effects on calibration, such as detector ageing or temperature and humidity variations. However, smaller samples also increase the number of cells with issues in the attenuation fit (examples shown below). Therefore, it is important to choose an optimal calibration sample size to balance both concerns. Since individual epochs in period 2 do not contain enough events for a successful attenuation fit, and variations between epochs are minimal, we decided to calibrate the entire period 2 together, without splitting it into any smaller calibration samples.

The epochs in period 2 mostly differ in the use of various \gls{FEB} firmwares or in the presence of trigger studies. We compare the energy deposition during the individual epochs in Fig.~\ref{fig:Calibhist_period2}. As can be seen, the difference between the energy response across the individual epochs is fairly small (within $\unit[2]{\%}$) and only in normalization, with the largest outliers seemingly epochs 2a and 2d. There is also no clear trend of energy response falling or raising with time (epoch labels are organized in time alphabetically).

\begin{figure}[!hbtp]
\centering
\begin{subfigure}[b]{0.495\textwidth}
\centering
\includegraphics[width=\textwidth]{Plots/TBCalibration/Attenprofs_P2Data_WPE_corr_xy_X_Combined.pdf}
\end{subfigure}
\begin{subfigure}[b]{0.495\textwidth}
\centering
\includegraphics[width=\textwidth]{Plots/TBCalibration/Attenprofs_P2Data_WPE_corr_xy_Y_Combined.pdf}
\end{subfigure}
\begin{subfigure}[b]{0.495\textwidth}
\centering
\includegraphics[width=\textwidth]{Plots/TBCalibration/Attenprofs_P2Data_CellPE_X_Combined.pdf}
\end{subfigure}
\begin{subfigure}[b]{0.495\textwidth}
\centering
\includegraphics[width=\textwidth]{Plots/TBCalibration/Attenprofs_P2Data_CellPE_Y_Combined.pdf}
\end{subfigure}
\begin{subfigure}[b]{0.495\textwidth}
\centering
\includegraphics[width=\textwidth]{Plots/TBCalibration/Attenprofs_P2Data_PlanePE_X_Combined.pdf}
\end{subfigure}
\begin{subfigure}[b]{0.495\textwidth}
\centering
\includegraphics[width=\textwidth]{Plots/TBCalibration/Attenprofs_P2Data_PlanePE_Y_Combined.pdf}
\end{subfigure}
\caption[Uncorrected energy response along $w$, cell and plane for period 2 data]{Uncorrected average energy response as a function of the position within a cell ($w$ - top), cell number (middle), or plane number (bottom) for various epochs in the Test Beam detector period 2 data of cosmic muons hits selected for calibration. Left side shows distributions for the X view (vertical) planes and right side for the Y view (horizontal) planes. Each plot is a profile histogram, with uncertainties representing statistical variations. It is clear that there is no significant difference in shape between the various epochs. The one  exception is plane 55, which has a visibly higher energy response than the rest of the planes, especially in epoch 2a, as can be seen in the bottom right plot.}
\label{fig:Calibhist_period2}
\end{figure}

The only noticeable variation of energy response across epochs in both normalization and shape can be seen on the distributions of the energy response as a function of planes, where the uncorrected response in plane 55 is noticeably higher than the rest of the period. The exact reason for this is unknown, although it is likely caused by a fault in one of the two \glspl{FEB} that make up the plane readout.

\subsection*{Period 2 relative calibration results}

The results of the attenuation fit for period 2 are summarised in Fig.~\ref{fig:CellCentreResponsePeriod2}, showing the map of the fitted response at the centre of each cell, with blank bins representing cells that failed the calibration condition and are left uncalibrated. Summary of the relative calibration results is shown in Tab.~\ref{tab:TestBeamPeriod2RelCalibResults}. There are 199 cells that failed the calibration condition out of the total 4032 cells, constituting 4.94\% of the detector left uncalibrated for period 2. The largest contribution to the uncalibrated cells are the peripheral cells on the edge of the detector, which contain too few events due to the tricell condition.

\begin{figure}[!hbtp]
\centering
\includegraphics[width=\textwidth]{Plots/TBCalibration/CellResponseAtCentre_period2_Limited_NOvAPlotStyle.pdf}
\caption[Map of fitted response at cell centre for period 2 data]{Overview of the attenuation fit results for the Test Beam detector period 2 data. Each cell represents the result of the attenuation fit to the energy response in the centre of that cell, with blank cells failing the calibration condition $\left(\chi^2>0.2\right)$. Cells 0 and 63, which are on the edges of the detector are mostly uncalibrated due to low statistics of calibration hits. Cell 31 and 63 in horizontal planes are underfilled, showing as rows of blank cells across the detector. This affects some of their neighbouring cells, such as cells 30 and 32 in plane 1, or cells 62 in all of the horizontal planes. Cells 0-31 for planes 55 and 57 have a visibly higher (plane 55) and lower (plane 57) energy response, caused by faulty \glspl{FEB}, which for some time wrongly recorded scaled response. Cells 2-4 and 45-47 in plane 55 were dead for some time during period 2, resulting in failing the calibration condition. There are a few other uncalibrated cells, which are concentrated at the end of the detector (right hand side), which failed the calibration condition due to large fluctuations at cell edges.}
\label{fig:CellCentreResponsePeriod2}
\end{figure}

\begin{table}[!hbtp]
\centering
\caption[Summary of relative calibration results for period 2]{Summary of relative calibration results for period 2 with the uncalibrated cells divided into four categories based on the main reason of failure, all described in text.}
\def\arraystretch{1.4}
\begin{tabular}{|cl|c|c|}
\hline
\multicolumn{2}{|c|}{\textbf{Calibration status}} & \textbf{Number of cells} & \textbf{Detector proportion}\\\hline
\multicolumn{2}{|c|}{Calibrated} & $3833$ & $\unit[95.06]{\%}$\\\hline
\parbox[t]{2mm}{\multirow{4}{*}{\rotatebox[origin=c]{90}{Uncalibrated }}} & Peripheral cells & $121$ & $\unit[3.00]{\%}$\\
 & Underfilled cells & $64$ & $\unit[1.59]{\%}$\\
 & Readout & $9$ & $\unit[0.22]{\%}$\\
 & Binning & $5$ & $\unit[0.12]{\%}$\\\hline
\end{tabular}
\label{tab:TestBeamPeriod2RelCalibResults}
\end{table}

%Classic response and expected effects
Most cells have the standard response, as discussed for simulation. However, some cells have one or more regions with a drop in the energy response, as shown in Fig.~\ref{fig:AttenfitResultsPeriod2_ZippedFibers}. These low regions are a real physical effect caused by zipped, or possibly even twisted, \gls{WLS} fibres \cite{NOvA-doc-43249}. This effect is present in all the \gls{NOvA} detectors. As can be seen, the attenuation fit is capable of fitting this irregular response and therefore the relative calibration corrects for this effect in data. However, zipped fibres are not included in simulation for any of the detectors, which could potentially cause discrepancy from data due to the \gls{ADC} threshold. It was decided that this does not have a significant impact and it would not be worth the amount of work required to include all the zipped fibres into the simulation.

\begin{figure}[!hbtp]
  \begin{subfigure}{0.495\textwidth}
    \includegraphics[width=\linewidth]{Plots/RelativeCalibrationResults/p2_008_028.png}
  \end{subfigure}
  \begin{subfigure}{0.495\textwidth}
    \includegraphics[width=\linewidth]{Plots/RelativeCalibrationResults/p2_022_035.png}
  \end{subfigure}
  \caption[Attenuation fits for standard cells in period 2 data]{Attenuation fits for a selection of cells in period 2. Left plot shows an example of the standard energy deposition in the Test Beam and right plot shows the effect of zipped fibres.}
  \label{fig:AttenfitResultsPeriod2_ZippedFibers}
\end{figure}

The attenuation fits for the underfilled cells fail the calibration condition as expected. On the other hand, most of their neighbouring cells in the middle of the detector (cells 30 and 32) successfully pass the calibration condition despite having fewer events. This is thanks to the decision to label the underfilled cells as bad channels, as shown in Fig.~\ref{fig:AttenfitResultsPeriod2_UnderfilledCells}. However, it appears that some cells neighbouring the underfilled cells near the edge of the detector have too few events to have satisfactory attenuation fits.

\begin{figure}[!hbtp]
  \begin{subfigure}{0.495\textwidth}
    \includegraphics[width=\linewidth]{Plots/RelativeCalibrationResults/p2_003_030.png}
  \end{subfigure}
  \begin{subfigure}{0.495\textwidth}
    \includegraphics[width=\linewidth]{Plots/RelativeCalibrationResults/p2_011_032.png}
  \end{subfigure}
  \begin{subfigure}{0.495\textwidth}
    \includegraphics[width=\linewidth]{Plots/RelativeCalibrationResults/p2_001_030.png}
  \end{subfigure}
  \begin{subfigure}{0.495\textwidth}
    \includegraphics[width=\linewidth]{Plots/RelativeCalibrationResults/p2_001_032.png}
  \end{subfigure} 
  \caption[Attenuation fits for underfilled cells in period 2 data]{Fit to the energy response in period 2. Showing examples of cells neighbouring the underfilled cells which have fewer events and therefore larger fluctuations than the `usual' Test Beam cell. Bottom two plots show examples of neighbouring cells to the underfilled cells, specifically in plane 1, which failed the calibration condition due to low statistics. This is a result of the combined effect of being a neighbour to the underfilled cell and on the edge of the detector.}
  \label{fig:AttenfitResultsPeriod2_UnderfilledCells}
\end{figure}

%Since the underfilled cells were marked as bad channels, we didn't attempt to calibrate them. Their neighbours have fewer events due to the tricell condition, but majority of them pass the calibration condition, as shown in Fig.~\ref{fig:AttenfitResultsPerio2_UnderfilledCells}. The decision to mark the underfilled cells as bad channel was motivated by the fact that bad channels get skipped by the tricell condition and the neighbouring cells to the underfilled cells can therefore be included in calibration. The fact that majority of the neighbouring cells to the underfilled cells do get calibrated clearly proves that this was a good decision.

%The neighbouring cells in plane 1 don't pass the calibration condition due to low statistics and therefore large fluctuations, as shown in . This is likely due to a combination of the tricell condition and plane 1 being on the edge of the detector, which typically has fewer (accepted) hits than the center, as shown in Fig.~\ref{fig:Calibhist_period2}.

%Faulty readout in general
The effects  of the issues with dead channels and with faulty readout electronics occurring during period 2, which were discussed above, can be clearly seen on the map of the attenuation fit results in Fig.~\ref{fig:CellCentreResponsePeriod2} and on the attenuation fits themselves in Fig.~\ref{fig:AttenfitResultsPeriod2_ReadoutIssues}. The (temporarily) dead channel in plane 55 contains too few events to pass the calibration condition. However, the channel in plane 48 was likely dead for a shorter duration, resulting in a successful attenuation fit, despite the lower number of hits compared to a standard cell. Cells corresponding to the entire readout affected have lower number of hits, resulting in some of them having attenuation fits failing the calibration condition.  Furthermore, these cells have a strikingly different energy response, even $3\times$ larger than the average in the case of plane 55. This is due to the corresponding \glspl{APD} or \glspl{FEB} incorrectly recording a scaled-up or scaled-down energy response than the real energy deposited in the detector. The cause for this scaled recorded response is not known. Since this effect is present for all data, not only for the cosmic muons used for calibration, it is important to correctly account for it in calibration. However, there is a reason for concern, as this issue can arise even if these \glspl{FEB} (or possibly \glspl{APD}) were only affected for a limited time out of the entire calibrated period. Since we are performing the attenuation fits on the average response across the entire calibrated period, if an \gls{FEB} records a standard response for half of the time and $7\times$ larger response for the seconds half, calibration is going to assume the response was $4\times$ larger the entire time, which would be incorrect. However, since both of the affected planes are in the back of the detector, we decided to ignore this effect for period 2.

\begin{figure}[!hbtp]
  \begin{subfigure}{0.495\textwidth}
    \includegraphics[width=\linewidth]{Plots/RelativeCalibrationResults/p2_055_046.png}
  \end{subfigure}
  \begin{subfigure}{0.495\textwidth}
    \includegraphics[width=\linewidth]{Plots/RelativeCalibrationResults/p2_055_045.png}
  \end{subfigure}
  \begin{subfigure}{0.495\textwidth}
    \includegraphics[width=\linewidth]{Plots/RelativeCalibrationResults/p2_055_011.png}
  \end{subfigure}
  \begin{subfigure}{0.495\textwidth}
    \includegraphics[width=\linewidth]{Plots/RelativeCalibrationResults/p2_057_011.png}
  \end{subfigure}
  \caption[Attenuation fits for cells with readout issues in period 2 data]{Fit to the energy response in period 2. Some channels were likely dead for some time, resulting with significantly less recorded events as shown on the top left plot. This also affect their neighbouring cells due to the tricell condition as shown on the top right. Planes 55 and 57, shown on the bottom left and bottom right plots respectively, correspond to one of the `faulty' \glspl{FEB} affected for some time. This results in a significantly different scale of energy response, which is much higher than the rest of the detector for plane 55, and smaller for plane 57.}
  \label{fig:AttenfitResultsPeriod2_ReadoutIssues}
\end{figure}

%Unexpected issue - Binning for the attenuation fits
An unexpected issue appeared for several cells located near the end of the Test Beam detector (relative to the beam). These cells have attenuation fits failing the calibration condition due to the unusually high response or a lack of events in histogram bins at the edges of the cell, as shown in Fig.~\ref{fig:AttenfitResultsPeriod2_CellEdge}. This is a combination of a real physical effect - caused by fewer hits at the edge of the detector, possibly also due to the fibre loops and fibre ends - and of the choice of binning for the attenuation profiles. All attenuation profiles for all the \gls{NOvA} detectors are created with 100 bins, extending beyond the physical dimensions of the detector. For example, in the Test Beam detector, the attenuation profiles range from $\unit[-150]{cm}$ to $\unit[150]{cm}$, while the actual half-length of a Test Beam cell is $\unit[131.07]{cm}$. This means that the attenuation profile bins near the physical edges of the cell contain fewer hits from inside of the detector, resulting in larger fluctuations. Since the attenuation fits are limited to the physical cell boundaries, these bins with larger variations can skew their results. This effect can be addressed either by changing the binning of the attenuation profiles to better match the physical dimension of the cell, by loosening the calibration condition for hits on the edges of the cell, or using larger samples for the attenuation fits to reduce variations. However, since the affected uncalibrated cells are in the end of the detector, we decided to ignore them.
%All the attenuation profiles are created with 100 bins, ranging in (-150,150) for TB, (-250,250) for ND and (-800,800) for FD. The range of the fit is the detHalfHight/Width from geometry, which is 131.07 for TB. So the bin on the edge of fit range for TB is (-132,-129).

\begin{figure}[!hbtp]
  \begin{subfigure}{0.495\textwidth}
    \includegraphics[width=\linewidth]{Plots/RelativeCalibrationResults/p2_054_050.png}
  \end{subfigure}
  \begin{subfigure}{0.495\textwidth}
    \includegraphics[width=\linewidth]{Plots/RelativeCalibrationResults/p2_058_024.png}
  \end{subfigure}
  \begin{subfigure}{0.495\textwidth}
    \includegraphics[width=\linewidth]{Plots/RelativeCalibrationResults/p2_058_025.png}
  \end{subfigure}
  \begin{subfigure}{0.495\textwidth}
    \includegraphics[width=\linewidth]{Plots/RelativeCalibrationResults/p2_060_032.png}
  \end{subfigure}
  \caption[Attenuation fits for cells with large fluctuations in period 2 data]{Fit to the energy response in period 2. Examples of cells that have an unusually high or low energy response at the edge of the cell, skewing the attenuation fits and resulting in them getting labelled as not calibrated. Cells shown on the two top plots and on the bottom left plot have a single bin on the edge of the fitted region (marked by dotted vertical lines) with noticeably higher average energy response. These anomalous bins typically only have a single entry that skews that attenuation fits and their $\chi^2$ calculations. Cell 32 in plane 60, shown in the bottom right plot, has bins on the edge of the cell with no entries, resulting in the same effect as the other cells mentioned above.}
  \label{fig:AttenfitResultsPeriod2_CellEdge}
\end{figure}

\section{Period 3 data}\label{sec:TBCalibration_period3}
The underfilled cells were refilled (or overfilled) during the period 3 data taking. This was the main motivation for dividing period 3 into individual epochs as shown in Tab.~\ref{tab:TestBeamPeriod3Epochs}. Another major event that could impact calibration is the replacement of several faulty \glspl{FEB}, which motivated the creation of epoch 3e.

\begin{table}[!hbtp]
\centering
\caption[Description of Test Beam period 3 epochs]{Test Beam period 3 epochs, their start dates and the reason for their separation.}
\def\arraystretch{1.4}
\begin{tabular}{m{0.11\textwidth} m{0.22\textwidth} m{0.55\textwidth}}
Name & Start date & Reason for creating the epoch\\\hline
Epoch 3a & January $12^{\textsf{th}}$ 2021 & Underfilled cells\\
Epoch 3b & April $21^{\textsf{st}}$ 2021 & Overfilling the back 9 horizontal planes and the 7th horizontal plane from the front\\
Epoch 3c & April $27^{\textsf{th}}$ 2021 & Overfilling of the 15 front horizontal planes (except the 7th, which was already done) and the 14th horizontal plane\\
Epoch 3d & April $30^{\textsf{th}}$ 2021 & Overfilling of the remaining 8 horizontal planes\\
Epoch 3e & May $12^{\textsf{th}}$ 2021 & FEB swaps
\end{tabular}
\label{tab:TestBeamPeriod3Epochs}
\end{table}

The refilling of the underfilled cells can be clearly seen on the cell and plane distribution of hits in Fig.~\ref{fig:CalibhistMap_period3} and on the distribution of energy deposition across horizontal (Y view) cells in Fig.~\ref{fig:Calibhist_period3}. The distributions of hits also shows a few channels that were dead for a certain time.
Additionally, the energy deposition distributions show, that one of the \glspl{FEB} was recording a scaled up/down energy  response, similarly to the faulty \glspl{FEB} in period 2. However, a can be seen in the distribution of hits, this particular faulty \gls{FEB} recorded the same number of events as were recorded in the surrounding modules. This is one of the \gls{FEB} that got replaced between epochs 3d and 3e and, as will be shown below, this is the \gls{FEB} with the largest impact on the calibration out of the faulty \glspl{FEB} replaced before the start of epoch 3e.

\begin{figure}[!hbtp]
\centering
\begin{subfigure}[b]{\textwidth}
\centering
\includegraphics[width=\textwidth]{Plots/TBCalibration/Attenprofs_P3Data_CellPlane_Epoch3a.pdf}
\end{subfigure}
\begin{subfigure}[b]{\textwidth}
\centering
\includegraphics[width=\textwidth]{Plots/TBCalibration/Attenprofs_P3Data_CellPlane_Epoch3de.pdf}
\end{subfigure}
\caption[Plane-Cell distribution of hits for the period 3 data sample]{Distribution of events in the period 3 Test Beam data calibration sample. Comparison of the epoch 3a data before the refilling of the underfilled cells 31 and 63, clearly visible by a row of empty bins,  and the combination of epochs 3d and 3e after the full refilling. There are also several cells that experienced readout issues, specifically cell 39 in plane 48 and cell 31 in plane 18.}
\label{fig:CalibhistMap_period3}
\end{figure}

\begin{figure}[!hbtp]
\centering
\begin{subfigure}[b]{0.495\textwidth}
\centering
\includegraphics[width=\textwidth]{Plots/TBCalibration/Attenprofs_P3Data_WPE_corr_xy_X_Combined.pdf}
\end{subfigure}
\begin{subfigure}[b]{0.495\textwidth}
\centering
\includegraphics[width=\textwidth]{Plots/TBCalibration/Attenprofs_P3Data_WPE_corr_xy_Y_Combined.pdf}
\end{subfigure}
\begin{subfigure}[b]{0.495\textwidth}
\centering
\includegraphics[width=\textwidth]{Plots/TBCalibration/Attenprofs_P3Data_CellPE_X_Combined.pdf}
\end{subfigure}
\begin{subfigure}[b]{0.495\textwidth}
\centering
\includegraphics[width=\textwidth]{Plots/TBCalibration/Attenprofs_P3Data_CellPE_Y_Combined.pdf}
\end{subfigure}
\begin{subfigure}[b]{0.495\textwidth}
\centering
\includegraphics[width=\textwidth]{Plots/TBCalibration/Attenprofs_P3Data_PlanePE_X_Combined.pdf}
\end{subfigure}
\begin{subfigure}[b]{0.495\textwidth}
\centering
\includegraphics[width=\textwidth]{Plots/TBCalibration/Attenprofs_P3Data_PlanePE_Y_Combined.pdf}
\end{subfigure}
\caption[Uncorrected energy response along $w$, cell and plane for period 3]{Uncorrected average energy response as a function of the position within a cell ($w$ - top), cell number (middle), or plane number (bottom) for various epochs in the Test Beam detector period 3 data of cosmic muons hits selected for calibration. Left side shows distributions for the X view (vertical) planes and right side for the Y view (horizontal) planes. Each plot is a profile histogram, with uncertainties representing statistical variations. The effect of staged refilling of the underfilled cells between the epochs can be seen in the middle right plot, where epoch 3a (orange) has all no underfilled cells refilled, and epochs 3d and 3e (green) have all the cells filled to the top. Comparing the distributions of energy deposition in X view between the cell and plane plots, it can be seen that the top \acrshort{FEB}/\acrshort{APD} in plane 58, which correspond cells 32-63, was faulty throughout period 3. Specifically, that the energy response in this module was larger in epoch 3a, then got lower in epochs 3b and 3c, until getting significantly lower for epochs 3d and 3e.}
\label{fig:Calibhist_period3}
\end{figure}

From the aforementioned considerations, we decided to calibrate epochs 3a, 3b and 3c together, which are all the epochs containing any underfilled cells, and to separately calibrate epochs 3d and 3e together. The faulty \gls{FEB} in the top of plane 58 is far enough in the back of the detector, that we didn't find it necessary to calibrate epochs 3d and 3e separately. Additionally, epochs 3b and 3c contain only few days worth of data, therefore they wouldn't have enough events for successful independent attenuation fits.

\subsection*{Combined epochs 3a, 3b and 3c relative calibration results}

The results of attenuation fits for the combined epochs 3a, 3b and 3c are summarised in Fig.~\ref{fig:CellCentreResponseEp3abc}, showing the map of the fitted response at the centre of each cell. There are 182 uncalibrated cells out of 4032, constituting 4.51\% of the detector, as shown in Tab.~\ref{tab:TestBeamEp3abcRelCalibResults}.

\begin{figure}[!hbtp]
\centering
\includegraphics[width=\textwidth]{Plots/TBCalibration/CellResponseAtCentre_epoch3abc_Limited_NOvAPlotStyle.pdf}
\caption[Map of fitted response at cell centre for epochs 3a, 3b and 3c data]{Overview of the relative calibration results for the Test Beam detector period 3, combined epochs 3a, 3b and 3c data. Each cell represents the result of the attenuation fit to the energy response in the centre of that cell. The blank bins represent uncalibrated cells. The rows of uncalibrated cells 31 and 62 are caused by the underfilled cells together with the tricell condition. The same effect affects cell 32 in plane 1. The two dark-red stripes correspond to two faulty \glspl{FEB} in planes 36 and 58. There are five additional uncalibrated cells, specifically cell 2 in plane 58, cells 21 and 32 in plane 60, and cells 31 and 38 in plane 63, which are uncalibrated due to large fluctuations at cell edges.}
\label{fig:CellCentreResponseEp3abc}
\end{figure}

\begin{table}[!hbtp]
\centering
\caption[Summary of relative calibration results for the combined epochs 3a, 3b and 3c]{Summary of relative calibration results for the combined epochs 3a, 3b and 3c with the uncalibrated cells divided into four categories based on the main reason of failure, all described in text.}
\def\arraystretch{1.4}
\begin{tabular}{|cl|c|c|}
\hline
\multicolumn{2}{|c|}{\textbf{Calibration status}} & \textbf{Number of cells} & \textbf{Detector proportion}\\\hline
\multicolumn{2}{|c|}{Calibrated} & $3850$ & $\unit[95.49]{\%}$\\\hline
\parbox[t]{2mm}{\multirow{4}{*}{\rotatebox[origin=c]{90}{Uncalibrated }}} & Peripheral cells & $128$ & $\unit[3.17]{\%}$\\
 & Underfilled cells & $49$ & $\unit[1.22]{\%}$\\
 & Readout & $0$ & $\unit[0.00]{\%}$\\
 & Binning & $5$ & $\unit[0.12]{\%}$\\\hline
\end{tabular}
\label{tab:TestBeamEp3abcRelCalibResults}
\end{table}

We can see that some of the underfilled cells that have been refilled for epochs 3b or 3c, but were underfilled for epoch 3a, which makes up the majority of this calibrated data, are now calibrated thanks to including these two short epochs into the same attenuation fit. Example of energy deposition in such a cell is shown on the left side of Fig.~\ref{fig:AttenfitResultsEpoch3abc_UnderfilledCellsNeighbours}. Same as in period 2, most of the neighbouring cells to the underfilled cells are calibrated, except for cells on the edge of the detector due to lower statistics.

\begin{figure}[h]
  \begin{subfigure}{0.495\textwidth}
    \includegraphics[width=\linewidth]{Plots/RelativeCalibrationResults/ep3abc_005_031.png}
  \end{subfigure}
  \begin{subfigure}{0.495\textwidth}
    \includegraphics[width=\linewidth]{Plots/RelativeCalibrationResults/ep3abc_001_032.png}
  \end{subfigure}
  \caption[Attenuation fits for re-filled cells in period 3 data]{Fit to the energy response in epochs 3a, 3b and 3c. Some underfilled cells that have been refilled in epochs 3b and 3c are now calibrated as shown on the left plot. Cell 32 in plane 1 is the only neighbouring cell to the underfilled cell that didn't manage to get calibrated due to low number of events.}
  \label{fig:AttenfitResultsEpoch3abc_UnderfilledCellsNeighbours}
\end{figure}

There is a couple of noticeably faulty \glspl{FEB} with a scaled energy response, shown in Fig.~\ref{fig:AttenfitResultsEpoch3abc_FaultyFEBs}. Besides the expected \gls{FEB} in plane 58, which has about $5\times$ larger response, there is also the \gls{FEB} in plane 36, which has about $2.5\times$ larger response compared to the average. This could mean that the \gls{FEB} in plane 36 was faulty only for a limited time compared to the \gls{FEB} in plane 58. This is a reason for concern, as the relative calibration correction for hits in this module, during the time when the \gls{FEB} wasn't faulty, would be too large (and therefore the `corrected response' would be too small). On the other hand, during the time when the \gls{FEB} was faulty, the correction would be too small and hence the corrected response would be too large. Given that plane 36 is in the middle of the detector, there is a chance this might noticeably affect some Test Beam analysis results. Therefore, it is possible this issues might have to be mitigated in the future, whether with an additional uncertainty, or by improving the calibration. It is currently difficult to address issues such as this in the \gls{NOvA} calibration. However, there is currently an effort underway to split the inputs for calibration by cells, rather than by time, which would make solving these issues much simpler. For the time being, we decided to ignore these faulty \glspl{FEB}.

\begin{figure}[h]
  \begin{subfigure}{0.495\textwidth}
    \includegraphics[width=\linewidth]{Plots/RelativeCalibrationResults/ep3abc_036_054.png}
  \end{subfigure}
  \begin{subfigure}{0.495\textwidth}
    \includegraphics[width=\linewidth]{Plots/RelativeCalibrationResults/ep3abc_058_048.png}
  \end{subfigure}
  \caption[Attenuation fits for cells with faulty readout in period 3 data]{Fit to the energy response in epochs 3a, 3b and 3c. The most obvious faulty FEBs that have a significantly larger energy response than their neighbours.}
  \label{fig:AttenfitResultsEpoch3abc_FaultyFEBs}
\end{figure}

Similarly to period 2, there are a few cells in the back of the detector that have a sharp rise in energy response at their edge, which causes their attenuation fit to fail the calibration condition. This can be seen in Fig.~\ref{fig:AttenfitResultsEpoch3abc_CellEdges}, where the significantly different mean responses at the edge bins is pulling the attenuation fit to incorrect values. Given this is concentrated in cells in the end of the detector, we decided to ignore this effect and leave these cells uncalibrated.

\begin{figure}[h]
  \begin{subfigure}{0.495\textwidth}
    \includegraphics[width=\linewidth]{Plots/RelativeCalibrationResults/ep3abc_058_002.png}
  \end{subfigure}
  \begin{subfigure}{0.495\textwidth}
    \includegraphics[width=\linewidth]{Plots/RelativeCalibrationResults/ep3abc_060_032.png}
  \end{subfigure}
  \caption[Attenuation fits for cells with large fluctuations in period 3 data]{Fit to the energy response in epochs 3a, 3b and 3c. Some cells are not calibrated due to large fluctuations at one edge of the cells.}
  \label{fig:AttenfitResultsEpoch3abc_CellEdges}
\end{figure}

\subsection*{Combined epochs 3d and 3e relative calibration results}

The attenuation fits results for epochs 3d and 3e are shown in Fig.~\ref{fig:CellCentreResponseEp3de}. There are 182 uncalibrated cells out of 4032 total cells, making up 4.51\% of the detector. The uncalibrated cells are now however almost entirely concentrated at the edges of the detector. Summary of the relative calibration results is shown in Tab.~\ref{tab:TestBeamEp3deRelCalibResults}.

\begin{figure}[!hbtp]
\centering
\includegraphics[width=\textwidth]{Plots/TBCalibration/CellResponseAtCentre_epoch3de_original_Limited_NOvAPlotStyle.pdf}
\caption[Map of fitted response at cell centre for epochs 3d and 3e data]{Overview of the relative calibration results for the Test Beam detector period 3, combined epochs 3d and 3e data. Each cell represents the result of the attenuation fit to the energy response in the centre of that cell. The blank cells are uncalibrated. The uncalibrated cells 30-32 in plane 17 and cells 5-7 in plane 63 are caused by a dead channel coupled with the effect of the tricell condition. The 8 previously underfilled cells 31 in planes 33, 35, 37, 41, 47, 49, 51 and 59 are uncalibrated due to the difference in the scintillator used for refilling, as described in text. There are 11 cells that are uncalibrated due to low number of events combined with the attenuation profile binning.}
\label{fig:CellCentreResponseEp3de}
\end{figure}

\begin{table}[!hbtp]
\centering
\caption[Summary of relative calibration results for the combined epochs 3d and 3e]{Summary of relative calibration results for the combined epochs 3d and 3e with the uncalibrated cells divided into four categories based on the main reason of failure, all described in text. Brackets show the number of cells that were originally calibrated (or uncalibrated, depending on the row) before the manual alteration of their $\chi^2$ values, as described in text. Proportions are calculated from the final cell counts.}
\def\arraystretch{1.4}
\begin{tabular}{|cl|c|c|}
\hline
\multicolumn{2}{|c|}{\textbf{Calibration status}} & \textbf{Number of cells} & \textbf{Detector proportion}\\\hline
\multicolumn{2}{|c|}{Calibrated} & $3858\ (3850)$ & $\unit[95.68]{\%}$\\\hline
\parbox[t]{2mm}{\multirow{4}{*}{\rotatebox[origin=c]{90}{Uncalibrated }}} & Peripheral cells & $126$ & $\unit[3.13]{\%}$\\
 & Underfilled cells & $31\ (39)$ & $\unit[0.77]{\%}$\\
 & Readout & $6$ & $\unit[0.15]{\%}$\\
 & Binning & $11$ & $\unit[0.27]{\%}$\\\hline
\end{tabular}
\label{tab:TestBeamEp3deRelCalibResults}
\end{table}

The expected effect of one of the two dead channels is shown in Fig.~\ref{fig:AttenfitResultsEpoch3de_LeftoverUnderfilledCell} together with some of the cells in the back of the detector, which have a rise or drop in energy deposition at their edge. This is similar to the effects seen in period 2 and epochs 3a+3b+3c and since it's again concentrated in the end of the detector, we ignore these cells and leave them uncalibrated.

\begin{figure}[h]
  \begin{subfigure}{0.495\textwidth}
    \includegraphics[width=\linewidth]{Plots/RelativeCalibrationResults/ep3de_017_031.png}
  \end{subfigure}
  \begin{subfigure}{0.495\textwidth}
    \includegraphics[width=\linewidth]{Plots/RelativeCalibrationResults/ep3de_017_032.png}
  \end{subfigure}
  \begin{subfigure}{0.495\textwidth}
    \includegraphics[width=\linewidth]{Plots/RelativeCalibrationResults/ep3de_050_018.png}
  \end{subfigure}
  \begin{subfigure}{0.495\textwidth}
    \includegraphics[width=\linewidth]{Plots/RelativeCalibrationResults/ep3de_062_006.png}
  \end{subfigure}  
  \caption[Attenuation fits for dead channels in period 3 data]{Fit to the energy response in epochs 3d and 3e. Top plots show the dead channel (left) and its immediate neighbour (right) affected by the tricell condition. Bottom plots show examples of cells with large fluctuations on their edges likely caused by low number of events combined with binning of attenuation profiles.}
  \label{fig:AttenfitResultsEpoch3de_LeftoverUnderfilledCell}
\end{figure}

Epochs 3d and 3e should have all the previously underfilled cells now refilled, but as can be seen in Fig. \ref{fig:CellCentreResponseEp3de}, there are several of these previously underfilled cells that are still uncalibrated. The energy deposition in these cells is shown in Fig.~\ref{fig:AttenfitResultsEpoch3de_RefilledDiscrepancy}. Here we can see that these cells have a fairly large discrepancy between the left and right sides of the cell. This is caused by using different scintillator oils for the initial filling and for the refilling (or overfilling). Specifically, as was described in Sec.~\ref{sec:TBExperiment}, these cells have been initially filled with the Ash River oil, or with the Texas oils, depending on the cell, which have a higher energy response compared to the \gls{NDOS} oil that was used for their overfilling. These scintillator oils clearly did not mix properly, which caused a discrepancy in the energy deposition in different parts of the cells.
\begin{figure}[h]
  \begin{subfigure}{0.495\textwidth}
    \includegraphics[width=\linewidth]{Plots/RelativeCalibrationResults/ep3de_033_031.png}
  \end{subfigure}
  \begin{subfigure}{0.495\textwidth}
    \includegraphics[width=\linewidth]{Plots/RelativeCalibrationResults/ep3de_059_031.png}
  \end{subfigure}
  \caption[Attenuation fits for cells with mixed scintillators in period 3 data]{Fit to the energy response in epochs 3d and 3e. The scintillator oil used for refilling of the underfilled cells has lower energy response than the oil used for the initial filling. These oils didn't mix properly causing a different energy response in the left and right side of the cell.}
  \label{fig:AttenfitResultsEpoch3de_RefilledDiscrepancy}
\end{figure}
This is a physical effect that should be accounted for in calibration, and, as we can see, the attenuation fits are actually performing reasonably well. Additionally, these cells are in the middle of the detector and leaving them uncalibrated would almost certainly have an impact on Test Beam analyses. The large $\chi^2$ value of the attenuation fit is most likely caused only by the unusual shape of the distribution, which the fit is not designed for. Therefore, we decided to manually change the $\chi^2$ values for these cells inside the csv tables (which hold the results of the attenuation fits), so that their $\chi^2<0.2$ and these cells are officially considered calibrated when applying the calibration results, even if they originally weren't. The map of the `corrected' distribution of the attenuation fit results for epochs 3d and 3e is shown in Fig.~\ref{fig:CellCentreResponseEp3de_updated}.

\begin{figure}[!hbtp]
\centering
\includegraphics[width=\textwidth]{Plots/TBCalibration/CellResponseAtCentre_epoch3de_Limited_NOvAPlotStyle.pdf}
\caption[Corrected map of fitted response at cell centre for epochs 3d and 3e data]{Overview of the final relative calibration results for the combined epochs 3d and 3e data after manually labelling the originally uncalibrated refilled cells as calibrated. Each cell represents the result of the attenuation fit to the energy response in the centre of that cell. The blank cells are uncalibrated and described in text.}
\label{fig:CellCentreResponseEp3de_updated}
\end{figure}

\section{Period 4 data}\label{sec:TBPeriod4}

The data collected during period 4 of the Test Beam run represent our best dataset, with nearly ideal detector conditions. There were a few commissioning runs in the very beginning of period 4, which uncovered some dead channels or faulty \glspl{FEB} that were immediately fixed. These initial runs constitute epoch 4a, shown on the top of Fig.~\ref{fig:CalibhistMap_period4}. Additionally, a few runs included studies where parts of the detector were masked to address \gls{FEB} saturation issues \cite{NOvA-doc-53658}, clearly visible in the middle of Fig.~\ref{fig:CalibhistMap_period4}. The bottom part of Fig.~\ref{fig:CalibhistMap_period4} shows the remainder of period 4 data, which do not have any noticeable faults in their hit distribution across the detector.

\begin{figure}[!hbtp]
\centering
\begin{subfigure}[b]{\textwidth}
\centering
\includegraphics[width=\textwidth]{Plots/TBCalibration/Attenprofs_P4Data_CellPlane_Epoch4a.pdf}
\end{subfigure}
\begin{subfigure}[b]{\textwidth}
\centering
\includegraphics[width=\textwidth]{Plots/TBCalibration/Attenprofs_P4Data_CellPlane_CellMasking.pdf}
\end{subfigure}
\begin{subfigure}[b]{\textwidth}
\centering
\includegraphics[width=\textwidth]{Plots/TBCalibration/Attenprofs_P4Data_CellPlane_GoodRuns.pdf}
\end{subfigure}
\caption[Plane-Cell distribution of hits for the period 4 data sample]{Distribution of events in the Test Beam period 4 data calibration sample. The top plot shows the first three commissioning runs with readout issues, the middle plot shows the status of the detector during the cell masking studies and the bottom plot shows the rest of the runs. Only the runs from the bottom plot (marked GoodRuns) are used for calibration.}
\label{fig:CalibhistMap_period4}
\end{figure}

Figure~\ref{fig:Calibhist_period4} shows, that the epoch 4a and the cell masking study had noticeable impacts on the energy deposition across the detector. Both of these special periods only span a short time and therefore contain very limited number of hits. We decided to ignore these runs and only calibrate the rest of period 4 data, using their results for all runs in period 4.

\begin{figure}[!hbtp]
\centering
\begin{subfigure}[b]{0.495\textwidth}
\centering
\includegraphics[width=\textwidth]{Plots/TBCalibration/Attenprofs_P4Data_WPE_corr_xy_X_Combined.pdf}
\end{subfigure}
\begin{subfigure}[b]{0.495\textwidth}
\centering
\includegraphics[width=\textwidth]{Plots/TBCalibration/Attenprofs_P4Data_WPE_corr_xy_Y_Combined.pdf}
\end{subfigure}
\begin{subfigure}[b]{0.495\textwidth}
\centering
\includegraphics[width=\textwidth]{Plots/TBCalibration/Attenprofs_P4Data_CellPE_X_Combined.pdf}
\end{subfigure}
\begin{subfigure}[b]{0.495\textwidth}
\centering
\includegraphics[width=\textwidth]{Plots/TBCalibration/Attenprofs_P4Data_CellPE_Y_Combined.pdf}
\end{subfigure}
\begin{subfigure}[b]{0.495\textwidth}
\centering
\includegraphics[width=\textwidth]{Plots/TBCalibration/Attenprofs_P4Data_PlanePE_X_Combined.pdf}
\end{subfigure}
\begin{subfigure}[b]{0.495\textwidth}
\centering
\includegraphics[width=\textwidth]{Plots/TBCalibration/Attenprofs_P4Data_PlanePE_Y_Combined.pdf}
\end{subfigure}
\caption[Uncorrected energy response along $w$, cell and plane for period 4]{Uncorrected average energy response as a function of the position within a cell ($w$ - top), cell number (middle), or plane number (bottom) for the Test Beam detector period 4 data of cosmic muons hits selected for calibration. Left side shows distributions for the X view (vertical) planes and right side for the Y view (horizontal) planes. Each plot is a profile histogram, with uncertainties representing statistical variations. The commissioning runs in epoch 4a and the runs during the cell masking studies have a visibly different energy deposition across all the shown variables compared to the rest of the period 4 runs.}
\label{fig:Calibhist_period4}
\end{figure}

\subsection*{Period 4 relative calibration results}

Results of the attenuation fits for period 4 are summarised in Fig.~\ref{fig:CellCentreResponsePeriod4} and Tab.~\ref{tab:TestBeamPeriod4RelCalibResults}. We can see that almost the entire detector is now calibrated, with only few exceptions on the edges of the detector and a single cell with an unusually high response at the edge (right plot of Fig.~\ref{fig:AttenfitResultsPeriod4}). We treated the formerly underfilled cells the same way as in epochs 3d and 3e, manually changing the $\chi^2$ of their attenuation fits inside the csv files to $<0.2$, therefore making them officially calibrated. There are 108 uncalibrated cells out of 4032, totalling 2.68\% of the detector.

\begin{figure}[!hbtp]
\centering
\includegraphics[width=\textwidth]{Plots/TBCalibration/CellResponseAtCentre_period4_original_Limited_NOvAPlotStyle.pdf}
\includegraphics[width=\textwidth]{Plots/TBCalibration/CellResponseAtCentre_period4_Limited_NOvAPlotStyle.pdf}
\caption[Map of fitted response at cell centre for period 4 data]{Overview of the relative calibration results for the Test Beam detector period 4 data. Top plot shows the results of the attenuation fit and bottom plot shows the final result for period 4 after manually labelling the originally uncalibrated refilled cells as calibrated. Each cell represents the result of the attenuation fit to the energy response in the centre of that cell. The blank cells are uncalibrated. The uncalibrated cells are concentrated on the edge of the detector, with a single cell 47 in plane 54 with an unusually high response at the edge of the cell. The 7 previously uncalibrated cells in the middle of the detector were artificially marked as calibrated after careful considerations.}
\label{fig:CellCentreResponsePeriod4}
\end{figure}

\begin{table}[!hbtp]
\centering
\caption[Summary of relative calibration results for period 4]{Summary of relative calibration results for period 4 with the uncalibrated cells divided into four categories based on the main reason of failure, all described in text. Brackets show the number of cells that were originally (un)calibrated before the manual alteration of their $\chi^2$ values, as described in text. Proportions are calculated from the final cell counts.}
\def\arraystretch{1.4}
\begin{tabular}{|cl|c|c|}
\hline
\multicolumn{2}{|c|}{\textbf{Calibration status}} & \textbf{Number of cells} & \textbf{Detector proportion}\\\hline
\multicolumn{2}{|c|}{Calibrated} & $3924 (3917)$ & $\unit[97.32]{\%}$\\\hline
\parbox[t]{2mm}{\multirow{4}{*}{\rotatebox[origin=c]{90}{Uncalibrated }}} & Peripheral cells & $97$ & $\unit[2.41]{\%}$\\
 & Underfilled cells & $10 (17)$ & $\unit[0.25]{\%}$\\
 & Readout & $0$ & $\unit[0.00]{\%}$\\
 & Binning & $1$ & $\unit[0.02]{\%}$\\\hline
\end{tabular}
\label{tab:TestBeamPeriod4RelCalibResults}
\end{table}

\begin{figure}[h]
  \begin{subfigure}{0.495\textwidth}
    \includegraphics[width=\linewidth]{Plots/RelativeCalibrationResults/p4_035_031.png}
  \end{subfigure}
  \begin{subfigure}{0.495\textwidth}
    \includegraphics[width=\linewidth]{Plots/RelativeCalibrationResults/p4_054_047.png}
  \end{subfigure}
  \caption[Attenuation fits for cells with mixed oils in period 4 data]{Fit to the energy response in period 4. Previously underfilled cells refilled with a scintillator of a different quality causing an unusual distribution of energy deposition (left). Unusually high energy response at the edge of the cell 47 (right).}
  \label{fig:AttenfitResultsPeriod4}
\end{figure}

\FloatBarrier
%%%%%%%%%%%%%%%%%%%%%%%%%%%%%%%%%%%%%%%%%%%%%%%%%%%%%%%%%%%%%%%%%%%%%%%%%%%%%%%
%%%%%%%%%%%%%%%%%%%%%%%%%%%%%%%%%%%%%%%%%%%%%%%%%%%%%%%%%%%%%%%%%%%%%%%%%%%%%%%
%%%
%%%                        Absolute calibration results
%%%
%%%%%%%%%%%%%%%%%%%%%%%%%%%%%%%%%%%%%%%%%%%%%%%%%%%%%%%%%%%%%%%%%%%%%%%%%%%%%%%
\section{Absolute calibration results}\label{sec:TBAbsoluteCalib}
The results of the relative calibration (without the threshold and shielding correction) are applied to the stopping muon sample to calculate the absolute energy scale,  which translates the energy response from \gls{PECorr} to $\unit{GeV}$, as described in Sec.~\ref{sec:NOvACalibration}. We apply the absolute calibration cuts to select minimum ionising muons, which represent a very well understood source of energy deposition. The absolute calibration cuts are mostly the same as for the other \gls{NOvA} detectors, selecting hits $1-\unit[2]{m}$ from the end of their tracks and removing uncalibrated and wrongly reconstructed hits by requiring non-zero path lengths, \gls{PE}$>0$, \gls{PECorr}$>0$, as well as \gls{PECorr}$\unit{/cm}<100$. Additionally, we constrain $w$ to a smaller allowed range: $-80<w<\unit[80]{cm}$, reflecting the smaller Test Beam cell length, removing hits approximately $\unit[0.5]{m}$ from each side of the detector.

Distributions of reconstructed and true energy responses, for both views, and for each data and simulation sample, are shown in Fig.~\ref{fig:AbsCalibNHitsMEU}. The mean of each of these distributions constitute the \gls{MEU}$_{\mathrm{Reco}}$ or \gls{MEU}$_{\mathrm{True}}$ values for both views. We calculate the statistical uncertainty on the \gls{MEU} values as the standard deviation of the corresponding distributions divided by the square root of the number of entries. To combine the result from the two views, we take the average over the view-dependent \gls{MEU} values to obtain the final \gls{MEU} value for each sample. This is the first time in the calibration chain where the two views, which were treated completely independently so far, are combined together. The uncertainties are added in the sum of squares. The total number of entries, the \gls{MEU} values for each sample and view, as well as the combined \gls{MEU} values with corresponding statistical uncertainties are shown in Tab.~\ref{tab:calib_summary_table}. Given the large number of entries in the energy response distributions, the statistical uncertainties on the \gls{MEU} values are negligible (around $0.05\%$). These are however not the final uncertainties of the absolute energy scale used in \gls{NOvA}. Instead, we use comparison to other standard candles, as was explained in Sec.~\ref{sec:NOvASystematics}.

\begin{figure}[h!]
  \begin{subfigure}{\textwidth}
    \centering
    \includegraphics[height=0.2\linewidth]{Plots/Calibana/legend.pdf}
  \end{subfigure}
  \vspace*{2mm}

  \begin{subfigure}{0.495\textwidth}
    \includegraphics[width=\linewidth]{Plots/Calibana/nhits_meu_x.pdf}
  \end{subfigure}
  \begin{subfigure}{0.495\textwidth}
    \includegraphics[width=\linewidth]{Plots/Calibana/nhits_meu_y.pdf}
  \end{subfigure}
  \begin{subfigure}{0.495\textwidth}
    \includegraphics[width=\linewidth]{Plots/Calibana/nhits_mev_x.pdf}
  \end{subfigure}
  \begin{subfigure}{0.495\textwidth}
    \includegraphics[width=\linewidth]{Plots/Calibana/nhits_mev_y.pdf}
  \end{subfigure}
  \caption[Reconstructed and true energy responses of stopping muons]{Distributions of the reconstructed (top) and true (bottom) energy response of stopping muons in the X (left) and Y (right) view within a $1-\unit[2]{m}$ track window from the end of their tracks. The mean of the reconstructed and true distributions of the response are the reconstructed and true MEU values respectively for the corresponding views.}
  \label{fig:AbsCalibNHitsMEU}
\end{figure}

\begin{table}[h!]
\centering
\caption[Summary of absolute calibration results]{Summary of absolute calibration results. \acrshort{MEU}$_{\mathrm{Reco}}$ values (top table), including the statistical uncertainty $\sigma_{\textsf{MEU}_{\mathrm{Reco}}}$, are in units of \acrshort{PECorr}$\unit{/cm}$ and \acrshort{MEU}$_{\mathrm{True}}$ values (bottom table) are in units of $\unit{MeV/cm}$}
\begin{tabular}{|c|c|c|c|c|c|c|c|}
\hline
\multicolumn{2}{|c|}{\multirow{2}{*}{Sample}} & \multicolumn{2}{c|}{X view} & \multicolumn{2}{c|}{Y view} & \multicolumn{2}{c|}{Combined}\\\cline{3-8}
\multicolumn{2}{|c|}{} & NHits & MEU & NHits & MEU & \cellcolor[HTML]{F8A102}MEU$_{\mathrm{Reco}}$ & $\sigma_{\textsf{MEU}_{\mathrm{Reco}}}$\\ \hline
 \parbox[t]{2mm}{\multirow{4}{*}{\rotatebox[origin=c]{90}{Data}}}
 & Period 2 & 2.322e+05 & 38.70 & 1.413e+06 & 39.40 & \cellcolor[HTML]{F8A102}39.05 & 0.02\\ \cline{2-8} 
 & Epochs 3abc & 2.638e+05 & 38.49 & 1.621e+06 & 39.40 & \cellcolor[HTML]{F8A102}38.94 & 0.02\\ \cline{2-8}
 & Epochs 3de & 1.049e+05 & 38.63 & 6.725e+05 & 39.42 & \cellcolor[HTML]{F8A102}39.02 & 0.03\\ \cline{2-8}
 & Period 4 & 5.268e+05 & 38.63 & 3.316e+06 & 39.40 & \cellcolor[HTML]{F8A102}39.01 & 0.01\\ \hline
\multicolumn{2}{|c|}{Simulation} & 2.829e+05 & 40.17 & 1.842e+06 & 39.93 & \cellcolor[HTML]{F8A102}40.05 & 0.02\\ \hline
\end{tabular}

\vspace*{2mm}
\begin{tabular}{|c|c|}
\hline
\cellcolor[HTML]{F8A102}MEU$_{\mathrm{True}}$ = 1.7722 $\unit{MeV/cm}$ & $\sigma_{\textsf{MEU}_{\mathrm{True}}}$ = 0.0003 $\unit{MeV/cm}$\\ \hline
\end{tabular}
\label{tab:calib_summary_table}
\end{table}

As expected, the comparison of the absolute calibration results in Fig.~\ref{fig:AbsCalibNHitsMEU} and Tab.~\ref{tab:calib_summary_table} demonstrates that the \gls{MEU} values across the four data samples are consistent, particularly in the Y view, which has larger statistics. However, the \gls{MEU}$_{Reco}$ values are noticeably higher for simulation than for data, especially in the X view (vertical planes). This discrepancy is anticipated, as through-going muons in the new data-based simulation (Sec.~\ref{sec:DataBasedSimulation}) have incorrect (smaller) incident energies, leading to smaller mean deposited energies  used in the attenuation fits and consequently larger relative calibration corrections. These larger corrections are then applied to the correctly simulated stopping muons, resulting in higher \gls{PECorr} and therefore larger \gls{MEU}$_{\mathrm{Reco}}$. Additionally, the true deposited energy, and therefore the \gls{MEU}$_{\mathrm{True}}$, which is also used for data, should be accurate.

%This is caused due to the data-based simulation we are using does not have a correct energy estimation for through-going muons, which have generally underestimated energies \cite{NOVA-doc-60026}. This results in an over-estimated correction from the relative calibration. However, this is not an issue, since we only use stopping muons to calculate the absolute energy scale and stopping muons have correct energies in the new simulation.

There is a noticeable discrepancy of about $\unit[1]{\%}$ in the \gls{MEU}$_{\mathrm{Reco}}$ values between the two views. Given the minimal statistical uncertainties and the consistency of results across the samples, this is unlikely to be a random effect. Additionally, the discrepancy between the two views is in a different direction for the data samples and for simulation. The actual reason for this discrepancy is unknown; it could be due to a real difference in the stopping muon distribution between the two views that is not accounted for, or a systematic difference in the calibration treatment of the two views. This effect is observed in all \gls{NOvA} detectors \cite{NOvA-doc-60709}. Since this means that the final result is technically incorrect for both views, one possible mitigation is to apply the absolute calibration results to each view separately. However, this contradicts the logic that stopping muons can serve as a standard candle, providing a single final calibration value.

%%%%%%%%%%%%%%%%%%%%%%%%%%%%%%%%%%%%%%%%%%%%%%%%%%%%%%%%%%%%%%%%%%%%%%%%%%%%%%%
%%%%%%%%%%%%%%%%%%%%%%%%%%%%%%%%%%%%%%%%%%%%%%%%%%%%%%%%%%%%%%%%%%%%%%%%%%%%%%%
%%%
%%%                      Final results and conclusions
%%%
%%%%%%%%%%%%%%%%%%%%%%%%%%%%%%%%%%%%%%%%%%%%%%%%%%%%%%%%%%%%%%%%%%%%%%%%%%%%%%%
\section{Validation}\label{sec:TBCalibValidation}

The initial validation of the Test Beam detector calibration is performed using the same cosmic muons that were used in the calibration process. This step is essential to ensure that the calibration performs as intended and successfully unifies the energy deposition of cosmic muons across the detector (as a function of $w$, cell and plane) and throughout the Test Beam detector runtime. By analysing these validation results and investigating any residual differences, we assess the stability and quality of the calibration.

After the initial validation with cosmic muons, we apply the calibration results to beam events. For the \gls{ND} and the \gls{FD}, we use a selection of standard candles, as described in Sec.~\ref{sec:NOvASystematics}, to evaluate the performance of the cosmic-based calibration on beam events and to assess the calibration systematic uncertainties. However, Test Beam offers a unique opportunity to validate the detector calibration directly with measurements from its beamline and to reassess and potentially reduce the systematic uncertainties associated with detector calibration in \gls{NOvA}. This is going to be one of the main results of the Test Beam experiment, therefore, we are focusing solely on the cosmic-based validation process.

The following section highlights the most important features observed during the validation, with additional plots provided in Appendix~\ref{sec:AppTBCalibValid} for the reader's convenience.

\subsection{Validation with stopping muons}

First, we examine the calibration effects on the same stopping muon hits as were used for the absolute calibration (Sec.~\ref{sec:TBAbsoluteCalib}), including all the absolute calibration cuts. These stopping muons, which have a very reliable and stable energy deposition, are used to compare the calibration performance amongst the various data and simulation samples. However, since we require hits to be within $\unit[1-2]{m}$ from the end of the stopping muons' tracks, they are not evenly distributed across the detector, particularly absent in its bottom parts. This causes large statistical uncertainties and scattered distributions along $w$ in the X view (for hits with $w<0$), and across cells in the Y view (for cells $<32$).

Figures \ref{fig:AbsCalibW1}-\ref{fig:AbsCalibDrift1} present the distributions of uncorrected (in \gls{PE}$\unit{/cm}$) and corrected (in $\unit{MeV/cm}$) energy deposition as a function of $w$, cell number, plane number, and time. The uncorrected energy deposition distributions are displaying the same general attributes as were discussed for each sample in previous sections. As expected, the corrected energy deposition is generally uniform across all the studied variables. However, some residual variations can be noticed and are discussed below.

% General w distribution - all good
The distributions as a function of $w$ (Fig.~\ref{fig:AbsCalibW1}) illustrate the successful uniformity of energy deposition after applying calibration. Excluding the region affected by the lack of stopping muon hits, the corrected energy deposition for each sample is uniform within $\pm\unit[0.5]{\%}$. Additionally, all four data samples are consistent with each other, and the discrepancy between data and simulation is within $\pm\unit[1.5]{\%}$.

\begin{figure}[!ht]
  \begin{subfigure}{\textwidth}
  \centering
    \includegraphics[height=0.2\linewidth]{Plots/Calibana/legend.pdf}
  \end{subfigure}
  \vspace*{2mm}
  
  \begin{subfigure}{0.495\textwidth}
    \includegraphics[width=\linewidth]{Plots/Calibana/pecm_w_x.pdf}
  \end{subfigure}
  \begin{subfigure}{0.495\textwidth}
    \includegraphics[width=\linewidth]{Plots/Calibana/pecm_w_y.pdf}
  \end{subfigure}
  \begin{subfigure}{0.495\textwidth}
    \includegraphics[width=\linewidth]{Plots/Calibana/recomevcm_w_x.pdf}
  \end{subfigure}
  \begin{subfigure}{0.495\textwidth}
    \includegraphics[width=\linewidth]{Plots/Calibana/recomevcm_w_y.pdf}
  \end{subfigure}
  \caption[Validation plots for stopping muons along w]{Distributions of uncorrected (top) and corrected (bottom) energy deposition for stopping \acrshort{MIP} muons in the X view (left) and the Y view (right) as a function of the position within a cell. Bottom panel of each plot shows the ratio of the simulation sample (gray) and the four data samples, labelled at the top. Ep3a labels a combination of epochs 3a+3b+3c and Ep3d labels epochs 3d+3e. The left half ($w<0$) of the X view distributions has large statistical uncertainties due to the low number of stopping muons at the bottom of the detector. The discrepancy between the data and the simulation samples for the corrected energy depositions is explained in text.}
  \label{fig:AbsCalibW1}
\end{figure}

% The uncorrected response is clearly getting lower with time
The distributions of uncorrected energy deposition as a function of $w$ (top of Fig.~\ref{fig:AbsCalibW1}) demonstrate a relative decrease in energy response over time, with period 4 data exhibiting a significantly smaller uncorrected energy response than period 2 data. This decrease, however, is corrected by calibration as expected.

% Generally different scale for simulation in X and Y views
The data-simulation discrepancy for the corrected energy deposition along $w$ (bottom of Fig.~\ref{fig:AbsCalibW1}) varies in the opposite direction between the X and Y views. This discrepancy arises from averaging the two view-dependent \gls{MEU}$_{\mathrm{Reco}}$ values, which show opposite variation in simulation compared to data, as explained in Sec.~\ref{sec:TBAbsoluteCalib}. Ideally, there should be no data-simulation discrepancy after applying the full calibration results. Therefore, applying the view-dependent absolute calibration results separately to each view, which would likely resolve this issue, is worthwhile to consider.

% Cell distributions - is this caused by the APD gains and the threshold correction?
The distributions of energy deposition across cells in Fig. \ref{fig:AbsCalibPlane1} exhibit greater variability after calibration compared to the $w$ dependence. This variability, particularly noticeable in the X view, is caused by issues with the threshold and shielding correction discussed in Sec.~\ref{sec:TBThresholdCorrection}. The threshold and shielding correction is not applied to these distributions, just as it is not applied to the stopping muon sample for absolute calibration or to beam events, as it is not supposed to affect them. However, since the incorrect threshold and shielding correction is applied during relative calibration, it is incorporated into the calibration results and consequently into the reconstructed deposited energy. The variability introduced by these faulty corrections is within $\pm\unit[3.5]{\%}$.

%Why does it appear there is a larger variation in the X view? Is it purely statistics?

\begin{figure}[!ht]
  \begin{subfigure}{\textwidth}
  \centering
    \includegraphics[height=0.2\linewidth]{Plots/Calibana/legend.pdf}
  \end{subfigure}
  \vspace*{2mm}

  \begin{subfigure}{0.495\textwidth}
    \includegraphics[width=\linewidth]{Plots/Calibana/pecm_cell_x.pdf}
  \end{subfigure}
  \begin{subfigure}{0.495\textwidth}
    \includegraphics[width=\linewidth]{Plots/Calibana/pecm_cell_y.pdf}
  \end{subfigure}
  \begin{subfigure}{0.495\textwidth}
    \includegraphics[width=\linewidth]{Plots/Calibana/recomevcm_cell_x.pdf}
  \end{subfigure}
  \begin{subfigure}{0.495\textwidth}
    \includegraphics[width=\linewidth]{Plots/Calibana/recomevcm_cell_y.pdf}
  \end{subfigure}
  \caption[Validation plots for stopping muons across cells]{Distributions of uncorrected (top) and corrected (bottom) energy deposition for stopping \acrshort{MIP} muons in the X view (left) and the Y view (right) as a function of the cell number. Bottom panel of each plot shows the ratio of the simulation sample (gray) and the four data samples, labelled at the top. Ep3a labels a combination of epochs 3a+3b+3c and Ep3d labels epochs 3d+3e. The left part (cell $\lesssim 25$) of the Y view distributions has large statistical uncertainties due to the low number of stopping muons at the bottom of the detector. Features described in text.}
  \label{fig:AbsCalibCell1}
\end{figure}

% Cell distribution in Y view - what is up with that middle cells - underfilled?
The distribution of corrected energy across cells in the Y view (bottom right of Fig.~\ref{fig:AbsCalibCell1}) additionally shows two cells with a noticeably lower energy response (\mbox{$\sim\unit[2-4]{\%}$}) for period 2 and epochs 3a+3b+3c compared to the rest of the samples. Specifically, this is the cell 31, which was underfilled during period 2 and epoch 3a, and its neighbouring cell 32. The variation for the underfilled cell is expected. However, it is unclear why one of the neighbouring cells to the underfilled cells is miscalibrated.

% Uncorrected energy response for different sammples - seems like the old scintillator is ageing the most, the Ash river the least and the Texas oil is somewhere in the middle
The relative differences in the uncorrected energy response across planes (top of Fig.~\ref{fig:AbsCalibPlane1}) between the three different data taking periods demonstrate that the decrease in energy deposition over time varies between planes. This variation is especially noticeable in the X view plot, but is equally present in the Y view planes. This effect is attributed to the differences in the quality of the scintillator oil used to fill the detector, as explained in Sec.~\ref{sec:TBExperiment}. Additionally, it indicates scintillator ageing, which results in a decrease of the scintillation light produced per deposited energy over time. Specifically, the \texttt{ND+NDOS} scintillator oil (planes 0-31) appears to have aged the most between periods 2 and 4, followed by the Texas \texttt{NDOS} oil (planes 53-62), while the Ash River oil (planes 32-52) aged the least. However, more quantitative studies are necessary to assess the ageing of the \gls{NOvA} scintillator, as its details are currently not well understood.

\begin{figure}[!ht]
  \begin{subfigure}{\textwidth}
  \centering
    \includegraphics[height=0.2\linewidth]{Plots/Calibana/legend.pdf}
  \end{subfigure}
  \vspace*{2mm}

  \begin{subfigure}{0.495\textwidth}
    \includegraphics[width=\linewidth]{Plots/Calibana/pecm_plane_x.pdf}
  \end{subfigure}
  \begin{subfigure}{0.495\textwidth}
    \includegraphics[width=\linewidth]{Plots/Calibana/pecm_plane_y.pdf}
  \end{subfigure}
  \begin{subfigure}{0.495\textwidth}
    \includegraphics[width=\linewidth]{Plots/Calibana/recomevcm_plane_x.pdf}
  \end{subfigure}
  \begin{subfigure}{0.495\textwidth}
    \includegraphics[width=\linewidth]{Plots/Calibana/recomevcm_plane_y.pdf}
  \end{subfigure}
  \caption[Validation plots for stopping muons across planes]{Distributions of uncorrected (top) and corrected (bottom) energy deposition for stopping \acrshort{MIP} muons in the X view (left) and the Y view (right) as a function of the plane number. Bottom panel of each plot shows the ratio of the simulation sample (gray) and the four data samples, labelled at the top. Ep3a labels a combination of epochs 3a+3b+3c and Ep3d labels epochs 3d+3e. Features described in text.}
  \label{fig:AbsCalibPlane1}
\end{figure}

% Plane distribution in X view - large discrepancy in ep3abc
The distribution of the corrected energy response across planes in the X view (bottom left of Fig.~\ref{fig:AbsCalibPlane1}) reveals a significantly smaller response ($\sim\unit[16]{\%}$) for epochs 3a+3b+3c in plane 36. This indicates that the relative calibration over-corrected the energy response due to through-going muons having an unusually high energy response (as shown in Fig. \ref{fig:AttenfitResultsEpoch3abc_FaultyFEBs}), but not the selected stopping muons. The most likely explanation is that the affected \gls{FEB} was `faulty' only for a certain period. Consequently, the corrected energy response would be accurate for the period when the \gls{FEB} was faulty, but would be under-estimated for the period when the \gls{FEB} functioned normally.

%Also planes 16 and 48, same as planes 17 49
%FB map tell Sim to simulation the FEBv5 lower than the rest, but readout simulation shifts it higher. That's why it is incorrect in the threshold correction and in simulation
We can observe the effect of different \gls{FEB} versions in both the corrected and uncorrected energy response distributions across planes, shown in Fig.~\ref{fig:AbsCalibPlane1}. Specifically, the \gls{FEB}v5 was used in planes 16, 17, 48 and 49, which is evident by the relatively larger corrected energy deposition, especially in the simulation and in the X view. Although this effect should be corrected during calibration, it is not, due to the incorrect incorporation of the variations between the \gls{FEB} versions in the simulation and its impact on the threshold and shielding correction, as discussed in Sec.~\ref{sec:TBThresholdCorrection}.

% Shape of plane distribution in X view
% Why are planes 2 (for all data periods) and planes 54 and 56 for all periods but ep3abc so much lower? But this seems to be only present of the stopping muons as will be explained below...
There are additional variations in the corrected energy deposition across planes (bottom of Fig.~\ref{fig:AbsCalibPlane1}). In the X view, there is a significant ($\sim\unit[6]{\%}$) drop in corrected energy response in planes 4 and 58, along with a smaller drop in some surrounding planes. The cause of this variation is unknown. However, it appears to be due to a discrepancy between through-going and stopping muons, as it disappears when analysing through-going muons, as explain below. In the Y view, the corrected response increases with planes  within the first half of the detector. This effect is only observed in data and not in simulation. The origin of this slope is unclear, however it is also absent when examining variations for through-going muons.

%Dependence in time - maybe also mention the systematic uncertainty?
%Large variability; Only normalizing mean; Visibly smaller uncorrected response in period 4, however the trend is not clear and the fluctuations are larger than the possible effect of detector ageing; Others addressed this by calibrating smaller samples; TB has more variability due to the environment and less stable detector running conditions; This provides an opportunity to test these effect as they are not well understood within NOvA
The distributions of energy deposition over time (Fig.~\ref{fig:AbsCalibDrift1}) reveal a complex dependency and significant variations. As shown, the calibration process currently only normalizes the mean of each calibrated sample, leaving time-dependent variations within the samples uncorrected. The uncorrected energy response is noticeably higher for period 2 compared to period 4, likely due to detector ageing, as previously discussed. However, there is no clear downward-going trend; the variations are larger than any consistent time-dependent pattern, except fir the second half of period 4.

\begin{figure}[!ht]
  \begin{subfigure}{\textwidth}
    \centering
    \includegraphics[height=0.2\linewidth]{Plots/Calibana/legend.pdf}
  \end{subfigure}
  \vspace*{2mm}
  
  \begin{subfigure}{0.495\textwidth}
    \includegraphics[width=\linewidth]{Plots/Calibana/pecm_time_x.pdf}
  \end{subfigure}
  \begin{subfigure}{0.495\textwidth}
    \includegraphics[width=\linewidth]{Plots/Calibana/pecm_time_y.pdf}
  \end{subfigure}
  \begin{subfigure}{0.495\textwidth}
    \includegraphics[width=\linewidth]{Plots/Calibana/recomevcm_time_x.pdf}
  \end{subfigure}
  \begin{subfigure}{0.495\textwidth}
    \includegraphics[width=\linewidth]{Plots/Calibana/recomevcm_time_y.pdf}
  \end{subfigure}
  \caption[Validation plots for stopping muons along time]{Distributions of uncorrected (top) and corrected (bottom) energy deposition for stopping \acrshort{MIP} muons in the X view (left) and the Y view (right) as a function of the event UNIX time (which starts at the beginning of the \acrshort{NOvA} data taking. Comparing the four data samples, labelled at the top. Ep3a labels a combination of epochs 3a+3b+3c and Ep3d labels epochs 3d+3e. Features described in text.}
  \label{fig:AbsCalibDrift1}
\end{figure}

In the \gls{ND} and \gls{FD} calibration, these time-dependent variations are partially addressed by reducing the size of the calibration samples to month-long epochs. This approach reduces variations in the calibrated energy deposition but generally results in larger portions of the detector being uncalibrated. However, this did not have a significant effect on the rate of calibrated cells for the \gls{ND} and \gls{FD} \cite{NOvA-doc-60838} and in the future can be explored for Test Beam detector as well. The \gls{ND} and \gls{FD} typically exhibit smaller variations in energy response over time compared to the Test Beam detector due to the Test Beam detector's less stable running conditions, such as larger fluctuations in temperature and humidity in the Test Beam hall.

The effect of these environmental factors on detector performance, along with scintillator and potential readout electronics ageing, are not well understood within \gls{NOvA}. As these effects are more pronounced in the Test Beam detector, and given the range of scintillator oils and readout electronics used, separating the effects of the individual factors is challenging and is currently the focus of ongoing studies \cite{NOvA-doc-59591}.

%%% PCListAna
\subsection{Validation with through-going muons}
To validate the calibration performance without the limitations imposed by stopping muons, we examine the effects of calibration on through-going muons, which were also used for the relative calibration. Unlike for to the stopping muon sample, we apply the threshold and shielding corrections for the through-going muons to truly verify the validity of the relative calibration process. Since the simulated through-going muons have incorrect incident energies, this validation focuses solely on variations in shape rather than scale.

% Stable runs - say that I'm only taking the data from stable runs
Considering the variability of detector performance over time, we use through-going muons collected during `stable runs' in period 4, as depicted in Fig.~\ref{fig:ValidStableRuns}. This period was selected by visually inspecting the time dependence of corrected energy deposition and choosing runs that correspond to seemingly stable energy deposition.

\begin{figure}[!ht]
  \centering
  \includegraphics[width=0.8\linewidth]{Plots/PCListAna/Period4StableRuns.pdf}
  \caption[Selection of stable runs from the period 4 Test Beam data]{Selection of stable runs from the period 4 Test Beam data. Showing the reconstructed energy response as a function of the run number for through-going cosmic muons. Shaded red area shows rejected runs.}
  \label{fig:ValidStableRuns}
\end{figure}

% Edge effects - show one plot and say it's taken care of by the calibration shape uncertainty (which I haven't done)
Furthermore, Fig.~\ref{fig:ValidLimW} shows significant variation of energy deposition at cell edges. This issue is addressed by the calibration shape systematic uncertainty discussed in Sec.~\ref{sec:NOvASystematics}. Therefore, for this validation study we ignore the edge variations, using the same $w$ limit as for the stopping muon sample:  $-80<w<\unit[80]{cm}$.

\begin{figure}[!ht]
  \centering
  \includegraphics[width=0.6\linewidth]{Plots/PCListAna/AbsCalCuts_TBData_p4_StableRuns.pdf}
  \caption[Removing edge hits for validation with through-going muons]{Removing edge hits for validation with through-going muons. Showing only events from the stable runs of the period 4 Test Beam cosmic data.}
  \label{fig:ValidLimW}
\end{figure}

With the aforementioned constraints, we expect uniform distributions with limited variations. Residual variations should be scattered throughout the detector without any particular pattern, indicating potential errors in the calibration process. It is worth noting that investigating these distributions without applying the threshold and shielding correction led to the discovery of the issues discussed in Sec.~\ref{sec:TBThresholdCorrection}. Any remaining variations point to potential systematic uncertainties in the calibration process.

Figure~\ref{fig:ValidPCListAnaProfDists} displays individual distributions of corrected energy response as a function of $w$, plane, and cell number for both data and simulation. For through-going muons with the threshold and shielding correction applied, the calibration procedure works as expected, with residual variations around $\pm\unit[0.2]{\%}$ for the $w$ and cell dependence. However, for plane dependence, these residual variations are larger: approximately $\pm\unit[0.5]{\%}$ in data and up to $\pm\unit[1]{\%}$ in simulation. Additionally, maps of corrected energy deposition as a function of both plane and cell numbers, along with their 1D projections, are shown in Fig.~\ref{fig:ValidPCListAnaMapData} for data and in Fig.~\ref{fig:ValidPCListAnaMapSim} for simulation. These plots demonstrate that the final residual variations for calibration reach approximately $\unit[1]{\%}$ in data and $\unit[4]{\%}$ in simulation.

% Individual variations - very small
\begin{figure}[!ht]
  \begin{subfigure}{0.495\textwidth}
    \includegraphics[width=\linewidth]{Plots/PCListAna/DataAndSim_recomevcm_ts_w_X.pdf}
  \end{subfigure}
  \begin{subfigure}{0.495\textwidth}
    \includegraphics[width=\linewidth]{Plots/PCListAna/DataAndSim_recomevcm_ts_w_y.pdf}
  \end{subfigure}
  \begin{subfigure}{0.495\textwidth}
    \includegraphics[width=\linewidth]{Plots/PCListAna/DataAndSim_recomevcm_ts_cell_x.pdf}
  \end{subfigure}
  \begin{subfigure}{0.495\textwidth}
    \includegraphics[width=\linewidth]{Plots/PCListAna/DataAndSim_recomevcm_ts_cell_y.pdf}
  \end{subfigure}
    \begin{subfigure}{0.495\textwidth}
    \includegraphics[width=\linewidth]{Plots/PCListAna/DataAndSim_recomevcm_ts_plane_x.pdf}
  \end{subfigure}
  \begin{subfigure}{0.495\textwidth}
    \includegraphics[width=\linewidth]{Plots/PCListAna/DataAndSim_recomevcm_ts_plane_y.pdf}
  \end{subfigure}
  \caption[Validation plots for through-going muons as a function of w, cell and plane]{Distributions of through-going cosmic muons with $w\in\left(-80,80\right)\unit{cm}$ as a function of $w$, cells and planes for stable runs in the Test Beam period 4 data (black) and data-based simulation (red). Bottom panel of each plot shows the ratio between the value for each bin and the mean across the entire y axis, calculated separately for data and simulation.}
  \label{fig:ValidPCListAnaProfDists}
\end{figure}

\begin{figure}[!ht]
  \centering
  \includegraphics[width=0.7\linewidth]{Plots/PCListAna/TBDataP4_recomevcm_ts_zoomed.pdf}
  
  \includegraphics[width=0.7\linewidth]{Plots/PCListAna/Variation_recomevcm_TBDataP4_StableRuns_LimW.pdf}
  \caption[Map of corrected energy deposition for through-going cosmic muons in data]{Top: Map of corrected energy deposition for through-going cosmic muons with $w\in\left(-80,80\right)\unit{cm}$ as a function of cell and plane numbers for stable runs in the Test Beam period 4 data. Bottom: Projection of each bin from the map. Red line shows the median value of the projection and red shaded area shows the $\pm\unit[1]{\%}$ range of the median value. Gray shaded bins show the excluded bins corresponding to faulty \acrshort{FEB}, visible as a stark red line in plane 19 on the map. Additional features explained in text.}
  \label{fig:ValidPCListAnaMapData}
\end{figure}

\begin{figure}[!ht]
  \centering
  \includegraphics[width=0.7\linewidth]{Plots/PCListAna/TBSimulation_CP_recomevcm_ts_zoomed.pdf}
  
  \includegraphics[width=0.7\linewidth]{Plots/PCListAna/Variation_recomevcm_TBSimulation_LimW.pdf}
  \caption[Map of corrected energy deposition for through-going cosmic muons in simulation]{Top: Map of corrected energy deposition for through-going cosmic muons with $w\in\left(-80,80\right)\unit{cm}$ as a function of cell and plane numbers for stable runs in the Test Beam data-based simulation. Bottom: Projection of each bin from the map. Red line shows the median value of the projection and red shaded area shows the $\pm\unit[4]{\%}$ range of the median value. Additional features explained in text.}
  \label{fig:ValidPCListAnaMapSim}
\end{figure}

% Variations in data
In the distributions for data, we observe a clear effect of the different scintillators used, which is related to varying levels of detector ageing among the scintillators, as previously discussed. This effect is particularly noticeable in this validation sample due to the selection of events from `stable runs', while the relative calibration was calculated for the entire period 4. Therefore, calibration scales the deposited energy with respect to the mean deposited energy for the full period 4. Consequently, planes 0-31 appear to have the lowest relative corrected response in Fig.~\ref{fig:ValidPCListAnaProfDists} and \ref{fig:ValidPCListAnaMapData} because the corresponding \texttt{ND+NDOS} scintillator oil aged the most compared to the mean. This issue could be addressed by splitting period 4 into smaller epochs for calibration.

% Variations in simulation
For simulation, the relative calibration is performed in bins of $w$ for each cell and each \gls{FB} bin but importantly not for each plane. As a result, the relative calibration in simulation does not specifically correct for plane dependence, leading to larger variations between planes.

Furthermore, one half of plane 19 in data shows a corrected energy response approximately $\unit[0.7]{\%}$ higher than the rest. This is due to a faulty \gls{FEB} that was likely malfunctioning for a brief period. Since this study aims to examine purely the residual variations from the calibration procedure, cells corresponding to this \gls{FEB} can be ignored. Additionally, planes 48 and 49 in simulation, which correspond to \gls{FEB}v5, display a corrected energy response about $\unit[2-2.2]{\%}$ higher than the rest. This discrepancy arises from the incorrect accounting for different \gls{FEB} versions in simulation, combined with consolidating the planes into \gls{FB} bins.

\section{Summary}\label{sec:TBCalibSummary}

%Everything was great
I have successfully calibrated the NOvA Test Beam detector in four independent Test Beam data samples and in simulation. This is a critical step in order to analyse the Test Beam data, enabling crucial improvements to our understanding of the detector response and particularly of the calibration for all the \gls{NOvA} detectors.

Each calibrated sample contains only a few uncalibrated cells, mainly concentrated on the periphery of the detector. Most of the other uncalibrated cells failed the calibration condition due to unavoidable circumstances, such as underfilled cells, or dead channels. However, several improvements could increase the number of calibrated cells or otherwise improve the calibration performance. Some of the improvements could benefit not only the Test Beam detector but all the \gls{NOvA} detectors. These are especially the improved threshold and shielding correction, the individual treatment of cells during calibration, or the time variation studies of the \gls{NOvA} scintillator and electronics at the Test Beam detector, all outlined below.

%%% Threshold and shielding corrections and simulation issues - all NOvA
Several issues discovered during the Test Beam calibration that will enhance calibration and simulation for all the \gls{NOvA} detectors include the problems found with the threshold and shielding correction and with the readout simulation, described in Sec.~\ref{sec:TBThresholdCorrection}. Ongoing work aims to address these issues by improving the logic of the threshold and shielding correction or basing it entirely on data. Additionally, plans are in place to remove the relative gain variation from the simulation, as the \gls{FB} map used during simulation already sufficiently describes the relative gain variations, making the current treatment redundant.

%%% Faulty FEBs - transpose files - all NOvA, but might be complicated for FD
One of the most common issues that could be resolved is the faulty \glspl{FEB}, which often malfunctioned only for a limited period. This is particularly concerning, as cells belonging to faulty \glspl{FEB} usually pass the calibration condition; however, if the corresponding \gls{FEB} was faulty only for a limited period, the calibration results for those cells would be incorrect. This issue is currently difficult to address because the inputs for the relative calibration are organized by run and subrun numbers instead of by cells, making it impossible to calibrate cells corresponding to a single \gls{FEB} separately from the rest. However, ongoing work is adapting both the input files and the calibration procedures to enable calibrating individual cells separately.

%%% Time dependece - needs to be studied further, unique opportunity to understand this in all of nova, can be addressed by splitting
The time dependence of energy deposition in the Test Beam detector discussed in Sec.~\ref{sec:TBCalibValidation} is currently being studied and could help explain several contributions to the time-dependency seen in all \gls{NOvA} detectors. Specifically, the Test Beam can study the environmental effects on energy deposition, as well as the ageing of different scintillators and electronics. The first results indicate that the \gls{NOvA} scintillator oils age differently based on their quality. 

Addressing the time-dependent variations in the energy deposition in Test Beam could include splitting the calibrated periods into smaller samples, similarly to the \gls{ND} and \gls{FD}, which are using month-long samples for calibration. However, this approach risks insufficient statistics for attenuation fits in various cells. This could in turn be addressed by tuning the binning of the attenuation profiles to better fit the real cell dimensions that are actually being fitted in the attenuation fits.

%%% Better binning
%Several cells have failed the calibration condition due to a combination of low statistics and improper binning in the attenuation profiles. To  correct the dependence on time, the idea is to use smaller samples calibration, this effect could be enlarged. It's possible to partially tackle it by improving the binning of the attenuation profiles t
%Other suggestion for improvement is to change the binning of profile histograms to avoid the issues of single strange bins on the edges of the attenuation fit.

%%% Manually changing the chi2
Another technical improvements to the calibration results could stem from manually altering the $\chi^2$ value of cells that have attenuation fits which visibly looks all right, to below $0.2$. This would officially rendering these cells calibrated, similarly to how it was done for some cells in epochs 3d+3e and period 4. However, this would require manually going over all the uncalibrated cells, which is only really possible for the Test Beam detector and only for larger calibrated samples. Reducing the size of the calibrated sample would make this extremely impractical.

%Philosophical question of whether to manually change the chi2 values just cause by eye the response looks all right.

%%% Applying the absolute calibration for the views separately
Lastly, there is a possible improvement to the data-simulation discrepancy, by applying the absolute calibration results to each view independently. This would however have to be seriously considered, as the stopping muon sample used for calibration should technically not have any variation between the views and therefore this might introduce bias when applying the calibration results to beam data.
%Also applying the absolute calibration to the two views independently. - Is this actually what we'd want? Can still discuss it here...

%%% Possibly using Test Beam data to calculate the absolute energy scale

%%% Systematics
%Also have to devise proper systematic uncertainty for the TB calibration.
%We haven't attempted to estimate the uncertainty of the calibration, although we can estimate the final residual variations after applying the calibration. However, more work will need to go into this,  specifically to understand the time dependence of the deposited energy and the edge effects.

Overall, the calibration of the \gls{NOvA} Test Beam detector was successful and represents a significant advancement, with ongoing efforts aimed at further refining the process and addressing the identified issues to improve calibration accuracy and reliability across all \gls{NOvA} detectors. Part of the ongoing work also aims to quantify the correctness of the Test Beam calibration, devising a concrete systematic uncertainty.

As a result of this effort, \gls{NOvA} will be able to improve its analyses by reducing calibration-related systematic uncertainties. One key improvement is the ability to assess the absolute energy scale uncertainty - the largest of the calibration-related systematic uncertainties - separately for different particle types. Existing studies indicate that this uncertainty could be reduced from an overall $5\%$ to approximately $1.5\%$ for muons and $3\%$ for electromagnetic showers \cite{NOVA-doc-53225}. However, these estimates require validation from the Test Beam experiment, which is expected to further refine and lower these uncertainties. Additionally, correcting for threshold and shielding effects, along with other proposed calibration improvements, will likely enhance the calibration performance and reduce variations between detector cells.

%%% Discussion - Systematic uncertainties

%Variation of the MIP muons energy deposition between 1-2 m from their track end is about 1.8\% \cite{AbsCal_technote_1stAna.pdf}.

%\cite{AbsCal_technote_1stAna.pdf} Sources of systematic uncertainty of particular concern are those introduced by residual variations remaining after calibration. Systematic errors are introduced by spatial and temporal variations in detector response. Further, any difference between the two detectors may introduce a relative shift in the energy scale between the detectors. Track end misreconstruction: For a track window starting at 100 cm from the track end, a conservative mis-reconstruction of the track end point by 10cm will shift the start of the track window to between 90cm and 110cm. This shift will alter the MEU value by less than 0.4\% over the range. Variations in space and time: If the calibration procedure was ideal the detector response would not vary with position in either data or MC. The calibration is not ideal and the detector response and recorded simulated energy deposition varies with position of the hit within the detector, such variations will introduce systematic errors. The position of a hit can be defined by the plane, cell within the plane, and distance along the
%\chapter{Measuring the Muon Neutrino Magnetic Moment}\label{sec:NeutrinoMagMoment}

\todo{Also check out NeutrinoMassesPheno2007.pdf, sec 6.4}

%%% ABSTRACT %%%
\todo{Write an introduction to the NuMM}
"In the standard model, neutrinos have small charge radii induced by radiative corrections. The predicted values of the electron and muon neutrino charge radii are less than an order of magnitude smaller than the current experimental upper limits and can be tested in the next generation of accelerator and reactor experiments through the observation of neutrino-electron elastic scattering and CEvNS. Precision measurements of the neutrino charge radii would either be an important confirmation of the standard model, or would discover new physics. The same types of experimental measurements are also sensitive to more exotic neutrino electromagnetic properties: magnetic moments and millicharges, which would be certainly due to new BSM physics. The discovery of millicharges or anomalously large neutrino magnetic moments would have also important implications for astrophysics and cosmology."\cite{SnowmassNeutrinoFrontierReport.pdf}

%[SNOWMASSLOI_NuMMAtNuMuBeams.pdf] Extensions to the Standard Model predict neutrino magnetic moments[1-3], regardless of whether neutrinos are Dirac or Majorana particles. .. Such an unambiguous excess could be interpreted, for example, as evidence[3] for the Majorana nature of the neutrino.

\section{Theory of neutrino magnetic moment}
%Neutrino electromagnetic properties have been proposed since the very beginning by Pauli to solve the discrepancies in the electron beta emission spectra. This was solved by discovering the neutron. Then again, neutrino magnetic moment was proposed as one of the solution to the solar neutrino problem 

% Although in the standard model neutrinos are electrically neutral and do not possess electric or magnetic dipole moments, they have a charge radius which is generated by radiative corrections. [...] In many extensions of the standard model neutrinos also acquire electromagnetic properties through quantum loop effects which allow direct interactions of neutrinos with electromagnetic fields and electromagnetic interactions of neutrinos with charged particles. Hence, the theoretical and experimental study of neutrino electromagnetic interactions is a powerful tool in the search for the fundamental theory beyond the standard model. Moreover, the electromagnetic interactions of neutrinos can generate important effects, especially in astrophysical environments, where neutrinos propagate over long distances in magnetic fields in vacuum and in matter. [nuElmagInt2015.pdf]

% ...the existence of neutrino masses and mixing implies that neutrinos have magnetic moments. Since their values depend on the specific theory which extends the standard model in order to accommodate neutrino masses and mixing, experimentalists and theorists are eagerly looking for them. [nuElmagInt2015.pdf]

%Systematic theoretical studies of neutrino electromagnetic properties started after it was shown that in the extended standard model with right-handed neutrinos the magnetic moment of a massive neutrino is, in general, nonvanishing and that its value is determined by the neutrino mass (Lee and Shrock, 1977; Marciano and Sanda, 1977; Petcov, 1977; Fujikawa and Shrock, 1980; Pal and Wolfenstein, 1982; Shrock, 1982; Bilenky and Petcov, 1987). [nuElmagInt2015.pdf]

%Neutrino electromagnetic properties are important because they are directly connected to fundamentals of particle physics. For example, neutrino electromagnetic properties can be used to distinguish Dirac and Majorana neutrinos, because Dirac neutrinos can have both diagonal and off-diagonal magnetic and electric dipole moments, whereas only the off-diagonal ones are allowed for Majorana neutrinos (Schechter and Valle, 1981; Kayser, 1982, 1984; Nieves, 1982; Pal and Wolfenstein, 1982; Shrock, 1982). This is shown in detail in Secs. III.A and III.B. Another important case in which Dirac and Majorana neutrinos have quite different observable effects is the spin-flavor precession in an external magnetic field discussed in Sec. VI.B. Neutrino electromagnetic properties are also probes of new physics beyond the standard model, because in the standard model neutrinos can have only a charge radius (see Secs. III.C and VII.B). The discovery of other neutrino electromagnetic properties would be a signal of new physics beyond the standard model (Bell et al., 2005, 2006; Bell, 2007; Novales-Sanchez et al., 2008). [nuElmagInt2015.pdf]

As was describe in Sec.~\ref{sec:NeutrinoTheory}, neutrinos in the \gls{SM} are massless and electrically neutral particles. However, even \gls{SM} neutrinos can have electromagnetic interaction through loop diagrams involving charged leptons and the W boson. These interactions are described by the neutrino charge radius, described in section \ref{sec:otherNuElmagProperties} \todo{Re-write this since I'm not going to include the other elmag properties section} \cite{SnowmassNeutrinoFrontierReport.pdf}.

%But this is only the neutrino charge radius, not the neutrino electric or magnetic moment (maybe also the anapole moment) "Hence, in the standard model the form factor can be interpreted as a neutrino charge radius or as an anapole moment (or as a combination of both). The standard model theory of the neutrino charge radius has a long history, with some controversies which are shortly summarized in the following." [nuElmagInt2015.pdf - sec.VIIB]

%Various theories beyond the Standard Model
In general \gls{BSM} theories, considering interactions with a single photon as shown on Fig.~\ref{fig:FeynmanNuElmagDiagram}, neutrino electromagnetic interactions can be described by an \textit{effective} interaction Hamiltonian \cite{nuElmagInt2015.pdf}
\begin{equation}
\mathcal{H}^{\left(\nu\right)}_{em}\left(x\right)=\sum^N_{k,j=1}\overline{\nu}_k\left(x\right)\Lambda^{kj}_{\mu}\nu_j\left(x\right)A^{\mu}\left(x\right).
\end{equation}
Here $\nu_k\left(x\right), k = 1,...,N,$ are neutrino fields in the mass basis with $N$ neutrino mass states and $x$ denotes the position. $\Lambda^{kj}_{\mu}$ is a general vertex function and $A^{\mu}\left(x\right)$ is the electromagnetic field.

\begin{figure}[hbtp]
\centering
\includegraphics[width=0.4\linewidth]{Plots/NuMM/FeynmanDiagramNuElmagInt.png}
\caption{Effective coupling of neutrinos with one photon electromagnetic field.}
\label{fig:FeynmanNuElmagDiagram}
\end{figure}

\iffalse
The amplitude of neutrino-to-neutrino interaction for \textbf{Dirac} neutrinos is
\begin{equation}
\braket{\nu_f\left(p_f\right)|j^{\left(\nu\right)}_{\mu}\left(x\right)|\nu_i\left(p_i\right)}=
e^{i\left(p_f-p_i\right)x}\overline{u}_f\left(p_f\right)\Lambda^{fi}_{\mu}\left(p_f,p_i\right)u_i\left(p_i\right),
\end{equation}
where $p_f$ and $p_i$ are the final and initial four momentums respectively and $u/\overline{u}$ are the solutions to the Dirac equation for a free particle. We take into account possible transitions between different mass states $\nu_i$ and $\nu_f$ \cite{nuElmagInt2015.pdf}. \todo{also describe what is j}
\fi

The vertex function $\Lambda^{fi}_{\mu}\left(q\right)$ is generally a matrix and in the most general case consistent with the \gls{SM} gauge invariance \cite{MostGeneralNuElmagVectorFunctionExpressionKayser.pdf, MostGeneralNuElmagVectorFunctionExpressionNieves.pdf} can be written in terms of linearly independent products of Dirac matrices $\left(\gamma\right)$ and only depends on the four momentum of the photon $\left(q=p_f-p_i\right)$:
\begin{align}
\Lambda^{fi}_{\mu}\left(q\right)=&
\mathbb{F}^{fi}_1\left(q^2\right)q_{\mu}+
\mathbb{F}^{fi}_2\left(q^2\right)q_{\mu}\gamma_5+
\mathbb{F}^{fi}_3\left(q^2\right)\gamma_{\mu}+
\mathbb{F}^{fi}_4\left(q^2\right)\gamma_{\mu}\gamma_5+\notag\\ &
\mathbb{F}^{fi}_5\left(q^2\right)\sigma_{\mu\nu}q^{\nu}+
\mathbb{F}^{fi}_6\left(q^2\right)\epsilon_{\mu\nu\rho\gamma}q^{\nu}\sigma^{\rho\gamma},
\end{align}
where $\mathbb{F}^{fi}_i\left(q^2\right)$ are six Lorentz invariant form factors and $\delta$ and $\epsilon$ are the Dirac delta and the Levi-Civita symbols respectively.

Applying conditions of hermiticity $\left(\mathcal{H}^{\left(\nu\right)\dagger}_{em}=\mathcal{H}^{\left(\nu\right)}_{em}\right)$ and of the gauge invariance of the electromagnetic field, the vertex function can be rewritten as
\begin{equation}
\Lambda^{fi}_{\mu}\left(q\right)=
\left(\gamma_{\mu}-q_{\mu}\slashed{q}/q^2\right)\left[
\mathbb{F}^{fi}_{Q}\left(q^2\right)+\mathbb{F}^{fi}_{A}\left(q^2\right)q^2\gamma_5\right]-
i\sigma_{\mu\nu}q^{\nu}\left[\mathbb{F}^{fi}_{M}\left(q^2\right)+i\mathbb{F}^{fi}_{E}\left(q^2\right)\gamma_5\right],
\end{equation}
where $\mathbb{F}^{fi}_Q,\mathbb{F}^{fi}_M,\mathbb{F}^{fi}_E$ and $\mathbb{F}^{fi}_A$ are hermitian matrices representing the charge, dipole magnetic, dipole electric and anapole neutrino form factors respectively. It is clear that the vertex function only depends on the square of the four momentum of the photon $q^2$. In coupling with a real photon $\left(q^2=0\right)$ these form factors become the neutrino charge and magnetic, electric and anapole moments. The neutrino charge radius corresponds to the second term in the expansion of the charge form factor \cite{nuElmagInt2015.pdf}.

The above expression can be simplified as \cite{NeutrinoPropertiesSnowmass2022.pdf}
\begin{equation}
\Lambda^{fi}_{\mu}\left(q\right)=\gamma_{\mu}\left(Q_{\nu_{fi}}+\frac{q^2}{6}\langle r^2\rangle_{\nu_{fi}}\right)-i\sigma_{\mu\nu}q^{\nu}\mu_{\nu_{fi}},
\end{equation}
where $Q_{\nu_{fi}}$, $\langle r^2\rangle_{\nu_{fi}}$, and $\mu_{\nu_{fi}}$ are the neutrino charge, effective charge radius (also containing anapole moment), and an effective magnetic moment (also containing electric moment) respectively. This is possible thanks to the similar effect of the neutrino charge radius and the anapole moment, or of the neutrino magnetic and electric moment respectively \cite{nuElmagInt2015.pdf}. These are the three neutrino electromagnetic properties (charge, charge radius and magnetic moment) measured in the experiments.

\todo{Add a note briefly describing the other elmag properties and mentioning that they could be measured as well, but not describe here. Maybe refer reader to the theoretical overview paper}

\iffalse
For antineutrinos the form factors are transformed as:
\begin{equation}\label{eqAnu1}
\overline{\mathbb{F}}^{fi}_{\Omega}=-\mathbb{F}^{if}_{\Omega}=-\left(\mathbb{F}^{fi}_{\Omega}\right)^{\star} \ \ \ \Omega=Q,M,E,
\end{equation}
\begin{equation}\label{eqAnu2}
\overline{\mathbb{F}}^{fi}_{A}=\mathbb{F}^{if}_{A}=\left(\mathbb{F}^{fi}_{A}\right)^{\star}.
\end{equation}
\todo{maybe describe what does this mean?}

In case of \textbf{Majorana neutrinos}, the general expression for the vertex function in terms of charge, magnetic, electric and anapole form factors looks the same as for Dirac neutrinos.
\todo{so does that mean that the interaction amplitude can be written in the same way for both Dirac and Majorana neutrinos?} However, since Majorana antineutrinos are the same particle as Majorana neutrinos, from eq.\ref{eqAnu1},\ref{eqAnu2} we can see that:
\begin{equation}\label{eqAntisymmetryCondition}
\mathbb{F}^M_{\Omega}=-\left(\mathbb{F}^M_{\Omega}\right)^T \ \ \ \Omega=Q,M,E,
\end{equation}
\begin{equation}
\mathbb{F}^M_{A}=\left(\mathbb{F}^M_A\right)^T.
\end{equation}
Therefore the Majorana charge, magnetic and electric form factor matrices are antisymmetric and the anapole form factor matrix is symmetric. This means that Majorana neutrino doesn't have any diagonal charge and dipole magnetic and electric moments, but it can have transition  charge and magnetic and electric moment \cite{nuElmagInt2015.pdf}.
\todo{Explain why is this worth mentioning or remove it if it's not}
\fi

%%%%%%%%%%%%%%%%%%%%%%%%%%%%%%%%%%%%%%%%
\subsection{Neutrino electric and magnetic dipole moments}
The size and effect of neutrino electromagnetic properties depend on the specific \gls{BSM} theory. Evaluating the one loop diagrams in the minimally extended \gls{SM} with three right-handed Dirac neutrinos as described in Sec.~\ref{sec:NuMass} gives the first approximation of the electric and magnetic moments:
\begin{equation}\label{eq:DiracMagMomExpression}
\begin{rcases}
\mu^D_{kj}\\
i\epsilon^D_{kj}
\end{rcases}
\simeq\frac{3eG_F}{16\sqrt{2}\pi^2}\left(m_k\pm m_j\right)\left(\delta_{kj}-\frac{1}{2}\sum_{l=e,\mu ,\tau}U^{\star}_{lk}U_{lj}\frac{m_l^2}{m_W^2}\right),
\end{equation}
where $m_k,m_j$ are the neutrino masses and $m_l$ are the masses of charged leptons which appear in the loop diagrams \cite{nuElmagInt2015.pdf}. Also, $D$ superscript denotes Dirac neutrinos, $e$ is the electron charge, $G_F$ is the Fermi coupling constant, and $U$ is the \gls{PMNS} neutrino oscillation matrix. Higher order electromagnetic corrections were neglected, but can also have a significant contribution, depending on the theory.

It can be seen that Dirac neutrinos have no diagonal electric moments $\left(\epsilon_{kk}^D=0\right)$ and their diagonal magnetic moments are approximately
\begin{equation}\label{eq:DiagMagMomVal}
\mu_{kk}^D\simeq\frac{3eG_Fm_k}{8\sqrt{2}\pi^2}\simeq 3.2\times 10^{-19}\left(\frac{m_k}{\textsf{eV}}\right)\mu_B,
\end{equation}
where $\mu_B$ is the Bohr magneton which represents the value of the electron magnetic moment \cite{nuElmagInt2015.pdf}. Neutrino magnetic moments are therefore strongly suppressed by the smallness of neutrino masses, with theoretical predictions in Eq.~\ref{eq:DiagMagMomVal} several orders of magnitude below the reach of current experiments \cite{NeutrinoPropertiesSnowmass2022.pdf}.

The transition magnetic moments from Eq.~\ref{eq:DiracMagMomExpression} are suppressed with respect to the largest of the diagonal magnetic moments by at least a factor of $10^{-4}$ due to the $m_W^2$ in the denominator. The transition electric moments are even smaller due to the mass difference in Eq.~\ref{eq:DiracMagMomExpression}. Therefore an experimental observation of a magnetic moment larger than in Eq.~\ref{eq:DiagMagMomVal} would indicate physics beyond the minimally extended \gls{SM} \cite{nuElmagInt2015.pdf,nuMMMajoranaBounds2006.pdf}.

\todo{Actually write why these values are different for Majorana neutrinos than for Dirac neutrinos}
Majorana neutrinos in a minimal extension can be obtained by either adding a $\textsf{SU}\left(2\right)_L$ Higgs triplet, or right handed neutrinos together with a $\textsf{SU}\left(2\right)_L$ Higgs singlet \cite{nuElmagInt2015.pdf}. If we neglect the Feynman diagrams which depend on the model of the scalar sector, the magnetic and electric dipole moments are
\begin{equation}
\mu_{kj}^M\simeq -\frac{3ieG_F}{16\sqrt{2}\pi^2}\left(m_k+m_j\right)\sum_{l=e,\mu ,\tau}\operatorname{Im}\left[U^{\star}_{lk}U_{lj}\right]\frac{m_l^2}{m_W^2},
\end{equation}
\begin{equation}
\epsilon_{kj}^M\simeq \frac{3ieG_F}{16\sqrt{2}\pi^2}\left(m_k-m_j\right)\sum_{l=e,\mu ,\tau}\operatorname{Re}\left[U^{\star}_{lk}U_{lj}\right]\frac{m_l^2}{m_W^2}.
\end{equation}
These are difficult to compare to the Dirac case, due to possible presence of Majorana phases in the \gls{PMNS} matrices, but it is clear that they have the same order of magnitude as Dirac transition dipole moments. However, the neglected model dependent contributions can enhance the transition dipole moments \cite{nuElmagInt2015.pdf}.

\todo{Re-read the natural upper bounds paper}
It is possible \cite{nuMMMajoranaBounds2006.pdf} to obtain a `natural' upper limits on the size of the neutrino magnetic moment by calculating its contribution to the neutrino mass by standard model radiative corrections. \todo{I don't think this is clear enough, how is this done} For Dirac neutrinos, the radiative correction induced by neutrino magnetic moment, generated at an energy scale $\Lambda_{NP}$, to the neutrino mass is generically
\begin{equation}
m_{\nu}^D\sim\frac{\mu_{\nu}^D}{3\times 10^{-15}\mu_B}\left[\Lambda\left(\textsf{TeV}\right)\right]^2\textsf{eV}.
\end{equation}
So for $\Lambda_{NP}\simeq 1\textsf{TeV}$ and $m_{\nu}\lesssim 0.3\textsf{eV}$ the limit becomes $\mu_{\nu}^D\lesssim 10^{-15}\mu_B$. This applies only if \gls{NP} is well above the electroweak scale ($\Lambda_{EW} \sim 100\textsf{GeV}$) \todo{Finish this sentence}. However, there are theories that contain a Dirac neutrino magnetic moment higher than this limit, for example in frameworks of minimal super-symmetric standard model, by adding more Higgs doublets, or by considering large extra dimensions \todo{Add references to the specific theories?} \cite{nuElmagInt2015.pdf}.

Similar limit for Majorana neutrino magnetic moment would be less stringent than for Dirac neutrinos due to the antisymmetry of the Majorana neutrino magnetic moment form factors \todo{Probably explain here a bit more what does this mean}. Considering $m_{\nu}\lesssim 0.3\textsf{eV}$, the limit can be expressed as 
\begin{align}
\mu_{\tau\mu},\mu_{\tau e} &\lesssim 10^{-9}\left[\Lambda\left(\textsf{TeV}\right)\right]^{-2}\\
\mu_{\mu e} &\lesssim 3\times 10^{-7}\left[\Lambda\left(\textsf{TeV}\right)\right]^{-2}
\end{align}
which is shown in the flavour basis \todo{Explain here what is the flavour basis}, which relates to the framework used previously via the \gls{PMNS} matrix as
\begin{equation}
\mu_{ij}=\sum_{\alpha\beta}\mu_{\alpha\beta}U^{\star}_{\alpha i}U_{\beta j},\ \ \ \alpha,\beta\in\left\lbrace e,\mu,\tau\right\rbrace.
\end{equation}

\todo{Add a discussion about the triangular inequalities}

These considerations imply, that if a magnetic moment $\mu\gtrsim 10^{-15}\mu_B$ would be measured, it is more plausible that neutrinos are Majorana fermions and that the scale of lepton violation would be well below the conventional see-saw scale \cite{nuMMMajoranaBounds2006.pdf} \todo{double check this claim, also reword this sentence}.

\subsubsection{Effective neutrino magnetic moment}
Since experiments detect neutrino flavour states, not the mass states, what we measure is an effective `flavour' magnetic moment $\mu_{eff}$. $\mu_{eff}$ is influenced by mixing of the neutrino magnetic moments (and electric moments) expressed in the mass basis (as described above) and neutrino oscillations \todo{This basis relation was already partly described above, mention that and combine the descriptions}. In the ultra-relativistic limit, the neutrino effective magnetic moment is
\begin{equation}
\mu_{\nu_l}^2\left(L,E_{\nu}\right)=\sum_j\left|\sum_k U^{\star}_{lk}e^{\mp i\Delta m^2_{kj}L/2E_{\nu}}\left(\mu_{jk}-i\epsilon_{jk}\right)\right|^2,
\end{equation}
where the minus sign in the exponent is for neutrinos and the plus sign for antineutrinos \cite{nuElmagInt2015.pdf}. Therefore the only difference between the effective neutrino and antineutrinos magnetic moment is in the phase induced by neutrino oscillations.

For experiments with baselines short enough that neutrino oscillations would not have time to develop $\left(\Delta m^2L/2E_{\nu}\ll\sim1\right)$, such as the \gls{NOvA} \gls{ND}, the effective magnetic moment can be expressed as
\begin{equation}
\mu_{\nu_l}^2=\mu_{\overline{\nu}_l}^2\simeq\sum_j\left|\sum_k U_{lk}^{\star}\left(\mu_{jk}-i\epsilon_{jk}\right)\right|^2=\left[U\left(\mu^2+\epsilon^2\right)U^{\dagger}+2\operatorname{Im}\left(U\mu\epsilon U^{\dagger}\right)\right]_{ll^{\prime}},
\end{equation}
which is independent of the neutrino energy \todo{Figure out how does this relate to the mag moment cross section which does depend on the neutrino energy!}.

\todo{Consider if this paragraph is actually important}
Since the effective magnetic moment depends on the flavour of the studied neutrino, it is different (but related) for neutrino experiments studying neutrinos from different sources. Additionally some experiments, namely solar neutrino experiments, need to include matter effects on the neutrino oscillations. Therefore the reports on the value (or upper limit) of the effective neutrino magnetic moment are not directly comparable between different types of neutrino experiments. Theorists publish papers trying to extrapolate the measured effective magnetic moments to each neutrino flavour, but necessarily apply assumptions that might not hold in all \gls{BSM} theories.

\subsection{Other neutrino electromagnetic properties}\label{sec:otherNuElmagProperties}
\note{I am not going to report results on these, so should I even mention them here? Maybe it's enough to just mention that they exist in the intro section...}
\todo{This section is not finished, most of this text is just copied from some theory papers for now}

\todo{See also StatusAndPerspectiveOfNuMM2016.pdf}

Neutrino electric charge is heavily constraint by the measurements on the neutrality of matter (since generally neutrinos having an electric charge would also mean that neutrons have charge which would affect all heavier nuclei). It is also constrained by the SN1987A, since neutrino having an effective charge would lengthen its path through the extragalactic magnetic fields and would arrive on earth later. It can also be obtained from nu-on-e scatter from the relationship between neutrino millicharge and magnetic moment. [nuElmagInt2015.pdf - sec. VIIA] \todo{Make this description shorter, just a single sentence and combine with the charge radius}

The neutrino charge radius is determined by the second term in the expansion of the neutrino charge form factor and can be interpreted using the Fourier transform of a spherically symmetric charge distribution. It can also be negative since the charge density is not a positively defined quantity. In the SM the charge radius has the form of (possible other definitions exist)
\begin{equation}
\langle r_{\nu_l}^2\rangle_{\textsf{SM}}=\frac{G_{\textsf{F}}}{4\sqrt{2}\pi^2}\left[3-2\log\left(\frac{m_l^2}{m_W^2}\right)\right].
\end{equation}
This corresponds to $\langle r_{\nu_{\mu}}^2\rangle_{\textsf{SM}}=2.4\times 10^{-33}\ \unit{cm^2}$ and similar scale for other neutrino flavours. [nuElmagInt2015.pdf - sec. VIIB]

[nuElmagInt2015.pdf - sec. VIIB]
The effect of the neutrino charge radius on the neutrino-on-electron scattering cross section is through the following shift of the vector coupling constant (Grau and Grifols, 1986; Degrassi, Sirlin, and VMarciano, 1989; Vogel and Engel, 1989; Hagiwara et al., 1994):
\begin{equation}
g_V^{\nu_l}\rightarrow g_V^{\nu_l}+\frac{2}{3}m_W^2\langle r_{\nu_l}^2\rangle\sin^2\theta_W
\end{equation}

[nuElmagInt2015.pdf - sec. VIIB]
The current experimental limits for muon neutrinos are from \todo{check the current exp. limits}  Hirsch, Nardi, and Restrepo (2003) who obtained the
following 90\% C.L. bounds on $\langle r_{\nu_\mu}^2\rangle$ from a reanalysis of
CHARM-II (Vilain et al., 1995) and CCFR (McFarland et al.,1998) data:
\begin{equation}
-0.52\times 10^{-32}<\langle r_{\nu_\mu}^2\rangle<0.68\times 10^{-32}\ \unit{cm^2}
\end{equation}

In the Standard Model, the neutrino anapole moment is somehow coupled with the neutrino charge radii and is functionally identical. the phenomenology of neutrino anapole moments is similar to that of neutrino charge radii. Hence, the limits on the neutrino charge radii discussed in Sec. VII.B also apply to the neutrino anapole moments multiplied by 6.  in the standard model the neutrino charge radius and the anapole moment are not defined separately and one can interpret arbitrarily the charge form factor as a charge radius or as an anapole moment. Therefore, the standard model values for the neutrino charge radii in Eqs. (7.35)–(7.38) can be interpreted also as values of the corresponding neutrino anapole moments. [nuElmagInt2015.pdf - sec. VIIC]

It is possible to consider  the toroidal dipole moment as a characteristic of the neutrino which is more convenient and transparent than the anapole moment for the description of T-invariant interactions with nonconservation of the P and C symmetries. the toroidal and anapole moments coincide in the static limit when the masses of the initial and final neutrino states are equal to each other. The toroidal (anapole) interactions of a Majorana as well as a Dirac neutrino are expected to contribute to the total cross section of neutrino elastic scattering off electrons, quarks, and nuclei. Because of the fact that the toroidal (anapole) interactions contribute to the helicity preserving part of the scattering of neutrinos on electrons, quarks, and nuclei, its contributions to cross sections are similar to those of the neutrino charge radius. In principle, these contributions can be probed and information about toroidal moments can be extracted in low-energy scattering experiments in the future. Different effects of the neutrino toroidal moment are discussed by Ginzburg and Tsytovich (1985), Bukina, Dubovik, and Kuznetsov (1998a, 1998b), and Dubovik and Kuznetsov (1998). In particular, it has been shown that the neutrino toroidal electromagnetic interactions can produce Cherenkov radiation of neutrinos propagating in a medium. [nuElmagInt2015.pdf - sec. VIIC]

%%%%%%%%%%%%%%%%%%%%%%%%%%%%%%%%%%%%%%%%%%%%%%%%

\subsection{Measuring neutrino magnetic moment}\label{sec:MeasuringNuMM}
The most sensitive method to measure neutrino magnetic moment is the low energy elastic scattering of (anti)neutrinos on electrons \cite{nuElmagInt2015.pdf}. The diagram for this interaction is shown in Fig.~\ref{fig:NuoneDiagram} displaying the two observables, the recoil electron's kinetic energy $\left(T_e=E_{e\prime}-m_e\right)$ and the recoil angle with respect to the incoming neutrino beam $\left(\theta\right)$.
\begin{figure}[hbtp]
\centering
\includegraphics[width=0.55\linewidth]{Plots/NuMM/NuoneInteraction.png}
\caption{Neutrino-on-electron elastic scattering diagram}
\label{fig:NuoneDiagram}
\end{figure}

\note{Is this derivation too trivial to mention in a thesis? Should I just mention the results? I wanted to have this in the technote, but probably too detailed for a thesis...}
\todo{Also change all we to passive voice - or should I keep we here?}
From simple $2\rightarrow 2$ kinematics we can calculate
\begin{equation}
\left(P_{\nu}-P_{e^{\prime}}\right)^2=\left(P_{\nu^{\prime}}-P_e\right)^2,
\end{equation}
\begin{equation}
m_{\nu}^2+m_e^2-2E_{\nu}E_{e^{\prime}}+2E_{\nu}p_{e^{\prime}}\cos\theta=m_{\nu}^2+m_e^2-2E_{\nu^{\prime}}m_e.
\end{equation}
Using the energy conservation
\begin{equation}
E_{\nu}+m_e=E_{\nu^{\prime}}+E_{e^{\prime}}=E_{\nu^{\prime}}+T_e+m_e\Rightarrow E_{\nu^{\prime}}=E_{\nu}-T_e
\end{equation}
we get
\begin{equation}
E_{\nu}p_{e^{\prime}}\cos\theta=E_{\nu}E_{e^{\prime}}-E_{\nu^{\prime}}m_e=E_{\nu}\left(T_e+m_e\right)-\left(E_{\nu}-T_e\right)m_e=T_e\left(E_{\nu}+m_e\right),
\end{equation}
\begin{equation}
\cos\theta=\frac{E_{\nu}+m_e}{E_{\nu}}\sqrt{\frac{T_e^2}{E_{e^{\prime}}^2-m_e^2}}=\frac{E_{\nu}+m_e}{E_{\nu}}\sqrt{\frac{T_e^2}{T_e^2+2T_em_e}}.
\end{equation}
And finally we get
\begin{equation}\label{eq:ThetaTRelation}
\cos\theta=\frac{E_{\nu}+m_e}{E_{\nu}}\sqrt{\frac{T_e}{T_e+2m_e}}.
\end{equation}

We can rearrange the Eq.~\ref{eq:ThetaTRelation} to get
\begin{equation}\label{eq:TThetaRelation}
T_e=\frac{2m_eE_\nu^2\cos^2\theta}{\left(E_\nu+m_e\right)^2-E_\nu^2\cos^2\theta}.
\end{equation}
Electron's kinetic energy is therefore kinematically constrained by the energy conservation as
\begin{equation}
T_e\leq\frac{2E_{\nu}^2}{2E_{\nu}+m_e},
\end{equation}
which corresponds to the $\cos\theta\rightarrow 1$ when the recoil electron goes exactly forward in the incident neutrino direction.

Considering $E_{\nu}\sim\textsf{GeV}$, we can approximate $\frac{m_e^2}{E_{\nu}^2}\rightarrow 0$ and from Fig.\ref{fig:TThetaDistribution} we can see that we can approximate all recoil angles to be very small, therefore $\theta^2\cong \left(1-\cos^2\theta\right)$. Using Eq.\ref{eq:ThetaTRelation} we get
\begin{equation}
T_e\theta^2\cong T_e\left(1-\left(\frac{E_\nu+m_e}{E_\nu}\right)^2\frac{T_e}{T_e+2m_e}\right)
=T_e\left(1-\left(1+\frac{2m_e}{E_\nu}\right)\frac{T_e}{T_e+2m_e}\right),
\end{equation}
therefore
\begin{equation}
T_e\theta^2\cong \frac{2m_eT_e}{T_e+2m_e}\left(1-\frac{T_e}{E_\nu}\right)=2m_e\left(\frac{1}{1+\frac{2m_e}{T_e}}\right)\left(1-\frac{T_e}{E_\nu}\right),
\end{equation}
and finally
\begin{equation}\label{eqTThetaSqExp}
T_e\theta^2\cong 2m_e\left(1-\frac{T_e}{E_{\nu}}\right)<2m_e.
\end{equation}

This is a strong limit that clearly distinguishes the \gls{nuone} elastic scattering events from other similar interaction involving single electron (mainly the $\nu_e$\gls{CC} interaction).

\begin{figure}[hbtp]
\centering
\includegraphics[width=.7\linewidth]{Plots/NuMM/KinematicsTOnTh.jpeg}
\caption[Electron recoil energy versus recoil angle]{Relation between the recoil electron's kinetic energy and angle for \acrshort{nuone} elastic scattering. The coverage of the \acrshort{NOvA} detectors for measuring the electron recoil energy is shown in blue. Only very forwards electron's are recorded in \acrshort{NOvA}.}
\label{fig:TThetaDistribution}
\end{figure}

\subsubsection{Neutrino magnetic moment cross section}
\note{Should this only be a subsubsection?}
In the ultra-relativistic limit, the neutrino magnetic moment changes the neutrino helicity, turning active neutrinos into sterile \todo{cite this properly}. Since the \gls{SM} weak interaction conserves helicity we can simply add the two contribution to the \gls{nuone} cross section incoherently \cite{nuElmagInt2015.pdf}:
\begin{equation}
\frac{d\sigma_{\nu_le^-}}{dT_e}=\left(\frac{d\sigma_{\nu_le^-}}{dT_e}\right)_{\textsf{SM}}+\left(\frac{d\sigma_{\nu_le^-}}{dT_e}\right)_{\textsf{MAG}}.
\end{equation}

The \gls{SM} contribution can be expressed as \cite{nuElmagInt2015.pdf}:
\begin{multline}
\left(\frac{d\sigma_{\nu_le^-}}{dT_e}\right)_{\textsf{SM}}=\frac{G_F^2m_e}{2\pi}\left\lbrace\left(g_V^{\nu_l}+g_A^{\nu_l}\right)^2+\left(g_V^{\nu_l}-g_A^{\nu_l}\right)^2\left(1-\frac{T_e}{E_{\nu}}\right)^2\right.\\
+\left.\left(\left(g_A^{\nu_l}\right)^2-\left(g_V^{\nu_l}\right)^2\right)\frac{m_eT_e}{E_{\nu}^2}\right\rbrace,
\end{multline}
where the coupling constants $g_V$ and $g_A$ are different for different neutrino flavours and for antineutrinos. Their values are:
\begin{align}
g_V^{\nu_e}&=2\sin^2\theta_W+1/2,\hspace{2.5cm} g_A^{\nu_e}=1/2,\\
g_V^{\nu_{\mu,\tau}}&=2\sin^2\theta_W-1/2,\hspace{2.25cm} g_A^{\nu_{\mu,\tau}}=-1/2.
\end{align}
For antineutrinos $g_A\rightarrow -g_A$.

\todo{Decide if this is actually useful or not}
Using Eq.~\ref{eq:TThetaRelation} it is possible to get the differential cross section for $\cos\theta$:
\begin{equation}
dT_e=\frac{4m_eE_\nu^2\left(m_e+E_\nu\right)^2}{\left[\left(m_e+E_\nu\right)^2-E_\nu^2\cos^2\theta\right]^2}\cos\theta d\cos\theta
\end{equation}
as
\begin{multline}
\left(\frac{d\sigma_{\nu_le^-}}{d\cos\theta}\right)_{\textsf{SM}}=
\frac{2G_F^2E_{\nu}^2m_e^2\cos\theta\left(E_{\nu}+m_e\right)^2}{\pi\left(\left(E_{\nu}+m_e\right)^2-E_{\nu}^2\cos^2\theta\right)^2}\\
\left\lbrace\left(g_V^{\nu_l}+g_A^{\nu_l}\right)^2+
\left(g_V^{\nu_l}-g_A^{\nu_l}\right)^2\left(1-\frac{2m_eE_{\nu}\cos^2\theta}{\left(E_{\nu}+m_e\right)^2-E_{\nu}^2\cos^2\theta}\right)^2\right.+\\
\left.\left(\left(g_A^{\nu_l}\right)^2-\left(g_V^{\nu_l}\right)^2\right)
\frac{2m_e^2\cos^2\theta}{\left(\left(E_{\nu}+m_e\right)^2-E_{\nu}^2\cos^2\theta\right)}\right\rbrace,
\end{multline}

\begin{table}{ht}
\centering
\caption{Neutrino-on-electron elastic scattering total cross sections. \todo{Move units to title and add cross sections with thresholds. Also reference this somewhere in text} from Fundamentals of neutrino Physics and Astrophysics, p.139}
\begin{tabular}{cc}
\hline
Process & Total cross section\\\hline
$\nu_e+e^-$ & $\simeq 93\times 10^{-43} E_\nu\unit{cm^2 GeV^{-1}}$\\
$\overline{\nu}_e+e^-$ & $\simeq \unit[39\times 10^{-43} E_\nu]{cm^2 GeV^{-1}}$\\
$\nu_{\mu,\tau}+e^-$ & $\simeq \unit[15\times 10^{-43} E_\nu]{cm^2 GeV^{-1}}$\\
$\overline{\nu}_{\mu,\tau}+e^-$ & $\simeq \unit[13\times 10^{-43} E_\nu]{cm^2 GeV^{-1}}$\\\hline
\end{tabular}
\end{table}

The neutrino magnetic moment contribution is \todo{include derivation from \cite{NeutrinoElmagFormFactors1989.pdf}} \cite{nuElmagInt2015.pdf}:
\begin{equation}
\left(\frac{d\sigma_{\nu_le^-}}{dT_e}\right)_{\textsf{MAG}}=\frac{\pi\alpha^2}{m_e^2}\left(\frac{1}{T_e}-\frac{1}{E_{\nu}}\right)\left(\frac{\mu_{\nu_l}}{\mu_B}\right)^2,
\end{equation}
where $\alpha$ is the fine structure constant \todo{Calculate the total mag moment cross sections}.

Comparison of the \gls{SM} and the neutrino magnetic moment cross sections is shown on Fig.\ref{fig:NuMMCrossSectionComparison}. Whereas the \gls{SM} cross section is flat with $T_e\rightarrow 0$, the neutrino magnetic moment cross section keeps increasing to infinity. However, this reach is limited by the experimental capabilities of detecting such low energetic neutrinos. Possible \gls{NOvA} coverage is shown in a shaded blue and it is uncertain we could actually reach as low as $100\ \unit{MeV}$ \todo{Change this claims a little bit}.

\begin{figure}[hbtp]
\centering
\includegraphics[width=.9\textwidth]{Plots/NuMM/dSdTNumuMMCompAltLim.pdf}
\caption[Comparison of the neutrino magnetic moment and Standard Model cross sections]{Comparison of the neutrino magnetic moment (coloured) and the \acrshort{SM} (black) cross sections for the \acrshort{nuone} elastic scattering. Different colours depict different values of the neutrino magnetic moment. Dashed lines are the individual cross sections and dotted lines are the added total cross section with the standard model contribution. \acrshort{NOvA} coverage of electron recoil energies is shown in shaded blue \todo{Reference the colours on the figures to the origins of the values (LSND and Biao)}.}
\label{fig:NuMMCrossSectionComparison}
\end{figure}

As can be seen in Fig.~\ref{fig:NuMMCrossSectionComparison} and Fig.~\ref{fig:NuMMCrossSectionRatios}, the magnetic moment contribution exceeds the \gls{SM} contribution for low enough $T_e$. This can be approximated as \cite{nuElmagInt2015.pdf}:
\begin{equation}
T_e\lesssim\frac{\pi^2\alpha^2}{G_F^2m_e^3}\left(\frac{\mu_{\nu}}{\mu_B}\right)^2\simeq 2.9\times 10^{19}\left(\frac{\mu_{\nu}}{\mu_B}\right)^2\left[\textsf{MeV}\right],
\end{equation}
which does not depend on the neutrino energy and makes experiments sensitive to lower energetic electrons more sensitive to the neutrino magnetic moment. This is especially true for the recent dark matter experiments which put stringent limits on the solar neutrino effective magnetic moment, as described in the following section.

\begin{figure}[hbtp]
\centering
\includegraphics[width=.9\textwidth]{Plots/NuMM/RatioNumuMMCompLinX.pdf}
\caption[Ratio of the neutrino magnetic moment and Standard Model cross sections]{Ratio of the neutrino magnetic moment cross section to the \acrshort{SM} cross section for the \acrshort{nuone} elastic scattering. Different colours depict different effective muon neutrino magnetic moment values.}
\label{fig:NuMMCrossSectionRatios}
\end{figure}

%%% End of theoretical overview
%%%%%%%%%%%%%%%%%%%%%%%%%%%%%%%%%%%%%%%%%%%%%%%%%%%%%%%%%%%%%%%%%%%%%%%%%%%%%%%

\iffalse
\section{Experimental overview}
\todo{Should I include cosmological implication here?}

\section{Event selection}

\section{Fitting and hypothesis testing, parameter estimation}
How do we find the value of or limit for the effective neutrino magnetic moment?

Large section on statistics in the PDG.

Maximum likelihood with binned data:

N bins with a vector of data $n=\left(n1,...,n_N \right)$ with expectation values $\mu=E\left[n\right]$ and probabilities $f\left(n;\mu\right)$. Suppose the mean values $\mu$ can be determined as a function of a set of parameters $\theta$ (I assume for us there's either only one parameter - magnetic moment, or three parameters - mag. moment, scale of SM signal and scale of SM background). Then one may maximize the likelihood function based on the contents of the bins.

If the $n_i$ is regarded as independent and Poisson distributed (which I'd say is the case for us), then the data are instead described by a product of Poisson probabilities,
\begin{equation}
f_p\left(n;\theta\right)=\prod_{i=1}^{N} \frac{\mu_i^{n_i}}{n_i!}e^{-\mu_i},
\end{equation}
where the mean values $\mu_i$ are given functions of $\theta$. The total number of events $n_{tot}$ thus follows a Poisson distribution with mean $\mu_{tot}=\sum_i \mu_i$.

When using maximum likelihood with binned data, one can find the maximum likelihood estimators and at the same time obtain a statistic usable for a test of goodness-of-fit. Maximizing the likelihood $L\left(\theta\right)=f_P\left(n;\theta\right)$ is equivalent to maximizing the likelihood ratio $\lambda\left(\theta\right)=f_P\left(n;\theta\right) / f\left(n;\hat{\mu}\right)$, where in the denominator $f\left(n;\hat{\mu}\right)$ is a model with an adjustable parameter for each bin, $\mu=\left(\mu_1,...,\mu_N\right)$, and the corresponding estimators are $\hat{\mu}=\left(n_1,...,n_N\right)$ (called the `saturated model').

Equivalently one often minimizes the quantity $-2\ln\lambda\left(\theta\right)$. For independent Poisson distributed $n_i$ this is
\begin{equation}
-2\ln\lambda\left(\theta\right)=2\sum_{i=1}^{N}\left[\mu_i\left(\theta\right)-n_i+n_i\ln\frac{n_i}{\mu_i\left(\theta\right)}\right],
\end{equation}
where for bins with $n_i=0$, the last term is zero. In our term $\mu_i\left(\theta\right)$ is the \textbf{expected number of events in bin i if magnetic moment is $\theta$} and $n_i$ is the observed (measured) number of events in that bin.

A smaller value of $-2\ln\lambda\left(\hat{\theta}\right)$ corresponds to better agreement between the data and the hypothesized form of $\mu\left(\theta\right)$. The value of $-2\ln\lambda\left(\hat{\theta}\right)$ can thus be translated into a \textbf{p-value as a measure of goodness-of-fit}. Assuming the model is correct, then according to \textbf{Wilk's theorem}, for \textbf{sufficiently large} $\mu_i$ and provided certain regularity conditions are met, \textbf{the minimum of $-2\ln\lambda$ follows a $\chi^2$ distribution.} If there are N bins and M fitter parameters, then the number of degrees of freedom for the $\chi^2$ distribution is $N-M$ if the data are threated as Poisson distributed - which they are for us.

The method of least squares coincides with the method of maximum likelihood in a special case where the independent variables are Gaussian distributed - so I suppose this means that if I have enough events in each single bin, then I could equate the method of log likelihood and the method of least squares...

\subsection{Nuisance parameters}
In general the model is not perfect, which is to say it cannot provide an accurate description of the data even at the most optimal point of its parameter space. As a result, the estimated parameters can have a systematic bias. One can improve the model by including in it additional parameters. That is, $P\left(x|\theta\right)$ is replaced by a more general model $P\left(x|\theta,\nu\right)$, which depends on parameters of interest $\theta$ and \textit{nuisance parameters} $\nu$. The additional parameters are not of intrinsic interest but must be included for the model to be sufficiently accurate for some point in the enlarged parameter space.

Although including additional parameters may eliminate or at least reduce the effect of systematic uncertainties, their presence will result in increased statistical uncertainties for the parameters of interest. This occurs because the estimators for the nuisance parameters and those of interest will in general be correlated, which results in an enlargement of the contour.

To reduce the impact of the nuisance parameters one often tries to constrain their values by means of control or calibration measurements, say, having data \textbf{y} (I assume for us this would represent a control sample - like they use in the ND group). For example, some components of y could represent estimates of the nuisance parameters, often from separate experiments. Suppose the measurements y are statistically independent from x and are described by a model $P\left(y|\nu\right)$. The joint model for both x and y is in this case therefore the product of the probabilities for x
and y, and thus the likelihood function for the full set of parameters is
\begin{equation}
L\left(\theta,\nu\right)=P\left(x|\theta,\nu\right)P\left(y|\nu\right).
\end{equation}
Note that in this case if one wants to simulate the experiment by means of Monte Carlo, both the primary and control measurements, x and y, must be generated for each repetition under assumption of fixed values for the parameters $\theta$ and $\nu$.

Using all of the parameters $\left(\theta,\nu\right)$  to find the statistical errors in the parameters of interest $\theta$ is equivalent to using the \textit{profile likelihood}, which depends only on $\theta$. It is defined as
\begin{equation}
L_p\left(\theta\right)=L\left(\theta,\hat{\nu}\left(\theta\right)\right),
\end{equation}
This equation is supposed to have double hat for the neutrino on RHS but that throws an error when compiling...
%L\left(\theta,\hat{\hat{\nu}}\left(\theta\right)\right),
where the double-hat notation indicates the profiled values of the parameters $\nu$, defined as values that maximize $L$ for the specified $\theta$.

\subsection{Unbinned parameter estimation}
If the total number of data values is small, the unbinned maximum likelihood method is preferred, since binning can only result in a loss of information, and hence the larger statistical errors for the parameter estimates.
Does't this mean that if the number of events for the neutrino magnetic moment analysis is small, it would be better to do a completely unbinned maximum likelihood method, instead of a single bin method?


\subsection{Discussion}

For the energy deposition of electron in LArTPC might be a good source this https://iopscience.iop.org/article/10.1088/1748-0221/15/03/P03022/pdf

\subsubsection{Subsubsection}
\fi
%\chapter{Conclusion}\label{sec:Conclusion}

Should be mirroring the abstract more than the introduction. this is the first first people read (after abstract)

Explicitly say what I have done in a very short summary

The conclusion of the Test Beam calibration is in ... The Test Beam detector was successfully calibrated and is used in the analyses...

The measurement of the effective muon neutrino magnetic moment is presented here... And the final number is . The POT

Also mention further work referring to summaries and such. Just summarise everything.

%\include{epilog}

\printglossaries

%%% Bibliography
\include{bibliography}

%%% Attachments to the thesis, if any. Each attachment must be
%%% referred to at least once from the text of the thesis. Attachments
%%% are numbered.
%%%
%%% The printed version should preferably contain attachments, which can be
%%% read (additional tables and charts, supplementary text, examples of
%%% program output, etc.). The electronic version is more suited for attachments
%%% which will likely be used in an electronic form rather than read (program
%%% source code, data files, interactive charts, etc.). Electronic attachments
%%% should be uploaded to SIS and optionally also included in the thesis on a~CD/DVD.
%%% Allowed file formats are specified in provision of the rector no. 72/2017.
\appendix
%\chapter{Attachments}

\chapter{Test Beam Calibration Validation Plots}\label{sec:AppTBCalibValid}

\section{Distributions for Stopping Muons}

\begin{figure}[!ht]
  \begin{subfigure}{\textwidth}
  \centering
    \includegraphics[height=0.2\linewidth]{Plots/Calibana/legend.pdf}
  \end{subfigure}
  \vspace*{2mm}
  
  \begin{subfigure}{0.495\textwidth}
    \includegraphics[width=\linewidth]{Plots/Calibana/pecm_w_x.pdf}
  \end{subfigure}
  \begin{subfigure}{0.495\textwidth}
    \includegraphics[width=\linewidth]{Plots/Calibana/pecm_w_y.pdf}
  \end{subfigure}
  \begin{subfigure}{0.495\textwidth}
    \includegraphics[width=\linewidth]{Plots/Calibana/pecorrcm_w_x.pdf}
  \end{subfigure}
  \begin{subfigure}{0.495\textwidth}
    \includegraphics[width=\linewidth]{Plots/Calibana/pecorrcm_w_y.pdf}
  \end{subfigure}
    \begin{subfigure}{0.495\textwidth}
    \includegraphics[width=\linewidth]{Plots/Calibana/recomevcm_w_x.pdf}
  \end{subfigure}
  \begin{subfigure}{0.495\textwidth}
    \includegraphics[width=\linewidth]{Plots/Calibana/recomevcm_w_y.pdf}
  \end{subfigure}
  \caption{Distributions of stopping muons within a 1-2 m track window from the end of their tracks across the position within a cell.}
  %\label{fig:AbsCalibW1}
\end{figure}

\begin{figure}[!ht]
  \begin{subfigure}{\textwidth}
  \centering
    \includegraphics[height=0.2\linewidth]{Plots/Calibana/legend.pdf}
  \end{subfigure}
  \vspace*{2mm}

  \begin{subfigure}{0.495\textwidth}
    \includegraphics[width=\linewidth]{Plots/Calibana/pe_w_x.pdf}
  \end{subfigure}
  \begin{subfigure}{0.495\textwidth}
    \includegraphics[width=\linewidth]{Plots/Calibana/pe_w_y.pdf}
  \end{subfigure}
  \begin{subfigure}{0.495\textwidth}
    \includegraphics[width=\linewidth]{Plots/Calibana/pecorr_w_x.pdf}
  \end{subfigure}
  \begin{subfigure}{0.495\textwidth}
    \includegraphics[width=\linewidth]{Plots/Calibana/pecorr_w_y.pdf}
  \end{subfigure}
  \begin{subfigure}{0.495\textwidth}
    \includegraphics[width=\linewidth]{Plots/Calibana/cm_w_x.pdf}
  \end{subfigure}
  \begin{subfigure}{0.495\textwidth}
    \includegraphics[width=\linewidth]{Plots/Calibana/cm_w_y.pdf}
  \end{subfigure}
  \caption{Distributions of stopping muons within a 1-2 m track window from the end of their tracks across the position within a cell.}
  %\label{fig:AbsCalibW2}
\end{figure}

\begin{figure}[!ht]
  \begin{subfigure}{\textwidth}
  \centering
    \includegraphics[height=0.2\linewidth]{Plots/Calibana/legend.pdf}
  \end{subfigure}
  \vspace*{2mm}

  \begin{subfigure}{0.495\textwidth}
    \includegraphics[width=\linewidth]{Plots/Calibana/pecm_cell_x.pdf}
  \end{subfigure}
  \begin{subfigure}{0.495\textwidth}
    \includegraphics[width=\linewidth]{Plots/Calibana/pecm_cell_y.pdf}
  \end{subfigure}
  \begin{subfigure}{0.495\textwidth}
    \includegraphics[width=\linewidth]{Plots/Calibana/pecorrcm_cell_x.pdf}
  \end{subfigure}
  \begin{subfigure}{0.495\textwidth}
    \includegraphics[width=\linewidth]{Plots/Calibana/pecorrcm_cell_y.pdf}
  \end{subfigure}
    \begin{subfigure}{0.495\textwidth}
    \includegraphics[width=\linewidth]{Plots/Calibana/recomevcm_cell_x.pdf}
  \end{subfigure}
  \begin{subfigure}{0.495\textwidth}
    \includegraphics[width=\linewidth]{Plots/Calibana/recomevcm_cell_y.pdf}
  \end{subfigure}
  \caption{Distributions of stopping muons within a 1-2 m track window from the end of their tracks across the cells of the detector.}
  %\label{fig:AbsCalibCell1}
\end{figure}

\begin{figure}[!ht]
  \begin{subfigure}{\textwidth}
  \centering
    \includegraphics[height=0.2\linewidth]{Plots/Calibana/legend.pdf}
  \end{subfigure}
  \vspace*{2mm}

  \begin{subfigure}{0.495\textwidth}
    \includegraphics[width=\linewidth]{Plots/Calibana/pe_cell_x.pdf}
  \end{subfigure}
  \begin{subfigure}{0.495\textwidth}
    \includegraphics[width=\linewidth]{Plots/Calibana/pe_cell_y.pdf}
  \end{subfigure}
  \begin{subfigure}{0.495\textwidth}
    \includegraphics[width=\linewidth]{Plots/Calibana/pecorr_cell_x.pdf}
  \end{subfigure}
  \begin{subfigure}{0.495\textwidth}
    \includegraphics[width=\linewidth]{Plots/Calibana/pecorr_cell_y.pdf}
  \end{subfigure}
  \begin{subfigure}{0.495\textwidth}
    \includegraphics[width=\linewidth]{Plots/Calibana/cm_cell_x.pdf}
  \end{subfigure}
  \begin{subfigure}{0.495\textwidth}
    \includegraphics[width=\linewidth]{Plots/Calibana/cm_cell_y.pdf}
  \end{subfigure}
  \caption{Distributions of stopping muons within a 1-2 m track window from the end of their tracks across the cells of the detector.}
  %\label{fig:AbsCalibCell2}
\end{figure}

\begin{figure}[!ht]
  \begin{subfigure}{\textwidth}
  \centering
    \includegraphics[height=0.2\linewidth]{Plots/Calibana/legend.pdf}
  \end{subfigure}
  \vspace*{2mm}

  \begin{subfigure}{0.495\textwidth}
    \includegraphics[width=\linewidth]{Plots/Calibana/pecm_plane_x.pdf}
  \end{subfigure}
  \begin{subfigure}{0.495\textwidth}
    \includegraphics[width=\linewidth]{Plots/Calibana/pecm_plane_y.pdf}
  \end{subfigure}
  \begin{subfigure}{0.495\textwidth}
    \includegraphics[width=\linewidth]{Plots/Calibana/pecorrcm_plane_x.pdf}
  \end{subfigure}
  \begin{subfigure}{0.495\textwidth}
    \includegraphics[width=\linewidth]{Plots/Calibana/pecorrcm_plane_y.pdf}
  \end{subfigure}
    \begin{subfigure}{0.495\textwidth}
    \includegraphics[width=\linewidth]{Plots/Calibana/recomevcm_plane_x.pdf}
  \end{subfigure}
  \begin{subfigure}{0.495\textwidth}
    \includegraphics[width=\linewidth]{Plots/Calibana/recomevcm_plane_y.pdf}
  \end{subfigure}
  \caption{Distributions of stopping muons within a 1-2 m track window from the end of their tracks across the planes of the detector.}
  %\label{fig:AbsCalibPlane1}
\end{figure}

\begin{figure}[!ht]
  \begin{subfigure}{\textwidth}
  \centering
    \includegraphics[height=0.2\linewidth]{Plots/Calibana/legend.pdf}
  \end{subfigure}
  \vspace*{2mm}

  \begin{subfigure}{0.495\textwidth}
    \includegraphics[width=\linewidth]{Plots/Calibana/pe_plane_x.pdf}
  \end{subfigure}
  \begin{subfigure}{0.495\textwidth}
    \includegraphics[width=\linewidth]{Plots/Calibana/pe_plane_y.pdf}
  \end{subfigure}
  \begin{subfigure}{0.495\textwidth}
    \includegraphics[width=\linewidth]{Plots/Calibana/pecorr_plane_x.pdf}
  \end{subfigure}
  \begin{subfigure}{0.495\textwidth}
    \includegraphics[width=\linewidth]{Plots/Calibana/pecorr_plane_y.pdf}
  \end{subfigure}
  \begin{subfigure}{0.495\textwidth}
    \includegraphics[width=\linewidth]{Plots/Calibana/cm_plane_x.pdf}
  \end{subfigure}
  \begin{subfigure}{0.495\textwidth}
    \includegraphics[width=\linewidth]{Plots/Calibana/cm_plane_y.pdf}
  \end{subfigure}
  \caption{Distributions of stopping muons within a 1-2 m track window from the end of their tracks across the planes of the detector.}
  %\label{fig:AbsCalibPlane2}
\end{figure}

%\subsection{Drift in TB data}

\begin{figure}[!ht]
  \begin{subfigure}{\textwidth}
    \centering
    \includegraphics[height=0.2\linewidth]{Plots/Calibana/legend.pdf}
  \end{subfigure}
  \vspace*{2mm}
  
  \begin{subfigure}{0.495\textwidth}
    \includegraphics[width=\linewidth]{Plots/Calibana/pecm_time_x.pdf}
  \end{subfigure}
  \begin{subfigure}{0.495\textwidth}
    \includegraphics[width=\linewidth]{Plots/Calibana/pecm_time_y.pdf}
  \end{subfigure}
  \begin{subfigure}{0.495\textwidth}
    \includegraphics[width=\linewidth]{Plots/Calibana/pecorrcm_time_x.pdf}
  \end{subfigure}
  \begin{subfigure}{0.495\textwidth}
    \includegraphics[width=\linewidth]{Plots/Calibana/pecorrcm_time_y.pdf}
  \end{subfigure}
    \begin{subfigure}{0.495\textwidth}
    \includegraphics[width=\linewidth]{Plots/Calibana/recomevcm_time_x.pdf}
  \end{subfigure}
  \begin{subfigure}{0.495\textwidth}
    \includegraphics[width=\linewidth]{Plots/Calibana/recomevcm_time_y.pdf}
  \end{subfigure}
  \caption{Distributions of stopping muons within a 1-2 m track window from the end of their tracks across the event UNIX time.}
  %\label{fig:AbsCalibDrift1}
\end{figure}

\begin{figure}[!ht]
  \begin{subfigure}{\textwidth}
    \centering
    \includegraphics[height=0.2\linewidth]{Plots/Calibana/legend.pdf}
  \end{subfigure}
  \vspace*{2mm}
  
  \begin{subfigure}{0.495\textwidth}
    \includegraphics[width=\linewidth]{Plots/Calibana/pe_time_x.pdf}
  \end{subfigure}
  \begin{subfigure}{0.495\textwidth}
    \includegraphics[width=\linewidth]{Plots/Calibana/pe_time_y.pdf}
  \end{subfigure}
  \begin{subfigure}{0.495\textwidth}
    \includegraphics[width=\linewidth]{Plots/Calibana/pecorr_time_x.pdf}
  \end{subfigure}
  \begin{subfigure}{0.495\textwidth}
    \includegraphics[width=\linewidth]{Plots/Calibana/pecorr_time_y.pdf}
  \end{subfigure}
  \begin{subfigure}{0.495\textwidth}
    \includegraphics[width=\linewidth]{Plots/Calibana/cm_time_x.pdf}
  \end{subfigure}
  \begin{subfigure}{0.495\textwidth}
    \includegraphics[width=\linewidth]{Plots/Calibana/cm_time_y.pdf}
  \end{subfigure}
  \caption{Distributions of stopping muons within a 1-2 m track window from the end of their tracks across the event UNIX time.}
  %\label{fig:AbsCalibDrift2}
\end{figure}


\begin{figure}[ht!]
  \begin{subfigure}{\textwidth}
	\centering
   	\includegraphics[height=0.2\linewidth]{Plots/Calibana/legend.pdf}
  \end{subfigure}
  \vspace*{2mm}
  
  \begin{subfigure}{0.495\textwidth}
    \includegraphics[width=\linewidth]{Plots/Calibana/nhits_w_x.pdf}
  \end{subfigure}
  \begin{subfigure}{0.495\textwidth}
    \includegraphics[width=\linewidth]{Plots/Calibana/nhits_w_y.pdf}
  \end{subfigure}
  \begin{subfigure}{0.495\textwidth}
    \includegraphics[width=\linewidth]{Plots/Calibana/nhits_cell_x.pdf}
  \end{subfigure}
  \begin{subfigure}{0.495\textwidth}
    \includegraphics[width=\linewidth]{Plots/Calibana/nhits_cell_y.pdf}
  \end{subfigure}
  \begin{subfigure}{0.495\textwidth}
    \includegraphics[width=\linewidth]{Plots/Calibana/nhits_plane_x.pdf}
  \end{subfigure}
  \begin{subfigure}{0.495\textwidth}
    \includegraphics[width=\linewidth]{Plots/Calibana/nhits_plane_y.pdf}
  \end{subfigure}
    \caption{Distributions of stopping muons within a 1-2 m track window from the end of their tracks.}
  %\label{fig:AbsCalibNHitsWCellPlane}
\end{figure}

\FloatBarrier
\section{Distributions for Through-going Muons}

\begin{figure}[!ht]
  \begin{subfigure}{0.495\textwidth}
    \includegraphics[width=\linewidth]{Plots/PCListAna/DataAndSim_pecm_ts_w_X.pdf}
  \end{subfigure}
  \begin{subfigure}{0.495\textwidth}
    \includegraphics[width=\linewidth]{Plots/PCListAna/DataAndSim_pecm_ts_w_y.pdf}
  \end{subfigure}
  \begin{subfigure}{0.495\textwidth}
    \includegraphics[width=\linewidth]{Plots/PCListAna/DataAndSim_pecorrcm_ts_w_x.pdf}
  \end{subfigure}
  \begin{subfigure}{0.495\textwidth}
    \includegraphics[width=\linewidth]{Plots/PCListAna/DataAndSim_pecorrcm_ts_w_y.pdf}
  \end{subfigure}
    \begin{subfigure}{0.495\textwidth}
    \includegraphics[width=\linewidth]{Plots/PCListAna/DataAndSim_recomevcm_ts_w_x.pdf}
  \end{subfigure}
  \begin{subfigure}{0.495\textwidth}
    \includegraphics[width=\linewidth]{Plots/PCListAna/DataAndSim_recomevcm_ts_w_y.pdf}
  \end{subfigure}
  \caption[Validation plots for through-going muons along w]{Distributions of through-going cosmic muons with $w\in\left(-80,80\right)\unit{cm}$ as a function of $w$ for stable runs in the Test Beam period 4 data (black) and data-based simulation (red). Bottom panel of each plot shows the ratio of each bin and the mean y axis, separately for data and simulation. Discrepancy in the right-most bin is solely due to binning.}
\end{figure}

\begin{figure}[!ht]
  \begin{subfigure}{0.495\textwidth}
    \includegraphics[width=\linewidth]{Plots/PCListAna/DataAndSim_pe_w_x.pdf}
  \end{subfigure}
  \begin{subfigure}{0.495\textwidth}
    \includegraphics[width=\linewidth]{Plots/PCListAna/DataAndSim_pe_w_y.pdf}
  \end{subfigure}
  \begin{subfigure}{0.495\textwidth}
    \includegraphics[width=\linewidth]{Plots/PCListAna/DataAndSim_pecorr_w_x.pdf}
  \end{subfigure}
  \begin{subfigure}{0.495\textwidth}
    \includegraphics[width=\linewidth]{Plots/PCListAna/DataAndSim_pecorr_w_y.pdf}
  \end{subfigure}
  \begin{subfigure}{0.495\textwidth}
    \includegraphics[width=\linewidth]{Plots/PCListAna/DataAndSim_cm_w_x.pdf}
  \end{subfigure}
  \begin{subfigure}{0.495\textwidth}
    \includegraphics[width=\linewidth]{Plots/PCListAna/DataAndSim_cm_w_y.pdf}
  \end{subfigure}
  \caption[Validation plots for through-going muons along w]{Distributions of through-going cosmic muons with $w\in\left(-80,80\right)\unit{cm}$ as a function of $w$ for stable runs in the Test Beam period 4 data (black) and data-based simulation (red). Bottom panel of each plot shows the ratio of each bin and the mean y axis, separately for data and simulation. Discrepancy in the right-most bin is solely due to binning.}
\end{figure}

\begin{figure}[!ht]
  \begin{subfigure}{0.495\textwidth}
    \includegraphics[width=\linewidth]{Plots/PCListAna/DataAndSim_pecm_ts_cell_X.pdf}
  \end{subfigure}
  \begin{subfigure}{0.495\textwidth}
    \includegraphics[width=\linewidth]{Plots/PCListAna/DataAndSim_pecm_ts_cell_y.pdf}
  \end{subfigure}
  \begin{subfigure}{0.495\textwidth}
    \includegraphics[width=\linewidth]{Plots/PCListAna/DataAndSim_pecorrcm_ts_cell_x.pdf}
  \end{subfigure}
  \begin{subfigure}{0.495\textwidth}
    \includegraphics[width=\linewidth]{Plots/PCListAna/DataAndSim_pecorrcm_ts_cell_y.pdf}
  \end{subfigure}
    \begin{subfigure}{0.495\textwidth}
    \includegraphics[width=\linewidth]{Plots/PCListAna/DataAndSim_recomevcm_ts_cell_x.pdf}
  \end{subfigure}
  \begin{subfigure}{0.495\textwidth}
    \includegraphics[width=\linewidth]{Plots/PCListAna/DataAndSim_recomevcm_ts_cell_y.pdf}
  \end{subfigure}
  \caption[Validation plots for through-going muons across cells]{Distributions of through-going cosmic muons with $w\in\left(-80,80\right)\unit{cm}$ as a function of cell number for stable runs in the Test Beam period 4 data (black) and data-based simulation (red). Bottom panel of each plot shows the ratio of each bin and the mean y axis, separately for data and simulation.}
\end{figure}

\begin{figure}[!ht]
  \begin{subfigure}{0.495\textwidth}
    \includegraphics[width=\linewidth]{Plots/PCListAna/DataAndSim_pe_cell_x.pdf}
  \end{subfigure}
  \begin{subfigure}{0.495\textwidth}
    \includegraphics[width=\linewidth]{Plots/PCListAna/DataAndSim_pe_cell_y.pdf}
  \end{subfigure}
  \begin{subfigure}{0.495\textwidth}
    \includegraphics[width=\linewidth]{Plots/PCListAna/DataAndSim_pecorr_cell_x.pdf}
  \end{subfigure}
  \begin{subfigure}{0.495\textwidth}
    \includegraphics[width=\linewidth]{Plots/PCListAna/DataAndSim_pecorr_cell_y.pdf}
  \end{subfigure}
  \begin{subfigure}{0.495\textwidth}
    \includegraphics[width=\linewidth]{Plots/PCListAna/DataAndSim_cm_cell_x.pdf}
  \end{subfigure}
  \begin{subfigure}{0.495\textwidth}
    \includegraphics[width=\linewidth]{Plots/PCListAna/DataAndSim_cm_cell_y.pdf}
  \end{subfigure}
  \caption[Validation plots for through-going muons across cells]{Distributions of through-going cosmic muons with $w\in\left(-80,80\right)\unit{cm}$ as a function of cell number for stable runs in the Test Beam period 4 data (black) and data-based simulation (red). Bottom panel of each plot shows the ratio of each bin and the mean y axis, separately for data and simulation.}
\end{figure}

\begin{figure}[!ht]
  \begin{subfigure}{0.495\textwidth}
    \includegraphics[width=\linewidth]{Plots/PCListAna/DataAndSim_pecm_ts_plane_X.pdf}
  \end{subfigure}
  \begin{subfigure}{0.495\textwidth}
    \includegraphics[width=\linewidth]{Plots/PCListAna/DataAndSim_pecm_ts_plane_y.pdf}
  \end{subfigure}
  \begin{subfigure}{0.495\textwidth}
    \includegraphics[width=\linewidth]{Plots/PCListAna/DataAndSim_pecorrcm_ts_plane_x.pdf}
  \end{subfigure}
  \begin{subfigure}{0.495\textwidth}
    \includegraphics[width=\linewidth]{Plots/PCListAna/DataAndSim_pecorrcm_ts_plane_y.pdf}
  \end{subfigure}
    \begin{subfigure}{0.495\textwidth}
    \includegraphics[width=\linewidth]{Plots/PCListAna/DataAndSim_recomevcm_ts_plane_x.pdf}
  \end{subfigure}
  \begin{subfigure}{0.495\textwidth}
    \includegraphics[width=\linewidth]{Plots/PCListAna/DataAndSim_recomevcm_ts_plane_y.pdf}
  \end{subfigure}
  \caption[Validation plots for through-going muons across planes]{Distributions of through-going cosmic muons with $w\in\left(-80,80\right)\unit{cm}$ as a function of plane number for stable runs in the Test Beam period 4 data (black) and data-based simulation (red). Bottom panel of each plot shows the ratio of each bin and the mean y axis, separately for data and simulation.}
\end{figure}

\begin{figure}[!ht]
  \begin{subfigure}{0.495\textwidth}
    \includegraphics[width=\linewidth]{Plots/PCListAna/DataAndSim_pe_plane_x.pdf}
  \end{subfigure}
  \begin{subfigure}{0.495\textwidth}
    \includegraphics[width=\linewidth]{Plots/PCListAna/DataAndSim_pe_plane_y.pdf}
  \end{subfigure}
  \begin{subfigure}{0.495\textwidth}
    \includegraphics[width=\linewidth]{Plots/PCListAna/DataAndSim_pecorr_plane_x.pdf}
  \end{subfigure}
  \begin{subfigure}{0.495\textwidth}
    \includegraphics[width=\linewidth]{Plots/PCListAna/DataAndSim_pecorr_plane_y.pdf}
  \end{subfigure}
  \begin{subfigure}{0.495\textwidth}
    \includegraphics[width=\linewidth]{Plots/PCListAna/DataAndSim_cm_plane_x.pdf}
  \end{subfigure}
  \begin{subfigure}{0.495\textwidth}
    \includegraphics[width=\linewidth]{Plots/PCListAna/DataAndSim_cm_plane_y.pdf}
  \end{subfigure}
  \caption[Validation plots for through-going muons across planes]{Distributions of through-going cosmic muons with $w\in\left(-80,80\right)\unit{cm}$ as a function of plane number for stable runs in the Test Beam period 4 data (black) and data-based simulation (red). Bottom panel of each plot shows the ratio of each bin and the mean y axis, separately for data and simulation.}
\end{figure}

\begin{figure}[!ht]
  \begin{subfigure}{0.495\textwidth}
    \includegraphics[width=\linewidth]{Plots/PCListAna/DataAndSim_pecm_ts_cosx_X.pdf}
  \end{subfigure}
  \begin{subfigure}{0.495\textwidth}
    \includegraphics[width=\linewidth]{Plots/PCListAna/DataAndSim_pecm_ts_cosx_y.pdf}
  \end{subfigure}
  \begin{subfigure}{0.495\textwidth}
    \includegraphics[width=\linewidth]{Plots/PCListAna/DataAndSim_pecorrcm_ts_cosx_x.pdf}
  \end{subfigure}
  \begin{subfigure}{0.495\textwidth}
    \includegraphics[width=\linewidth]{Plots/PCListAna/DataAndSim_pecorrcm_ts_cosx_y.pdf}
  \end{subfigure}
    \begin{subfigure}{0.495\textwidth}
    \includegraphics[width=\linewidth]{Plots/PCListAna/DataAndSim_recomevcm_ts_cosx_x.pdf}
  \end{subfigure}
  \begin{subfigure}{0.495\textwidth}
    \includegraphics[width=\linewidth]{Plots/PCListAna/DataAndSim_recomevcm_ts_cosx_y.pdf}
  \end{subfigure}
  \caption[Validation plots for through-going muons along angle from the x axis]{Distributions of through-going cosmic muons with $w\in\left(-80,80\right)\unit{cm}$ as a function of the cosine of the angle from the x axis for stable runs in the Test Beam period 4 data (black) and data-based simulation (red). Bottom panel of each plot shows the ratio of each bin and the mean y axis, separately for data and simulation.}
\end{figure}

\begin{figure}[!ht]
  \begin{subfigure}{0.495\textwidth}
    \includegraphics[width=\linewidth]{Plots/PCListAna/DataAndSim_pe_cosx_x.pdf}
  \end{subfigure}
  \begin{subfigure}{0.495\textwidth}
    \includegraphics[width=\linewidth]{Plots/PCListAna/DataAndSim_pe_cosx_y.pdf}
  \end{subfigure}
  \begin{subfigure}{0.495\textwidth}
    \includegraphics[width=\linewidth]{Plots/PCListAna/DataAndSim_pecorr_cosx_x.pdf}
  \end{subfigure}
  \begin{subfigure}{0.495\textwidth}
    \includegraphics[width=\linewidth]{Plots/PCListAna/DataAndSim_pecorr_cosx_y.pdf}
  \end{subfigure}
  \begin{subfigure}{0.495\textwidth}
    \includegraphics[width=\linewidth]{Plots/PCListAna/DataAndSim_cm_cosx_x.pdf}
  \end{subfigure}
  \begin{subfigure}{0.495\textwidth}
    \includegraphics[width=\linewidth]{Plots/PCListAna/DataAndSim_cm_cosx_y.pdf}
  \end{subfigure}
  \caption[Validation plots for through-going muons along angle from the x axis]{Distributions of through-going cosmic muons with $w\in\left(-80,80\right)\unit{cm}$ as a function of the cosine of the angle from the x axis for stable runs in the Test Beam period 4 data (black) and data-based simulation (red). Bottom panel of each plot shows the ratio of each bin and the mean y axis, separately for data and simulation.}
\end{figure}

\begin{figure}[!ht]
  \begin{subfigure}{0.495\textwidth}
    \includegraphics[width=\linewidth]{Plots/PCListAna/DataAndSim_pecm_ts_cosy_X.pdf}
  \end{subfigure}
  \begin{subfigure}{0.495\textwidth}
    \includegraphics[width=\linewidth]{Plots/PCListAna/DataAndSim_pecm_ts_cosy_y.pdf}
  \end{subfigure}
  \begin{subfigure}{0.495\textwidth}
    \includegraphics[width=\linewidth]{Plots/PCListAna/DataAndSim_pecorrcm_ts_cosy_x.pdf}
  \end{subfigure}
  \begin{subfigure}{0.495\textwidth}
    \includegraphics[width=\linewidth]{Plots/PCListAna/DataAndSim_pecorrcm_ts_cosy_y.pdf}
  \end{subfigure}
    \begin{subfigure}{0.495\textwidth}
    \includegraphics[width=\linewidth]{Plots/PCListAna/DataAndSim_recomevcm_ts_cosy_x.pdf}
  \end{subfigure}
  \begin{subfigure}{0.495\textwidth}
    \includegraphics[width=\linewidth]{Plots/PCListAna/DataAndSim_recomevcm_ts_cosy_y.pdf}
  \end{subfigure}
  \caption[Validation plots for through-going muons along angle from the y axis]{Distributions of through-going cosmic muons with $w\in\left(-80,80\right)\unit{cm}$ as a function of the cosine of the angle from the y axis for stable runs in the Test Beam period 4 data (black) and data-based simulation (red). Bottom panel of each plot shows the ratio of each bin and the mean y axis, separately for data and simulation.}
\end{figure}

\begin{figure}[!ht]
  \begin{subfigure}{0.495\textwidth}
    \includegraphics[width=\linewidth]{Plots/PCListAna/DataAndSim_pe_cosy_x.pdf}
  \end{subfigure}
  \begin{subfigure}{0.495\textwidth}
    \includegraphics[width=\linewidth]{Plots/PCListAna/DataAndSim_pe_cosy_y.pdf}
  \end{subfigure}
  \begin{subfigure}{0.495\textwidth}
    \includegraphics[width=\linewidth]{Plots/PCListAna/DataAndSim_pecorr_cosy_x.pdf}
  \end{subfigure}
  \begin{subfigure}{0.495\textwidth}
    \includegraphics[width=\linewidth]{Plots/PCListAna/DataAndSim_pecorr_cosy_y.pdf}
  \end{subfigure}
  \begin{subfigure}{0.495\textwidth}
    \includegraphics[width=\linewidth]{Plots/PCListAna/DataAndSim_cm_cosy_x.pdf}
  \end{subfigure}
  \begin{subfigure}{0.495\textwidth}
    \includegraphics[width=\linewidth]{Plots/PCListAna/DataAndSim_cm_cosy_y.pdf}
  \end{subfigure}
  \caption[Validation plots for through-going muons along angle from the y axis]{Distributions of through-going cosmic muons with $w\in\left(-80,80\right)\unit{cm}$ as a function of the cosine of the angle from the y axis for stable runs in the Test Beam period 4 data (black) and data-based simulation (red). Bottom panel of each plot shows the ratio of each bin and the mean y axis, separately for data and simulation.}
\end{figure}

\begin{figure}[!ht]
  \begin{subfigure}{0.495\textwidth}
    \includegraphics[width=\linewidth]{Plots/PCListAna/DataAndSim_pecm_ts_cosz_X.pdf}
  \end{subfigure}
  \begin{subfigure}{0.495\textwidth}
    \includegraphics[width=\linewidth]{Plots/PCListAna/DataAndSim_pecm_ts_cosz_y.pdf}
  \end{subfigure}
  \begin{subfigure}{0.495\textwidth}
    \includegraphics[width=\linewidth]{Plots/PCListAna/DataAndSim_pecorrcm_ts_cosz_x.pdf}
  \end{subfigure}
  \begin{subfigure}{0.495\textwidth}
    \includegraphics[width=\linewidth]{Plots/PCListAna/DataAndSim_pecorrcm_ts_cosz_y.pdf}
  \end{subfigure}
    \begin{subfigure}{0.495\textwidth}
    \includegraphics[width=\linewidth]{Plots/PCListAna/DataAndSim_recomevcm_ts_cosz_x.pdf}
  \end{subfigure}
  \begin{subfigure}{0.495\textwidth}
    \includegraphics[width=\linewidth]{Plots/PCListAna/DataAndSim_recomevcm_ts_cosz_y.pdf}
  \end{subfigure}
  \caption[Validation plots for through-going muons along angle from the z axis]{Distributions of through-going cosmic muons with $w\in\left(-80,80\right)\unit{cm}$ as a function of the cosine of the angle from the z axis for stable runs in the Test Beam period 4 data (black) and data-based simulation (red). Bottom panel of each plot shows the ratio of each bin and the mean y axis, separately for data and simulation.}
\end{figure}

\begin{figure}[!ht]
  \begin{subfigure}{0.495\textwidth}
    \includegraphics[width=\linewidth]{Plots/PCListAna/DataAndSim_pe_cosz_x.pdf}
  \end{subfigure}
  \begin{subfigure}{0.495\textwidth}
    \includegraphics[width=\linewidth]{Plots/PCListAna/DataAndSim_pe_cosz_y.pdf}
  \end{subfigure}
  \begin{subfigure}{0.495\textwidth}
    \includegraphics[width=\linewidth]{Plots/PCListAna/DataAndSim_pecorr_cosz_x.pdf}
  \end{subfigure}
  \begin{subfigure}{0.495\textwidth}
    \includegraphics[width=\linewidth]{Plots/PCListAna/DataAndSim_pecorr_cosz_y.pdf}
  \end{subfigure}
  \begin{subfigure}{0.495\textwidth}
    \includegraphics[width=\linewidth]{Plots/PCListAna/DataAndSim_cm_cosz_x.pdf}
  \end{subfigure}
  \begin{subfigure}{0.495\textwidth}
    \includegraphics[width=\linewidth]{Plots/PCListAna/DataAndSim_cm_cosz_y.pdf}
  \end{subfigure}
  \caption[Validation plots for through-going muons along angle from the z axis]{Distributions of through-going cosmic muons with $w\in\left(-80,80\right)\unit{cm}$ as a function of the cosine of the angle from the x axis for stable runs in the Test Beam period 4 data (black) and data-based simulation (red). Bottom panel of each plot shows the ratio of each bin and the mean y axis, separately for data and simulation.}
\end{figure}

%\section{First Attachment}

\openright
\end{document}
