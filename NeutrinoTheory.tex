\chapter{Theory of neutrino physics}\label{sec:NeutrinoTheory}

Just give a short overview of the historical context, but mainly focusing on the actual description of the neutrino theory, mostly stating the fact rather than giving a large background

I should discuss everything that is even briefly mentioned in the neutrino magnetic moment theory section.
\begin{itemize}
\item Dirac vs Majorana neutrinos
\item Neutrino masses
\item Neutrino interactions with electrons and nuclei
\item Neutrino oscillations and their implications
\end{itemize}

The story:
\begin{enumerate}
\item Brief history up to neutrinos being in the SM
\item Description of neutrinos in the SM
\item Interactions of neutrinos and their detection
\item Production of neutrinos
\item Solar and atmospheric neutrino anomalies and neutrino oscillations
\item Detail of neutrino oscillations for three flavours
\item Current state of neutrino oscillation measurements
\item Mass ordering, octant, delta CP
\item Neutrino masses - generation and measurements
\item Dirac V Majorana neutrinos
\end{enumerate}

%%%%%%%%%%%%%%%%%%%%%%%%%%%%%%%%%%%%%%%%%%%%%%%%%%%%%%%%%%%%%%%%%%%%%%%%%%%%%%%
%%% 1. Brief history up to neutrinos being in the SM

Neutrinos were first introduced by Pauli \cite{PauliNeutrinoProposalLetter.pdf,TheIdeaOfTheNeutrino.pdf} as very light electrically neutral particles with half-spin and a possible magnetic moment \cite{NeutrinoMagMomentImplications1934.pdf}. They were a crucial part of Enrico Fermi's successful theory of $\beta$ decays \cite{FermisTheoryOfBetaDecayOriginal.pdf, FermisTheoryOfBetaDecay.pdf}, which solidified their importance in particle physics even before their first experimental detection.
Fermi's theory developed into the \gls{SM} of particle physics \cite{SMGlashow.pdf,SMWeinberg.pdf,SMSalam.pdf}, which in its current form contains three generations of fermions. Each generations contains two quarks, one charged lepton and an associated neutrino with no mass or magnetic moment. 

\gls{SM} is mathematically described by a Lagrangian, in which neutrinos are expressed as two-component left-handed chiral fields $\nu_{\alpha L}$, where $\alpha=e,\mu,\tau$ denotes the three neutrino generations, also called flavours \cite{LandauParityViolationForNus.pdf, LeeYangNuAsMasslessWeylSpinor.pdf, SalamNuAsMasslessWeylSpinors.pdf}. Neutrinos form weak isospin doublets with their associated left handed charged lepton fields $\alpha_L$. Unlike for the charged leptons, there is no right handed neutrino singlet in the \gls{SM}. This means that neutrinos cannot obtain a (Dirac) mass term, since the mass terms for fermions arise from the Higgs mechanism \cite{HiggsMechanismOriginal1964.pdf, HiggMechanismEnglertBrut1964.pdf, HiggsMechanismGuralnikHagenKibble1964.pdf} via the Yukawa coupling of the fermion and the Higgs fields \cite{YukawaLagrangiaWeinberg1967.pdf}, which requires a combination of left-handed and right-handed fields \cite{FundamentalsOfNeutrinoPhysics.pdf}.

%\cite{FundamentalsOfNeutrinoPhysics.pdf} In the SM, the mass of fermions arises as a result of the Higgs mechanism through the presence of Yukawa couplings of the fermion fields with the Higgs doublet. ... a fermion mass term must involve a coupling of left-handed and right-handed fileds, so it is clear that in the SM neutrinos are massless, because their fields do not have a right-handed components

The interaction terms for neutrinos can be separated into two components defined by the vector gauge field they interact with. They are describing the \gls{CC} and the \gls{NC} interactions, based on whether they interact with the $W_\mu$ or $Z_\mu$ fields, which describe the $W^\pm$ or $Z^0$ weak boson respectively. Neglecting the non-neutrino components, the neutrinos interaction terms are \cite{FundamentalsOfNeutrinoPhysics.pdf}
\begin{equation}\label{eq:NuIntLagrangian}
\mathcal{L}_{\textsc{CC}}^{\textsc{SM}}=-\frac{g_w}{2\sqrt{2}}j^\mu_W W_\mu +\textsf{h.c.},\,\,\,\textsf{and}\,\,\,
\mathcal{L}_{\textsc{NC}}^{\textsc{SM}}=-\frac{g_w}{2\cos\left(\theta_W\right)}j^\mu_Z Z_\mu.
\end{equation}
Here $g_w$ is the weak coupling constant, $\theta_W$ is the Weinberg angle and $j^\mu_W$ and $j^\mu_Z$ are the weak currents expressed as
\begin{equation}
j^\mu_W=2\sum_{\alpha=e,\mu,\tau}\overline{\nu}_{\alpha L}\gamma^\mu\alpha_L,
\end{equation}
\begin{equation}
j^\mu_Z=2\sum_{\alpha=e,\mu,\tau} g^\nu_L \overline{\nu}_{\alpha L} \gamma^\mu \nu_{\alpha L}.
\end{equation}
$\gamma^\mu,\mu=0,1,2,3$, are the four Dirac gamma matrices and $g_L^\nu$ is the coupling term, which for neutrinos $g_L^\nu=+1/2$.

The two terms of the interaction Lagrangian from Eq.~\ref{eq:NuIntLagrangian} describe the possible neutrino interaction vertices, shown in Fig.~\ref{fig:FeynmanNuIntVertices}. These diagrams show the \gls{CC} and \gls{NC} interaction of neutrinos and antineutrinos and, in case of the \gls{CC} diagram, can also be flipped around the vertical axis to show the production of neutrinos from the weak interaction (or decays) of leptons. They can also be rotated $90^{\circ}$ to show annihilation, or production, of the neutrino-lepton (for \gls{CC}), or neutrino-antineutrino (for \gls{NC}) pairs.

\begin{figure}[hbtp]
\centering
%\includegraphics[height=3.2cm]{Plots/Theory/NeutrinoCCAnnihilationVertices.png}
\includegraphics[width=.4\linewidth]{Plots/Theory/NeutrinoCCInteractionVertices.png}
\hspace{0.1\linewidth}
\includegraphics[width=.4\linewidth]{Plots/Theory/NeutrinoNCInteractionVertices.png}
\caption[Neutrino interaction vertices in the SM]{Neutrino interaction vertices in the \acrshort{SM} via the weak charged currents (left) and the neutral currents (right).}
\label{fig:FeynmanNuIntVertices}
\end{figure}


% in weak isospin doublets \cite{FundamentalsOfNeutrinoPhysics.pdf} grouped with their lepton left handed chiral components of charge leptons. Since SM neutrinos do not have a right handed counterpart, they can't acquire mass through the Yukawa lagrangian which combined the left and right handed fields. Their interaction lagrangian is...

%Neutrinos in the \gls{SM} are described as massless with a left-handed chiral field, with no right-handed neutrino chiral field counter part. Neutrinos make a lepton doublet together with their associated leptons. There are no neutrino mass terms in the \gls{SM} Lagrangian and the interaction terms can be divided into two types: the \gls{CC} and the \gls{NC} interactions \cite{FundamentalsOfNeutrinoPhysics.pdf}

%Neutrinos are grouped together with their corresponding charged lepton to form isospin doublets with no right handed neutrino singlet counterpart. In the SM a fermion mass term must involve a coupling of left-handed and right-handed fields and since neutrinos only have left handed fields they can't obtain

%[nuMM/nuElmagInt2015.pdf] However, there was no sign of a neutrino mass. After the discovery of parity violation in 1957, Landau (1957), Lee and Yang (1957), and Salam (1957) proposed the two-component theory of massless neutrinos, in which a neutrino is described by a Weyl spinor and there are only left-handed neutrinos and right-handed antineutrinos. It was, however, clear (Case, 1957; Mclennan, 1957; Radicati and Touschek, 1957) that two-component neutrinos could be massive Majorana fermions and that the two-component theory of a massless neutrino is equivalent to the Majorana theory in the limit of zero neutrino mass. The two-component theory of massless neutrinos was later incorporated in the standard model of Glashow (1961), Weinberg (1967), and Salam (1969), in which neutrinos are massless and have only weak interactions. In the standard model Majorana neutrino masses are forbidden by the $\textsf{SU}\left(2\right)_L\times \textsf{U}\left(1\right)_{\gamma}$ symmetry.

%[nuMM/nuElmagInt2015.pdf] In the standard model of electroweak interactions (Glashow, 1961; Weinberg, 1967; Salam, 1969), neutrinos are described by two-component massless left-handed Weyl spinors (Giunti and Kim, 2007). The masslessness of neutrinos is due to the absence of right-handed neutrino fields, without which it is not possible to have Dirac mass terms, and to the absence of Higgs triplets, without which it is not possible to have Majorana mass terms.

%[OverviewOfNeutrinoPhysicsPheno2024.pdf] three neutrino flavours produced with charged antilepton, or producing a charged lepton in CC weak interaction processes. Therefore neutrino have exhibited a polarisation in a direction that is opposite to their motion, or, equivalently, with negative helicity. For antineutrino it is the opposite. To account for this neutrino are described in the SM with a left handed chiral field $\nu_{\alpha L}\left(x\right)$. In the absence of neutrino masses, this field destroys (creates) neutrinos (antineutrinos) with negative (positive) helicity.  In the SM, neutrinos are considered massless, and no right-handed (RH) neutrino chiral field is included in its content as they would be singlet under the SM gauge group. As such, the corresponding RH neutrinos would be completely inert. Neutrinos and their lepton LH fields form SU(2) doublets $\psi_{\alpha L}\left(x\right)=\left(\nu_{\alpha L}\left(x\right),\alpha_L\left(x\right)\right)$ with hypercharge +1.

%%%%%%%%%%%%%%%%%%%%%%%%%%%%%%%%%%%%%%%%%%%%%%%%%%%%%%%%%%%%%%%%%%%%%%%%%%%%%%%
%%% 2. Neutrino production and sources
\section{Neutrino Production}
Some of the most common neutrino and antineutrino production channels include nucleon transitions via \gls{CC} weak interactions. Specifically, the transition of a neutron into a proton, either as a decay of a free neutron, or as a $\beta^-$ decay for neutrons bound in a nucleus, produces an electron and an electron antineutrino:
\begin{equation}
n\rightarrow p+e^-+\overline{\nu}_e.
\end{equation}
The shape of the electron spectrum from a $\beta^-$ decay was the reason Pauli proposed the existence of the neutrino \cite{PauliNeutrinoProposalLetter.pdf}. Additionally, this channel is an abundant source of $\overline{\nu}_e$ from nuclear reactors, which were the first artificial sources of neutrinos, increasing the neutrino flux by about 100 million compared to the naturally occurring sources, enabling the first ever detection of a neutrino \cite{CowanReinesFirstAttempt.pdf, CowanReinesConfirmation.pdf, NeutrinoPhysicsCowanReines.pdf}.

Similarly, the production of an electron neutrino via the transition of a proton into a neutron can occur inside the nucleus either as the $\beta^+$ decay:
\begin{equation}
p\rightarrow n+e^++\nu_e,
\end{equation}
or via the electron capture:
\begin{equation}
p+e^-\rightarrow n+\nu_e.
\end{equation}
This channel occurs in stars and in the first phase of supernovae \cite{FundamentalsOfNeutrinoPhysics.pdf}. However, most supernovae neutrinos are created via a thermal pair production via \gls{NC} interaction
\begin{equation}
e^-+e^+\rightarrow\nu_\alpha+\overline{\nu}_\alpha
\end{equation}
producing neutrinos and antineutrinos of all flavours. The neutrino pair production via the decay of $Z^0$ was studied in great detail \cite{ZDecay.pdf}, since the magnitude of the decay width depends on the number of light active neutrino flavours. The current best fit is $N_\nu=2.984$ \cite{ZDecayPrecise.pdf}.
%Similar interactions produced the currently unobservable relic neutrinos, which were produced during the Big Bang and have extremely low energies.

%In 1990 the L3 Collaboration studied properties of the $Z^0$ boson and fitted to its peak cross-section and decay width to determine the total number of active (interacting with $Z^0$) light ($m_{\nu}<m_{Z}/2$) neutrino flavours ($N_{\nu}$). They found the best fit integer value to be 3 and ruled out the possibility of four or more active light neutrino flavours at $4\sigma$ \cite{ZDecay.pdf}. Latest most precise results put the fitted value to $N_{\nu}=2.9840\pm 0.0082$ \cite{ZDecayPrecise.pdf}.

An abundant source of $\nu_\mu$ and $\overline{\nu}_\mu$ is the decay of pions and muons
\begin{align}
p+X \rightarrow \pi^\pm \rightarrow &\mu^\pm + \nu_\mu\left(\overline{\nu}_\mu\right) \\
 & \mu^\pm \rightarrow e^\pm + \nu_\mu\left(\overline{\nu}_\mu\right) + \nu_e\left(\overline{\nu}_e\right),
\end{align}
which naturally occurs in Earth's atmosphere from the interaction of cosmic ray protons. Notice, that if all muons decay by the time they reach Earth's surface, the ratio of $\nu_\mu : \nu_e$ should be exactly 2:1. This processed is also used in the modern accelerator-based source of neutrinos, which accelerate protons to the desired energies, shoot them onto a fixed target, and focus the resulting hadrons to achieve a highly pure and precise source of $\nu_\mu$ or $\overline{\nu}_\mu$ \cite{GoodmanAdvancesInNeutrinoPhysics.pdf, SchwartzAccelerators.pdf}. Similarly, decays of heavier hadrons, such as kaons and charmed particles, also produce neutrinos, including $\nu_\tau$ and $\overline{\nu}_\tau$ \cite{ObservationOfTauNeutrino.pdf, FinalTauNeutrinoResultsDONUT2008.pdf}.

%[Master's] The advent of fission reactors brought increase of neutrino rate of about $10^7$, as well as higher neutrino energies, making the neutrino detection worth reinvestigating \cite{NeutrinoPhysicsCowanReines.pdf}. In 1960 Mel Schwartz designed the first neutrino beam made by accelerated protons striking a target, producing pions (mostly), which would decay into neutrinos \cite{GoodmanAdvancesInNeutrinoPhysics.pdf}\cite{SchwartzAccelerators.pdf}.

%%%%%%%%%%%%%%%%%%%%%%%%%%%%%%%%%%%%%%%%%%%%%%%%%%%%%%%%%%%%%%%%%%%%%%%%%%%%%%%
%%% 3. Interaction of neutrinos
\section{Neutrino Interactions}
The interaction of neutrinos can either be categorized based on the target, which is generally either electron or a nucleus, or on the neutrino energy. Neutrino-electron interactions either occur via elastic scattering, which has the same neutrino and an electron in the final state, or via the inverse muon (or tau) decay, which contains a muon (or tau) in the final state. Both of these interactions are purely governed by \gls{QED} and are theoretically very well understood, using their measurements to provide constraints on the parameters within the \gls{QED} theory. While the neutrino-on-electron elastic scattering does not have a threshold energy and can occur for any neutrinos, the inverse muon decay has a threshold energy of $\unit[10.92]{GeV}$, and the inverse tau decay $>\unit[3]{TeV}$ \cite{FundamentalsOfNeutrinoPhysics.pdf, NeutrinoOnElectronElScatteringTheory2003.pdf}.

The interaction of neutrinos on a nucleus is more complicated, due to contributions from possible nuclear effects and the non-trivial nature of the individual nucleons. At low neutrino energies, the only currently detectable

CEvENS \cite{CEvENSFirstObservation2017.pdf}

For low energies neutrinos typically interact with the entire nucleus, either via CEvENS, or either via various nuclear models. At higher energies neutrinos start to probe the individual nucleons. At higher energies still, the produced nucleons can get excited and at higher energies still neutrinos start to probe inside the nucleons themselves.

For neutrinos interacting on the nuclei, we distinguish between different types of interactions based on what happens to the nucleus. If it's an interaction with a single proton or neutron, we call this \gls{QE} interaction. If this proton gets excited into a $\Delta$ resonance (which then generally decays into a $\pi$), we call this Resonant production, if neutrino penetrates through the nucleon and interacts directly with a quark inside it, we call this \gls{DIS} interaction. This is shown on Fig.~\ref{fig:NuCCCrossSection}. There can be additional subtypes due to nuclear effects, namely the 2p2h interaction \todo{Find a reference for the 2p2h} also called \gls{MEC}, or possible \gls{FSI}.

\begin{figure}[hbtp]
\centering
\includegraphics[width=0.8\linewidth]{Plots/Theory/NeutrinoCCCrossSections.png}
\caption[Neutrino CC cross sections based on the interaction types]{Neutrino \acrshort{CC} cross sections based on the interaction types. Figure from \cite{NeutrinoCCCrossSectionPlot.pdf} compares the measured data \cite{NeutrinoIntOverview2012.pdf} and the prediction \cite{NuanceNuIntSimulation2002.pdf}}
\label{fig:NuCCCrossSection}
\end{figure}

Inversely, the detection of neutrinos is usually done by reverting the above mentioned processes. The discussion of neutrino interactions is in \cite{NeutrinoIntOverview2012.pdf}

For example for the $\overline{\nu}_e$ we can use the so-called inverse beta decay
\begin{equation}
\overline{\nu}_e +p\rightarrow n+e^+
\end{equation}
was used for the first detection of neutrinos by Cowan and Reines \cite{CowanReinesFirstAttempt.pdf, CowanReinesConfirmation.pdf}. This is currently used in the reactor neutrino experiments \note{Should I mention some reactor experiments here?}

\cite{FundamentalsOfNeutrinoPhysics.pdf} 
nu-on-e is for example used for the detection of solar neutrinos in the Kamiokande experiments
Nu-on-e is mainly sensitive to electron neutrinos, whose cross section is about 6 times larger than for the muon/tau neutrinos.

%%%%%%%%%%%%%%%%%%%%%%%%%%%%%%%%%%%%%%%%%%%%%%%%%%%%%%%%%%%%%%%%%%%%%%%%%%%%%%%
%%% Masters
%%% Experimental evidence for neutrino %%%
%Soon after Fermi's description of neutrino interaction, in 1934, Bethe and Peierls realized the possibility of reverting the process of beta decay as a mean of direct detection of the neutrino \cite{BethePeierlsDirectDetection.pdf}. For example an interaction in which incident neutrino interacts with proton, transforming it into neutron and creating a positron. From the lifetime of then-known beta decays they estimated the interaction cross-section to be $<\unit[10^{-44}]{cm^2}$ for a neutrino with a few $\unit{MeV}$ energy, or about $10^{-20}$ times the more familiar nuclear values at the time.

%In 1953 C.~L.~Cowan and F.~Reines placed a liquid scintillation detector near the Hanford reactor reporting uncertain results \cite{CowanReinesFirstAttempt.pdf}. They later moved the detector to the Savannah River Plant and in 1956 confirmed \cite{CowanReinesConfirmation.pdf} the detection of the antineutrino, verifying the neutrino hypothesis \cite{NeutrinoPhysicsCowanReines.pdf}.
%(\ovnu{$\nu$} interacting with proton, producing neutron and positron (\ovnu{$\nu$}$+p\rightarrow n+e^+$))

% Electron neutrino
The first electron neutrino detection was by the Homestake neutrino experiment detecting solar neutrinos \cite{Homestake1968.pdf}
$\nu_e+n\rightarrow p+e^-$

%%% Muon neutrino %%%
Leon Lederman, Jack Steinberger and others joined Schwartz and using a spark chamber detector in 1962 observed \cite{MuonNeutrinoDetection.pdf} for the first time the muon neutrino $\nu_{\mu}$. Atmospheric neutrinos were first observed by the Kolar Gold Field Mine in South India \cite{FirstAtmosphericNuDetIndia.pdf, FirstAtmosphericNuDetIndia2.pdf} and in the East Rand Proprietary Gold Mine in South Africa \cite{FirstAtmNuDetectionSouthAfrica1965.pdf}.

% Tau neutrino
After this result it was only a matter of time, before the third neutrino, the tau neutrino ($\nu_{\tau}$) was discovered. Evidence for that were shown in 2000 from the DONUT Collaboration at Fermilab \cite{ObservationOfTauNeutrino.pdf}.
%%% End of master's on neutrino history
%%%%%%%%%%%%%%%%%%%%%%%%%%%%%%%%%%%%%%%%%%%%%%%%%%%%%%%%%%%%%%%%%%%%%%%%%%%%%%%

I think I should mention here the basic neutrino interactions and their corresponding cross section. For neutrino on nucleons, the total cross section per neutrino energy is around $\unit[0.7\times 10^{-38}]{cm^2GeV^{-1}}$ for neutrinos and half that for antineutrinos. For neutrino on electron interactions, the total cross section per neutrino energy is more similar to $10^-41-\unit[10^{-42}]{cm^2GeV^{-1}}$.

The main neutrino interactions are
\begin{equation}
\nu_l+n\rightarrow p+l^-
\end{equation}
\begin{equation}
\overline{\nu}_l+p\rightarrow n+l^+
\end{equation}
\begin{equation}
\nu_l+N\rightarrow\nu_l+N
\end{equation}
\begin{equation}
\overline{\nu}_l+N\rightarrow\overline{\nu}_l+N
\end{equation}


Problems of the SM
%[OverviewOfNeutrinoPhysicsPheno2024.pdf] Besides, the SM falls short in describing other challenging issues, including the long-standing problems of explaining the nature of dark matter [12] and the overabundance of matter over antimatter in the Universe [13]. These fundamental problems may be inter-related, with neutrinos potentially playing a primary role in establishing such connections

\todo{Describe neutrino interactions, CC, NC elastic. QE, Res., DIS. Also nuclear effects - MEC, FSI,...}

\section{Neutrino Oscillation}
Neutrino oscillate and therefore have mass. Describe neutrino oscillations and the current status of their measurements. Maybe also when they were discovered and how?



%Experimental evidence that lead to neutrino oscillations
[Master's] Several experimental indications for neutrino oscillations were found shortly after its theoretical predictions. Already in 1968 their Homestake Solar Neutrino Observatory saw a solar neutrino flux less than 3 Solar Neutrino Units (SNU = one interaction per $10^{36}$ target atoms $\unit{s^{-1}}$), well below the solar model prediction of the time \cite{Homestake1968.pdf}. This discrepancy became the \textit{“solar neutrino problem”}, which is in line with neutrino oscillations, but no direct implications could have been drawn since it might have been caused by a lack of understanding of nuclear physics, astrophysics of the Sun, or particle physics of the neutrino\cite{GoodmanAdvancesInNeutrinoPhysics.pdf}. Kamiokande experiment, which confirmed the results from Homestake \cite{Kamiokande96.pdf}.

The Solar neutrino anomaly was also resolved, when the Sudbury Neutrino Observatory (SNO) provided $>5\sigma$ evidence for  solar $\nu_e$ oscillations in 2002, independent on the solar model \cite{NCOscInSNOSecondOscResult.pdf}. While other solar neutrino experiments measured solar $\nu_e$ only via the charged current (CC) interactions
\begin{equation}
\nu_e+n\rightarrow p+e^- \qquad (CC),
\end{equation}
SNO had an ability to also detect neutrinos via the neutral current (NC) interaction
\begin{equation}
\nu+X\rightarrow\nu+X^{\prime} \qquad (NC),
\end{equation}
which are equally sensitive to all active neutrino flavours and their rate is therefore unaffected by standard neutrino oscillations. SNO could compare CC and NC event rates and conclude that $\nu_e$ from the Sun oscillate into other neutrino flavours along the way \cite{NCOscInSNOSecondOscResult.pdf}.

%Atmospheric neutrino experiments - atmospheric neutrino anomaly - how many neutrinos
Measuring atmospheric neutrinos brought about another neutrino conundrum, the \textit{Atmospheric neutrino anomaly}. It came from the disagreement between experiments such as NUSEX\cite{NUSEX89.pdf} and Fr\'{e}jus\cite{Frejus95.pdf}, which used iron calorimeters detectors, and experiments IMP\cite{IMP92.pdf} and Kamiokande\cite{Kamiokande94.pdf}, which used water Cherenkov detectors. All of these experiments were looking for a deficit of $\nu_{\mu}$, or an excess of $\nu_e$, compared to prediction. While the first two experiments saw a good agreement between experimental results and predictions, the latter two did not and suggested the possibility of neutrino oscillations, which could explain their disagreement. Solution to the Atmospheric neutrino anomaly came in 1998, when the Super-Kamiokande (SK) experiment showed for the first time the~experimental evidence for neutrino oscillations \cite{EvidenceForAtmoOscFirstEverOscRes.pdf}. SK has however also disfavoured the two neutrino hypothesis, with regards to the existence of an additional neutrino flavour. 


%%%%%%%%%%%%%%%%%%%%%%%%%%%%%%%%%%%%%%%%%%%%%%%%%%%%%%%%%%%%%%%%%%%%%%%%%%%%%%%
%%% Master's on neutrino oscillations

The idea that neutrinos can oscillate between the individual flavours originates \cite{Pontecorvo57.pdf,Pontecorvo58.pdf} from the $K^{0}\leftrightarrow \overline{K^0}$ oscillations, which was adapted to neutrinos \cite{MNS1962Osc.pdf,Pontecorvo69.pdf} by considering that the weak interaction neutrino eigenstates $\nu_\alpha$ produced in \gls{CC} weak interaction are not identical to the mass neutrino eigenstates $\nu_k$ and are related by
\begin{equation}
\ket{\nu_{\alpha}}=\sum_{k} U_{\alpha k}^{*}\ket{\nu_{k}}.
\end{equation}
Here $U$ is a unitary matrix now know by the authors of neutrino oscillations as \gls{PMNS} matrix \cite{FundamentalsOfNeutrinoPhysics.pdf, Gonzalez-GarciaNuMassesAndMixing.pdf}.

%This led B. Pontecorvo, inspired by already known $K^{0}\leftrightarrow \overline{K^0}$ oscillations, to consider $\nu\leftrightarrow\overline{\nu}$ transitions (oscillations), in case the conservation of neutrino charge does not apply\cite{Pontecorvo57.pdf}. Pontecorvo later built upon this statement in 1958 considering that oscillations between $\nu$ and $\overline{\nu}$ are due to them being combinations of particles $\nu_1$ and $\nu_2$ and that the transformation lifetime is related to the mass difference between $\nu_1$ and $\nu_2$ \cite{Pontecorvo58.pdf}, laying foundation for neutrino oscillations as we know them today.

%In 1962 Z. Maki, M. Nakagawa and S. Sakata applied Pontecorvo's idea of neutrino oscillations to \textit{weak neutrino} eigenstates $\nu_{\alpha}$ ($\nu_e$, $\nu_{\mu}$) produced in weak interactions. They assumed that oscillation $\nu_{\alpha}\leftrightarrow\nu_{\beta}$ are driven by a non-zero mass difference (therefore if true implying at least one neutrino has a non-zero mass) between \textit{true neutrinos} (= mass neutrino eigenstates $\nu_i$ ($\nu_1$, $\nu_2$)), which are related to weak eigenstates via a linear combination. This relation in general case looks like \cite{MNS1962Osc.pdf}
%where $U$ is a unitary matrix now know as Pontecorvo-Maki-Nakagawa-Sakata (PMNS) matrix and $n$ is the (general) number of light neutrino species \cite{Gonzalez-GarciaNuMassesAndMixing.pdf}.

The massive neutrino states $\ket{\nu_i}$ are eigenstates of the hamiltonian
\begin{equation}
\mathcal{H}\ket{\nu_k}=E_i\ket{\nu_k},
\end{equation}
where 
\begin{equation}
E_k=\sqrt{\overrightarrow{p}^2+m_k^2}
\end{equation}
and since neutrinos are generally ultrarelativistic, we can approximate their energy as
\begin{equation}\label{eq:NuOscUltrarelativisticApprox}
E_k\xrightarrow{m^2\ll p^2\approx E^2}E+\frac{m_k^2}{E}.
\end{equation}
From the Schrodinger equation
\begin{equation}
i\frac{d}{dt}\ket{\nu_k\left(t\right)}=\mathcal{H}\ket{\nu_k\left(t\right)}
\end{equation}
we get that the massive neutrino states evolve as plane waves 
\begin{equation}
\ket{\nu_k\left(t\right)}=e^{-iE_kt}\ket{\nu_k}.
\end{equation}
Also thanks to the unitarity of the mixing matrix we get
\begin{equation}
\ket{\nu_k}=\sum_\alpha U_{\alpha k}\ket{\nu_\alpha}
\end{equation}
and therefore
\begin{equation}
\ket{\nu_{\alpha}\left( t\right) }=\sum_{\beta} \sum_{k} U_{\alpha k}^{*} U_{\beta k} e^{-i E_kt} \ket{\nu_\beta}.
\end{equation}
Since both the massive neutrino states and the flavour neutrino states are built in an orthogonal basis $\braket{\nu_k|\nu_j}=\delta_{kj}$ and $\braket{\nu_{\alpha}|\nu_{\beta}}=\delta_{\alpha\beta}$ we can write the amplitude of the $\nu_\alpha\rightarrow\nu_\beta$ transitions as
\begin{equation}
A_{\nu_\alpha\rightarrow\nu_\beta}\left(t\right)\equiv\braket{\nu_\beta|\nu_\alpha\left(t\right)}= \sum_{k} U_{\alpha k}^{*} U_{\beta k} e^{-i E_kt}
\end{equation}

Given the ultrarelativistic approximation in Eq.~\ref{eq:NuOscUltrarelativisticApprox}, assuming the time $t$ is equivalent to the distance $L$, which is easier to measure in an experiment, we get the probability that $\nu_\alpha$ oscillates into $\nu_\beta$ over the course of distance $L$ with an energy $E$ is
\begin{align}
P_{\nu_{\alpha}\rightarrow\nu_{\beta}}\left( L\right) =
\left|A_{\nu_\alpha\rightarrow\nu_\beta}\left(t\right)\right|^2 &=
\sum_{k, j}U_{\alpha k}^*U_{\beta j}U_{\alpha j}U_{\beta j}^*e^{-i\left(E_k-E_j\right)L}\\
P_{\nu_{\alpha}\rightarrow\nu_{\beta}}\left( L\right) &= \sum_{k, j}U_{\alpha k}^*U_{\beta j}U_{\alpha j}U_{\beta j}^*e^{-i\frac{\Delta m_{kj}^2 L}{2E}}.
\end{align}

We defined the so-called neutrino mass splitting as
\begin{equation}\label{Deltamsq}
\Delta m_{kj}^{2}\equiv m_{k}^{2}-m_{j}^{2}.
\end{equation}

\iffalse
Oscillation probability can be also expressed as
\begin{align}\label{OscProb}
P_{\nu_{\alpha}\rightarrow\nu_{\beta}}\left( L\right)= \delta_{\alpha\beta}
& -4\sum_{i>j}\text{Re} \left( U_{\beta i}U_{\alpha i}^{*} U_{\beta j}^{*}U_{\alpha j}\right)
\sin^{2}\Delta_{ij}\nonumber \\
& +2\sum_{i>j} \text{Im} \left( U_{\beta i}U_{\alpha i}^{*}U_{\beta j}^{*}U_{\alpha j}\right) \sin 2\Delta_{ij},
\end{align}
where\cite{Gonzalez-GarciaNuMassesAndMixing.pdf} \[\Delta_{ij}\equiv\Delta m_{ij}^{2}\frac{L}{4E}=1.267\frac{\Delta m_{ij}^{2}}{\unit{eV^2}}\frac{L/E}{\unit{m}/\unit{MeV}} .\]

Since real neutrino beams are not monochromatic, what is measured in experiments is an \textbf{average} oscillation probability with $\left\langle \sin^{2}\Delta_{ij}\right\rangle$ and $\left\langle\sin2\Delta_{ij}\right\rangle $ in eq.\ref{OscProb}. We can notice that if $E/L\gg\Delta m_{ij}^{2}$ the oscillation does not show any effect yet and if $E/L\ll\Delta m_{ij}^2$ the oscillating phase goes through many cycles and is averaged to $\left\langle \sin^{2}\Delta_{ij}\right\rangle=1/2$. Therefore different experimental settings can measure different oscillation parameters \cite{PDG.pdf}.
\fi

The \gls{PMNS} matrix for mixing of three massive (Dirac) neutrinos with three neutrino flavours can be parametrized with four independent parameters: three mixing angles ($\theta_{12}$, $\theta_{13}, \theta_{23}$) and a $\delta_{CP}$ phase, which, if different from 0 or $\pi$, implies the \gls{CP} symmetry violation, and the other two are $\alpha$ and $\beta$, so called Majorana phases, which are non zero only if neutrinos are Majorana (neutrinos and antineutrinos are described by just one field, i.e. neutrinos are the same particle as antineutrinos). Majorana phases play no role in neutrino oscillations, so they are usually left out in the description \cite{PDG.pdf}. The PMNS matrix in this case can be parametrized as

\[
U=
\begin{pmatrix}
 U_{e1}     & U_{e2}     & U_{e3}    \\
 U_{\mu 1}  & U_{\mu 2}  & U_{\mu 3} \\
 U_{\tau 1} & U_{\tau 2} & U_{\tau 3}
\end{pmatrix}
=
\]
\begin{equation}\label{param3}
=
\begin{pmatrix}
 1 & 0       & 0      \\
 0 & c_{23}  & s_{23} \\
 0 & -s_{23} & c_{23}
\end{pmatrix}
\begin{pmatrix}
 c_{13}              & 0 & s_{13}e^{-i\delta} \\
 0                   & 1 & 0                  \\
 -s_{13}e^{i\delta} & 0 & c_{13}
\end{pmatrix}
\begin{pmatrix}
 c_{12}  & s_{12} & 0 \\
 -s_{12} & c_{12} & 0 \\
 0       & 0      & 1
\end{pmatrix}
\begin{pmatrix}
 1 & 0           & 0 \\
 0 & e^{i\alpha} & 0 \\
 0 & 0           & e^{i\beta}
\end{pmatrix},
\end{equation}
where $c_{ij}\equiv\cos\theta_{ij}$ and $s_{ij}\equiv\sin\theta_{ij}$.

Other than the PMNS matrix, neutrino oscillations depend on the mass squared differences (eq.\ref{Deltamsq}). In case of 3 neutrinos, those are $\Delta m^2_{21}$ and $\Delta m^2_{31}$. $\Delta m^2_{21}$ mainly drives oscillations of solar neutrinos and is therefore often denoted as $\Delta m^2_{\odot}$ or $\Delta m^2_{sol}$, while $\Delta m^2_{31}$ drives oscillations on the scale for atmospheric neutrinos  and is often written as $\Delta m^2_{atm}$ \cite{Gonzalez-GarciaNuMassesAndMixing.pdf}. There can only be two independent mass squared differences for oscillation of three neutrinos, since 
\begin{equation}
\Delta m^2_{21} + \Delta m^2_{32} + \Delta m^2_{13} = 0.
\end{equation}

%%%Matter effect
\subsection{The Matter Effect}
Possible explanation of the Solar neutrino problem was proposed in 1978 by L. Wolfenstein, who considered the effect of matter on neutrino oscillations \cite{Wolfenstein78.pdf}. His modification of neutrino oscillations when passing through matter arises from the coherent forward scattering of electron neutrinos, as a result of their charged current (CC) interaction with electrons, which are abundant in matter, as opposed to other lepton flavours, muons and tauons, resulting in an imbalance between $\nu_e$ and $\nu_{\mu}/\nu_{\tau}$. This manifests as an effective potential, which depends on the density and composition of the matter \cite{Wolfenstein78.pdf}. This idea was later further developed for neutrinos passing through the Sun by Mikheyev and Smirnov in 1985 \cite{MikheyevSmirnov85.pdf}\cite{Gonzalez-GarciaNuMassesAndMixing.pdf} and we now call this effect the Mikheyev-Smirnov-Wolfenstein (MSW) effect.

To showcase this effect we consider only two neutrino flavours, $\nu_e$ and $\nu_X$, where $X$ denotes a combination of all other non-electron flavours. Vacuum oscillations are in this two-neutrino approximation driven by a single mass splitting $\Delta m^2$ and the corresponding PMNS matrix is a rotational matrix parametrized by one angle $\theta$:
\begin{equation}
U=
\begin{pmatrix}
 \cos\theta  & \sin\theta    \\
 -\sin\theta & \cos\theta
\end{pmatrix}.
\end{equation}

The MSW effect can be described as the presence of an Effective Potential \cite{CERNSchool2001.pdf}
\begin{equation}
V=\pm\sqrt{2}G_{F}N_{e}=\pm 3.8\times 10^{-14}\left(\frac{\rho}{\unit{g}\ \unit{cm^{-3}}}\right)\left(\frac{Y_e}{0.5}\right)\unit{eV},
\end{equation}
where $G_F$ is the Fermi coupling constant, $N_e$ is the electron density, $Y_e$ is the electron number per nucleon and plus or minus sign is for neutrinos or antineutrinos respectively.

This potential can be seen as having the effect of modifying the $\Delta m^2$ and $\theta$ of the neutrino oscillations: \cite{CERNSchool2001.pdf}
\begin{equation}\label{MSWEffect}
\sin^22\theta_m=\frac{\sin^22\theta}{\sin^22\theta+\left(\cos 2\theta\mp\xi\right)^2}
\end{equation}
\begin{equation}
\left(\Delta m^2\right)_{\textsl{eff}}=\Delta m^2\times\sqrt{\sin^22\Theta+\left(\cos 2\theta\mp\xi\right)^2},
\end{equation}
where 
\begin{equation}
\xi=\frac{2\sqrt{2}G_FN_e}{\Delta m^2}.
\end{equation}
%%% End of master's on neutrino oscillations
%%%%%%%%%%%%%%%%%%%%%%%%%%%%%%%%%%%%%%%%%%%%%%%%%%%%%%%%%%%%%%%%%%%%%%%%%%%%%%%

%%% Current state of neutrino oscillation measurements
Need to mention the importance of the $\delta_{CP}$ measurements, maybe even the Sakharov conditions... Also why do we care about the $\Delta m^2$ or the octants and what are those... Maybe cite the Snowmass report and the PDG.

\section{Neutrino Mass}
Experiments for their values? Theoretical predictions for how they obtained them

Theories of neutrino mass generation

[Fundamentals of neutrinos physics and astrophysics]
The only extension of the SM that is needed is the introduction of right-handed components $\nu_{\alpha R}$ of the neutrino fields. Such a model is sometimes called the \textit{minimally extended Standard Model}. The right handed neutrino fields are fundamentally different from the other elementary fermion fields because they are invariant under the symmetries of the SM: they are \textbf{singlets} of $\textsf{SU}(3)_C\times\textsf{SU}(2)_L$ and have hypercharge $Y=0$. The right handed neutrino fields are called sterile [883] because they do not participate in weak interactions and their only interaction is gravitational. their right handedness is not required though! could also be left handed but have to be singlets and therefore sterile!

In the minimally extended standard model with three right handed neutrino fields, the SM Higgs-lepton Yukawa Lagrangian is extended by adding a lepton term with the same structure as the second term on the right handed side, which generates the masses of up-type quarks
\begin{equation}
\mathcal{L}_Y=
-\sum_{\alpha,\beta=e,\mu,\tau} Y_{\alpha\beta}^{\prime l} \overline{L}_{\alpha L}\Phi l_{\beta R}^{\prime}
-\sum_{\alpha,\beta=e,\mu,\tau} Y_{\alpha\beta}^{\prime \nu} \overline{L}_{\alpha L}\tilde{\Phi} \nu_{\beta R}^{\prime}
+\textsf{h.c.},
\end{equation}
where $Y^{\prime \nu}$ is a new matrix of Yukawa couplings.

Using the unitary gauge we can diagonalize the Yukawa couplings we obtain
\begin{equation}
\mathcal{L}_Y=
-\sum_{\alpha=e,\mu,\tau}\frac{y_\alpha^l v}{\sqrt{2}}\overline{l}_\alpha l_\alpha
-\sum_{k=1}^N \frac{y_k^\nu v}{\sqrt{2}}\overline{\nu}_k\nu_k
-\sum_{\alpha=e,\mu,\tau}\frac{y_\alpha^l}{\sqrt{2}}\overline{l}_\alpha l_\alpha H
-\sum_{k=1}^N \frac{y_k^\nu}{\sqrt{2}}\overline{\nu}_k\nu_k H
\end{equation}

Therefore the neutrino masses are given by
\begin{equation}
m_k=\frac{y_k^\nu v}{\sqrt{2}}\,\,\,\left(k=1,...,N\right),
\end{equation}
and massive Dirac neutrinos couple to the Higgs field through the last term. Note that the neutrinos masses are proportional to the Higgs VEV $v$, as the masses of charged leptons and quarks. However, it is known that the masses of neutrinos are much smaller than those of charged leptons and quarks, but there is no explanations here of the very small values of the eigenvalues $Y_k^{\nu}$ of the Higgs-neutrino Yukawa coupling matrix that are needed. The lagrangian defined this way does not conserve the lepton flavour number, which leads to neutrino oscillations. The Dirac character of massive neutrinos is closely related to the invariance of the total Lagrangian under the global U(1) gauge transformations.

The sterile neutrino fields do not participate in weak interaction with both their left and right components, but can couple with the ordinary neutrinos through the mass therm, generating a complicated mixing between active and sterile degrees of freedom.Since at present there is no indication of the existence of such additional sterile Dirac neutrino fields, ockham's razor suggests to ignore them...

%[nuMM/nuElmagInt2015.pdf] We now know that neutrinos are massive, because many experiments observed neutrino oscillations (Giunti and Kim, 2007; Bilenky, 2010; Xing and Zhou, 2011; Beringer et al., 2012; Gonzalez-Garcia et al., 2012; Bellini et al., 2014), which are generated by neutrino masses and mixing (Pontecorvo, 1957, 1958, 1968; Maki, Nakagawa, and Sakata, 1962). Therefore, the standard model must be extended to account for the neutrino masses. There are many possible extensions of the standard model that predict different properties for neutrinos (Ramond, 1999; Mohapatra and Pal, 2004; Xing and Zhou, 2011). Among them, most important is their fundamental Dirac or Majorana character. In many extensions of the standard model neutrinos also acquire electromagnetic properties through quantum loop effects which allow direct interactions of neutrinos with electromagnetic fields and electromagnetic interactions of neutrinos with charged particles.

\subsection{Majorana neutrinos}
[Fundamentals of neutrinos physics and astrophysics,p.190]
If the neutrino is massless, since the left handed chiral component of the neutrino field obeys the Weyl equation in both the Dirac and Majorana descriptions and the right handed chiral component is irrelevant for neutrino interactions, the Dirac and Majorana theories are physically equivalent. From this it is clear that in practice one can distinguish a Dirac from a Majorana neutrino only by measuring some effect due to the neutrino mass. Moreover, the mass effect must not be of kinematical nature, because the kinematical effects of Dirac and Majorana masses are the same. For example, the Dirac and Majorana nature of neutrinos cannot be revealed through neutrino oscillations! The most promising way to find if neutrinos are Majorana particles is the search for neutrinoless double beta decay.

[OverviewOfNeutrinoPhysicsPheno2024.pdf] In contrast, the Majorana phases do not enter the flavour neutrino oscillation probabilities [22, 85], but contribute to the $\beta\beta_{0\nu}$ decay rate