\chapter{Theory of neutrino physics}\label{sec:NeutrinoTheory}

Very brief history - Pauli, Fermi,...
Fermi was the first to use them in his beta decay theory, after Pauli proposed them in his letter. First time detected by Reines and Cowan in 1956.

Pauli introduced the idea of the neutrino as a neutral particle living inside a nucleus, with a possible magnetic moment and proton like mass. This properties were then divided between a neutron and a neutrino and neutrino was first used in a theory by Fermi. T

OverviewOfNeutrinoPhysicsPheno2024.pdf seems like a really good overview of basically everything that I want to cover in this chapter, including the mass generation mechanisms.

%[OverviewOfNeutrinoPhysicsPheno2024.pdf] Since W. Pauli’s original proposal of the neutrino idea [1], E. Fermi’s first description of the weak force [2], and the pioneering antineutrino detection with the Cowan-Reines experiment [3, 4], our understanding of neutrino properties and their interactions has continuously expanded.

%[nuMM/nuElmagInt2015.pdf] However, there was no sign of a neutrino mass. After the discovery of parity violation in 1957, Landau (1957), Lee and Yang (1957), and Salam (1957) proposed the two-component theory of massless neutrinos, in which a neutrino is described by a Weyl spinor and there are only left-handed neutrinos and right-handed antineutrinos. It was, however, clear (Case, 1957; Mclennan, 1957; Radicati and Touschek, 1957) that two-component neutrinos could be massive Majorana fermions and that the two-component theory of a massless neutrino is equivalent to the Majorana theory in the limit of zero neutrino mass. The two-component theory of massless neutrinos was later incorporated in the standard model of Glashow (1961), Weinberg (1967), and Salam (1969), in which neutrinos are massless and have only weak interactions. In the standard model Majorana neutrino masses are forbidden by the $\textsf{SU}\left(2\right)_L\times \textsf{U}\left(1\right)_{\gamma}$ symmetry.

\section{Neutrinos in the Standard Model}
Also say what is the standard model
Neutrinos are \textbf{massless} fermions, their interactions are...
Maybe discuss the parts of the SM that concern neutrinos

[OverviewOfNeutrinoPhysicsPheno2024.pdf]
three neutrino flavours produced with charged antilepton, or producing a charged lepton in CC weak interaction processes. Therefore neutrino have exhibited a polarisation in a direction that is opposite to their motion, or, equivalently, with negative helicity. For antineutrino it is the opposite. To account for this neutrino are described in the SM with a left handed chiral field $\nu_{\alpha L}\left(x\right)$. In the absence of neutrino masses, this field destroys (creates) neutrinos (antineutrinos) with negative (positive) helicity.  In the SM, neutrinos are considered massless, and no right-handed (RH) neutrino chiral field is included in its content as they would be singlet under the SM gauge group. As such, the corresponding RH neutrinos would be completely inert. Neutrinos and their lepton LH fields form SU(2) doublets $\psi_{\alpha L}\left(x\right)=\left(\nu_{\alpha L}\left(x\right),\alpha_L\left(x}\right)\right)$ with hypercharge +1.

%[nuMM/nuElmagInt2015.pdf] In the standard model of electroweak interactions (Glashow, 1961; Weinberg, 1967; Salam, 1969), neutrinos are described by two-component massless left-handed Weyl spinors (Giunti and Kim, 2007). The masslessness of neutrinos is due to the absence of right-handed neutrino fields, without which it is not possible to have Dirac mass terms, and to the absence of Higgs triplets, without which it is not possible to have Majorana mass terms.

Problems of the SM
%[OverviewOfNeutrinoPhysicsPheno2024.pdf] Besides, the SM falls short in describing other challenging issues, including the long-standing problems of explaining the nature of dark matter [12] and the overabundance of matter over antimatter in the Universe [13]. These fundamental problems may be inter-related, with neutrinos potentially playing a primary role in establishing such connections

\todo{Describe neutrino interactions, CC, NC elastic. QE, Res., DIS. Also nuclear effects - MEC, FSI,...}

\section{Neutrinos beyond the Standard Model}
Neutrino oscillate and therefore have mass.

%[nuMM/nuElmagInt2015.pdf] We now know that neutrinos are massive, because many experiments observed neutrino oscillations (Giunti and Kim, 2007; Bilenky, 2010; Xing and Zhou, 2011; Beringer et al., 2012; Gonzalez-Garcia et al., 2012; Bellini et al., 2014), which are generated by neutrino masses and mixing (Pontecorvo, 1957, 1958, 1968; Maki, Nakagawa, and Sakata, 1962). Therefore, the standard model must be extended to account for the neutrino masses. There are many possible extensions of the standard model that predict different properties for neutrinos (Ramond, 1999; Mohapatra and Pal, 2004; Xing and Zhou, 2011). Among them, most important is their fundamental Dirac or Majorana character. In many extensions of the standard model neutrinos also acquire electromagnetic properties through quantum loop effects which allow direct interactions of neutrinos with electromagnetic fields and electromagnetic interactions of neutrinos with charged particles.

Theories of neutrino mass generation

I should discuss everything that is even briefly mentioned in the neutrino magnetic moment theory section.
\begin{itemize}
\item Dirac vs Majorana neutrinos
\item Neutrino masses
\item Neutrino interactions with electrons and nuclei
\item Neutrino oscillations and their implications
\end{itemize}