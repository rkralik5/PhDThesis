\chapwithtoc{Preface}\label{sec:Preface}

This preface outlines the structure of this thesis and clarifies my contributions to the research presented in each chapter.

%This Preface provides an overview of the structure of this thesis and introduces the research presented in this thesis, highlighting my personal contributions to the work presented herein. I am detailing the resources used in each chapter, with adequate citations provided in the text. Throughout the thesis I resorted to using private information available solely to members of the NOvA experiment. This is to adequately highlight the contribution and research developed by my colleagues in the NOvA experiments. However, the private documentation has been limited to the absolute minimum and all effort has been made to use publicly available resources to allow for correct validation of resources.

Chapter~\ref{sec:NeutrinoTheory} provides a literature review of the current theoretical and experimental landscape of neutrino physics, with emphasis on neutrino interactions, oscillations, and mass generation mechanisms. The theory of neutrino electromagnetic interactions is also introduced, focusing on the necessary theoretical background for the measurement of the effective muon neutrino magnetic moment.
% relevant literature of neutrinos and their place in the particle physics theory and experiments. I am focusing on neutrino interactions and oscillations, as these are the main topics studied in the NOvA experiment and I provide the necessary theoretical foundation for the research presented in the subsequent chapters.
%I talk about the various source (Sec.~\ref{sec:NeutrinoProduction}) and detection (Sec.~\ref{sec:NeutrinoInteractions}) possibilities. In Sec.~\ref{sec:NeutrinoOscillation} I discuss the neutrino oscillation phenomenology and the current state of its experimental measurements. Section~\ref{sec:NuMass} discusses the possibilities for introducing the neutrino mass into the Standard Model and talks about their implications.

Chapter~\ref{sec:NOvA} introduces the NOvA experiment, primarily relying on publicly available resources, with internal NOvA documentation referenced only when greater technical detail is necessary. Throughout, I aim to ensure proper attribution to my colleagues' contributions while prioritising publicly accessible resources whenever possible. My contribution to this chapter includes analysing hadron production data from the NA61 experiment, which will be used in the future upgrades of the neutrino beam prediction in NOvA, as discussed in Sec.~\ref{sec:NOvASimulation}.% This chapter provides a comprehensive overview of the experimental setup technical aspects relevant to either the measurement of the effective muon neutrino magnetic moment, or to the calibration of the NOvA Test Beam detector. %This chapter is mainly a collection of work of my colleagues from the NOvA collaboration, their technical design reports for the NOvA experiment and official results.

Chapter~\ref{sec:DataBasedSimulation} presents the data-based simulation of cosmic muons used for calibration, originally created by my colleague Teresa Lackey which I significantly improved. My contributions include enhancing event selection, implementing energy and charge assignment, and producing and validating the resulting simulation samples.

Chapter~\ref{sec:TBCalibrationSection} covers the calibration of the NOvA Test Beam detector, a technical project I undertook to reduce systematic uncertainties within the NOvA experiment. The Test Beam detector calibration is based on the calibration framework used for the other NOvA detectors, which I adapted for the Test Beam detector. My work involved adapting the NOvA calibration framework to work with fibre brightness bins and recreating the Test Beam fibre brightness map and the threshold and shielding corrections. Ultimately, I completed and validated the first full calibration of the NOvA Test Beam detector.

Chapter~\ref{sec:NeutrinoMagMoment} details the measurement of the effective muon neutrino magnetic moment. This analysis was initially suggested to me by my colleague, Matt Strait. The data and simulation samples, including the enhanced and the systematically shifted simulation samples, were created by my colleagues, together with the radiative correction weight. My contributions include assisting in the development of the enhanced $\nu_e$CCMEC sample and the neutrino magnetic moment weight, eliminating the need for a dedicated simulation. I also developed the event selection and investigated the effect of systematic uncertainties using tools and methods adapted from other NOvA analyses. Additionally, I implemented statistical analysis, customizing a fitting framework originally developed for the NOvA light dark matter analysis.

Finally, chapter~\ref{sec:Conclusion} concludes with a summary of the findings of this thesis.

\iffalse
%%% PPFX
%\section*{Reducing the Neutrino Flux Systematic Uncertainty}

%To account for the inherently imprecise theoretical models used in the neutrino flux simulation models, NOvA experiments uses results of external measurements of hadrons interactions that result in the neutrino beam, as described in Sec.~\ref{sec:NOvASimulation}. Currently, NOvA only uses a limited number of the available external measurements, which results in a considerably large neutrino flux systematic uncertainty, especially for analyses using only the Near Detector (see Sec.~\ref{sec:NOvASystematics}). I worked on the implementation of the recent Kaon and Pion production measurements from $\unit[31]{GeV}$ protons interacting on a thin Carbon target, measured by the NA61 experiment~\cite{2015_hadron_prod_pC_2009data.pdf}. This data has the potential to significantly improve the high neutrino energy region, dominated by neutrinos produced by Kaon decays. However, NA61 made additional measurement of hadron incident interactions~\cite{2019_had_prod_at_Pi_on_C_and_Be.pdf}, and proton-on-carbon interactions with $\unit[120]{GeV}$ protons~\cite{NA61_hadprodFrompC_120GeV_2023.pdf, NA61_ResonanceProdFrompC_120GeV_2023.pdf} and with NOvA-replica target~\cite{ThickTargetLimit.pdf}, which will have even larger impact. Additionally, there is another experiment with dedicated measurements for the neutrino flux prediction improvement, named EMPHATIC~\cite{EMPHATICProposal2019.pdf}.  Therefore, due to the technical difficulty of updating the neutrino flux prediction, it was decided to postpone the update of the NOvA neutrino flux simulation until those data are available to us.


%For the most common $\pi$ production, PPFX uses the NA49 measurements \cite{NA49:Inclusive_production_of_charged_pions.pdf} of $\unit[158]{GeV/c}$ protons interacting on a thin (few percent of interaction length) carbon target, with a few data points from Barton et al~\cite{BartonHadProd1983.pdf} to expand the kinematic coverage. These then have to be scaled to the $\unit[20-120]{GeV/c}$ incident proton energies seen in NOvA using the FLUKA \cite{FLUKA_01,FLUKA_02} MC generator. For the $K$ production from $p+C$ interaction, important for higher neutrino energies and electron neutrinos, PPFX uses the NA49 $K$ data \cite{NA49DataKaons.pdf} together with the NA49 $\pi$ data \cite{NA49:Inclusive_production_of_charged_pions.pdf} multiplied by the $K/\pi$ ratios of yields on thin carbon target from the MIPP experiment \cite{pionToKaonIn_pC.pdf}. Lastly, for the nucleon production, PPFX uses the NA49 data on QE interactions \cite{NA49pc-proton2013.pdf}. All the other interactions inside NuMI, such as interaction in non-carbon targets, or interactions with hadrons other than protons, are either extrapolated from the previously mentioned measurements, or are not corrected for and a significant systematic uncertainty is assigned to them \cite{NuMIFlux.pdf}.

%There are two new experiments that measure the production and interaction of hadrons on various targets and incident energies, specifically designed to improve the prediction of neutrino beams. I worked on implementing data from the NA61 experiment on hadron production from $p+C$ interaction on a thin carbon target at $\unit[31]{GeV/c}$~\cite{2015_hadron_prod_pC_2009data.pdf}, motivated by possible reduction in the $K$ production systematic uncertainty. This work is still ongoing and will be implemented into PPFX and NOvA together with the rest of the NA61 measurement. The most impactful measurement will be of the hadron production from $p+C$ interaction on a thin carbon target at $\unit[120]{GeV/c}$ \cite{NA61_hadprodFrompC_120GeV_2023.pdf} (no energy scaling required), measurements of $p+C$ and $p+Be$ at different incident energies \cite{2019_NA61_ProdAndInelXSec_protonOnDiffTargets60And120GeV._results.pdf}, $\pi+C$ and $\pi+Be$ measurements at $\unit[60]{GeV/c}$~\cite{2019_had_prod_at_Pi_on_C_and_Be.pdf}, resonance production measurements from $\unit[120]{GeV/C}$ $p+C$ \cite{NA61_ResonanceProdFrompC_120GeV_2023.pdf}, and probably the most impactful one, the yet unpublished measurement of hadron production yield on a NOvA-era NuMI replica target at $\unit[120]{GeV/c}$ \cite{ThickTargetLimit.pdf}. NA61 also measured the hadron production yield for the T2K experiment's replica target \cite{2019_hadron_yields_T2K_replica.pdf}, which significantly reduced the neutrino flux systematic uncertainty for the T2K measurements \cite{ThickTargetLimit.pdf}. The second experiment is EMPHATIC~\cite{EMPHATICProposal2019.pdf}, which is currently analysing a broad range of hadron production measurements, mainly the secondary and tertiary interactions of various projectiles with a wide range of incident energies and thin target materials, complementary to the NA61 measurements.

\section*{Data-based Simulation of Cosmic Muons}
%What I had in the beginning:
 I also inherited the first-version of the data-based simulation of cosmic muons for the Test Beam detector. However, this version of the simulation was just directly taken from the simulation of the muon-removed sample and did not work properly for the Test Beam detector. 

%What I worked on:
I developed and validated the event selection of cosmic muons for the data-based simulation, implemented the energy and charge correction. I developed the data-based simulation of cosmic muons described in Sec. Specifically, I developed the event selection, implemented the energy and charge corrections, actually produced the simulation and the calibration samples, and validated them

%%% Test Beam calibration
\section*{Calibration of the NOvA Test Beam Detector}
%What I had in the beginning
The Test Beam detector calibration (Sec.~\ref{sec:TestBeamCalibration}) uses the same calibration techniques as are used for the other NOvA detectors. I inherited the Test Beam detector calibration when it was technically working, but not for simulation. Therefore, I was the first person to calibrate the simulated Test Beam detector and therefore complete the entire Test Beam detector calibration chain. This allowed for the first proper production of data for the Test Beam analysers. Starting from here, I included the fibre brightness for the Test Beam detector in the same way as is used for the Near and Far NOvA detectors (Sec.~\ref{sec:FibreBrightnessTB}).

%What I worked on
I implemented the NOvA calibration procedure in full for the Test Beam calibration, specifically adding the Fibre Brightness dependency and implementing it for all the data and simulation samples. I've created my own fibre brightness maps, my own threshold and shielding corrections (found out about the issue), improved the NOvA calibration code, made the attenuation fits for all the samples, did the same for absolute calibration, figured out there is a mistake in geometry, figured out we need to add the underfilled cells to the dead channels, changed the TS correction limits, improved the systematic uncertainty for the absolute calibration, validated the calibration, developed code for the validation


%%% Magnetic moment ana
\section*{Measurement of the Effective Muon Neutrino Magnetic Moment}
%What was the state in the beginning:
I was just told its an interesting analysis, there was a previous NOvA thesis, but it wasn't very well developed, did not use the standard NOvA techniques and did not properly incorporate systematics into their fits/limits. From the ND group, I got the event classifier and the event selection, which however did not work very well for our signal. Additionally, I got the nuone and the nueccmec enhanced samples and the radiative correction weight from them. From NOvA in general I got the nominal ND sample and the data sample, the cross section and PPFX weights and the fitting infrastructure in general. From the LDM analysis I got the general structure of the fitting framework.

%What I worked on
I have done the literature review of what is the neutrino magnetic moment and what are its current limits and state of the art measurements. I designed the neutrino magnetic moment weight and developed my own event selection, including using the TMVA, which is not used elsewhere, and the general analysis infrastructure for the neutrino magnetic moment ana. I helped (re)produce the systematic samples for the nueccmec enhanced sample. Technically I also investigated the nominal ND sample and whether it would be better to use the decaf sample, but this led nowhere... Technically I also analysed the electron recoil energy and angle resolution and decided on the binning (although not finished and talked about here). I did the systematics study and implemented the systematic shifts for the detector systematics for numm. I implemented the fitting framework for the neutrino magnetic moment analysis for both the template fit (not used here) and the counting experiment. I did the actual fit and got the final results.
\fi