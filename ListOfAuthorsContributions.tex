\chapwithtoc{List of Author's Contributions}\label{sec:ListOfAuthorsContributions}

%%% PPFX
\section*{Reducing the Neutrino Beam Systematic Uncertainty}
To account for the inherently imprecise theoretical models used in GEANT4, we use the PPFX to incorporate external measurements of yields and cross sections of hadron interactions inside the target and other NuMI materials into the prediction~\cite{NuMIFlux.pdf}. The current version of PPFX is limited by the results available during its creation and only corrects the most frequent interactions while assigning conservative systematic uncertainties to the rest (see Sec.~\ref{sec:NOvASystematics}). For the most common $\pi$ production, PPFX uses the NA49 measurements \cite{NA49:Inclusive_production_of_charged_pions.pdf} of $\unit[158]{GeV/c}$ protons interacting on a thin (few percent of interaction length) carbon target, with a few data points from Barton et al~\cite{BartonHadProd1983.pdf} to expand the kinematic coverage. These then have to be scaled to the $\unit[20-120]{GeV/c}$ incident proton energies seen in NOvA using the FLUKA \cite{FLUKA_01,FLUKA_02} MC generator. For the $K$ production from $p+C$ interaction, important for higher neutrino energies and electron neutrinos, PPFX uses the NA49 $K$ data \cite{NA49DataKaons.pdf} together with the NA49 $\pi$ data \cite{NA49:Inclusive_production_of_charged_pions.pdf} multiplied by the $K/\pi$ ratios of yields on thin carbon target from the MIPP experiment \cite{pionToKaonIn_pC.pdf}. Lastly, for the nucleon production, PPFX uses the NA49 data on QE interactions \cite{NA49pc-proton2013.pdf}. All the other interactions inside NuMI, such as interaction in non-carbon targets, or interactions with hadrons other than protons, are either extrapolated from the previously mentioned measurements, or are not corrected for and a significant systematic uncertainty is assigned to them \cite{NuMIFlux.pdf}.

There are two new experiments that measure the production and interaction of hadrons on various targets and incident energies, specifically designed to improve the prediction of neutrino beams. I worked on implementing data from the NA61 experiment on hadron production from $p+C$ interaction on a thin carbon target at $\unit[31]{GeV/c}$~\cite{2015_hadron_prod_pC_2009data.pdf}, motivated by possible reduction in the $K$ production systematic uncertainty. This work is still ongoing and will be implemented into PPFX and NOvA together with the rest of the NA61 measurement. The most impactful measurement will be of the hadron production from $p+C$ interaction on a thin carbon target at $\unit[120]{GeV/c}$ \cite{NA61_hadprodFrompC_120GeV_2023.pdf} (no energy scaling required), measurements of $p+C$ and $p+Be$ at different incident energies \cite{2019_NA61_ProdAndInelXSec_protonOnDiffTargets60And120GeV._results.pdf}, $\pi+C$ and $\pi+Be$ measurements at $\unit[60]{GeV/c}$~\cite{2019_had_prod_at_Pi_on_C_and_Be.pdf}, resonance production measurements from $\unit[120]{GeV/C}$ $p+C$ \cite{NA61_ResonanceProdFrompC_120GeV_2023.pdf}, and probably the most impactful one, the yet unpublished measurement of hadron production yield on a NOvA-era NuMI replica target at $\unit[120]{GeV/c}$ \cite{ThickTargetLimit.pdf}. NA61 also measured the hadron production yield for the T2K experiment's replica target \cite{2019_hadron_yields_T2K_replica.pdf}, which significantly reduced the neutrino flux systematic uncertainty for the T2K measurements \cite{ThickTargetLimit.pdf}. The second experiment is EMPHATIC~\cite{EMPHATICProposal2019.pdf}, which is currently analysing a broad range of hadron production measurements, mainly the secondary and tertiary interactions of various projectiles with a wide range of incident energies and thin target materials, complementary to the NA61 measurements.

%%% Test Beam calibration

what I worked on
\begin{itemize}
\item Should I talk about TB operations as well? - probably not
\item I developed the data-based simulation of cosmic muons described in Sec. Specifically, I developed the event selection, implemented the energy and charge corrections, actually produced the simulation and the calibration samples, and validated them
\item I implemented the NOvA calibration procedure in full for the Test Beam calibration, specifically adding the Fibre Brightness dependency and implementing it for all the data and simulation samples. I've created my own fibre brightness maps, my own threshold and shielding corrections (found out about the issue), improved the NOvA calibration code, made the attenuation fits for all the samples, did the same for absolute calibration, figured out there is a mistake in geometry, figured out we need to add the underfilled cells to the dead channels, changed the TS correction limits, improved the systematic uncertainty for the absolute calibration, validated the calibration, developed code for the validation
\end{itemize}

%%% Magnetic moment ana

what I worked on
\begin{itemize}
\item I scavenged the theory to find out what is neutrino magnetic moment, 
\end{itemize}