\chapter{Conclusion}\label{sec:Conclusion}
%Explicitly say what I have done in a very short summary

In this thesis, I presented the search for the effective muon neutrino magnetic moment by looking for an excess of \gls{nuone} elastic scattering events above the \gls{SM} background. Using the \gls{NOvA} \gls{ND} data collected between 2014 and 2021, corresponding to an exposure of $13.8\times10^{20}$~\gls{POT}, no excess was observed. A goodness-of-fit test for the \gls{SM}-only hypothesis yielded a p-value of 0.31. I placed an upper limit on the effective muon neutrino magnetic moment at $\mu_{\nu_\mu}<19.1\times 10^{-10}\mu_B$ at $90\%$~\gls{CL}. Future iterations of this analysis will enhance \gls{NOvA}'s sensitivity through increased statistics and a reduction in systematics uncertainties, especially in neutrino beam prediction and detector modelling.

The NOvA Test Beam experiment will be crucial in minimizing detector-related uncertainties. I described the first full calibration of the Test Beam detector, which is essential for applying the Test Beam data in reducing systematic uncertainties in \gls{NOvA}. This effort included the development of a dedicated simulation of cosmic muons applicable beyond the Test Beam detector calibration.