\chapter{Conclusion}\label{sec:Conclusion}
%Explicitly say what I have done in a very short summary

In this thesis, I presented the search for the effective muon neutrino magnetic moment by looking for an excess of \gls{nuone} elastic scattering events above the \gls{SM} background. Using \gls{NOvA} \gls{ND} data collected between 2014 and 2021, corresponding to an exposure of $13.8\times10^{20}$~\gls{POT}, no significant excess was observed. A goodness-of-fit test for the \gls{SM}-only hypothesis yielded a p-value of 0.31. I placed an upper limit on the effective muon neutrino magnetic moment at $\mu_{\nu_\mu}<19.1\times 10^{-10}\mu_B$ at $90\%$~\gls{CL}.

This limit is less stringent than both the current best limit for muon neutrinos of $\mu_{\nu_\mu}<6.8\times 10^{-10}\mu_B$  at $\unit[90]{\%}$ \gls{CL} \cite{LSNDLimits2001.pdf} and the previous \gls{NOvA} result of \mbox{$\mu_{\nu_\mu}<15.8\times 10^{-10}\mu_B$} at $\unit[90]{\%}$ \gls{CL} \cite{nuMM-thesis-biaow.pdf}. The difference is primarily due to a more careful treatment of backgrounds and systematic uncertainties, which reduced the significance of the result.

Future iterations of this analysis can improve \gls{NOvA}’s sensitivity through several key advancements. The most significant improvements will come from increased statistics, particularly by incorporating antineutrino data. Additionally, enhanced event selection, including lower-energy events and the use of dedicated control samples, will refine signal identification. Further improvements in the analysis methods and reduction of systematic uncertainties, particularly in neutrino beam prediction and detector modelling, will also strengthen future results.

%%% Test Beam detector calibration
The \gls{NOvA} Test Beam experiment plays a crucial role in reducing detector-related uncertainties by providing a controlled environment to study known particles with well-understood energies inside a small-scale \gls{NOvA} detector. This work presented the first complete calibration of the Test Beam detector - an essential step in leveraging Test Beam data to reduce systematic uncertainties in \gls{NOvA}. While most detector cells were successfully calibrated, some remained uncalibrated due to underfilled cells or dead channels. Several improvements could further enhance calibration performance for the Test Beam detector and across \gls{NOvA}.

Key areas for improvement include refining the threshold and shielding corrections, potentially making them entirely data-driven, to improve calibration accuracy and reducing cell-by-cell variations. Additionally, enabling per-cell calibration rather than relying on run and subrun numbers would help correct issues caused by faulty \glspl{FEB}, ensuring that temporarily malfunctioning cells are properly calibrated. This requires modifying input files and calibration procedures to allow for per-cell adjustments. The Test Beam detector also provides a unique opportunity to study environmental effects and scintillator ageing, which could help explain long-term drifts observed in \gls{NOvA} detectors. Future work should explore dividing the Test Beam calibration into shorter time periods while ensuring sufficient statistics for attenuation fits. Further improvements could come from optimizing attenuation profile binning to better match actual cell dimensions, enhancing overall calibration precision.

%%% Data-based simulation
This work included the development of a dedicated data-based simulation of cosmic muons. Compared to the \gls{CRY} \gls{MC} simulation used in other detectors, this new approach improves efficiency while avoiding bias from the input data. Although originally designed for Test Beam calibration, it has broader applications, including \gls{ND} and \gls{FD}  calibration and other cosmic ray studies. One limitation is that the simulation does not accurately model the energies of through-going muons. While this does not impact its use for calibration, it may be a constraint for other applications. However, future improvements could address this by incorporating real measurements of the cosmic muon energy distribution. Additional enhancements, such as improved track reconstruction and refined event selection, could further optimize the simulation.