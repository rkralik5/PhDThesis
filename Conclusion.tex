\chapter{Conclusion}\label{sec:Conclusion}
%Explicitly say what I have done in a very short summary

In this thesis we presented results of the search for the muon neutrino magnetic moment by searching for an excess of \gls{nuone} events above the \gls{SM} background. We used muon-neutrino dominated data from the \gls{NOvA} \gls{ND} collected between 2014 and 2021, corresponding to an exposure of $13.8\times10^{20}$~\gls{POT}. We did not find any excess of \gls{nuone} events with a goodness-of-fit test for the Standard Model-only hypothesis yielding a p-value of 0.31. We placed an upper limit on the effective muon neutrino magnetic moment at $\mu_{\nu_\mu}<19.1\times 10^{-10}\mu_B$ at $90\%$~\gls{CL}.

Multitude possibilities of improvements for the future iterations of this analysis will allow \gls{NOvA} to provide competitive results on the effective muon neutrino magnetic moment.

The uncertainties in this analysis are dominated by systematic uncertainties for neutrino beam prediction and detector modelling, as well as the statistical uncertainty of the predicted signal over the predicted background. The systematic uncertainties on the neutrino beam prediction are being tackled by improving the neutrino beam prediction with a novel set of data from the NA61 and EMPHATIC experiments.

The improvements for the systematic uncertainties on the detector modelling are relying on the results from the \gls{NOvA} Test Beam experiment. We presented an overview of the Test Beam detector and its calibration, which is an essential step in ensuring the applicability of the Test Beam detector data. We described the first full calibration of the \gls{NOvA} Test Beam detector, including generating a novel prediction of cosmic muons, as well as numerous improvements and future possibilities for the \gls{NOvA} calibration procedures.